\section{Reaction-Diffusion Equation} 
\label{sec:ReactionDiffusionEquations}

\subsection{Governing equations}
\label{subsec:ReactionDiffusionGoverningEquations}

The \emph{generalised reaction-diffusion
equation}\index{reaction-diffusion equation!generalised} on a domain
$\Omega\in\rntopology{n}$ with boundary
$\Gamma=\boundary{\Omega}\in\rntopology{n-1}$ in \OpenCMISS can be
stated as
\begin{equation}
  \addtolength{\fboxsep}{5pt}
  \boxed{
    \fnof{a}{\vectr{x}}\delby{\fnof{u}{\vectr{x},t}}{t}+
    \divergence{}{\pbrac{\fnof{\diffusivitytensor}{\vectr{x}}\gradient{}{\fnof{u}{\vectr{x},t}}}}+
    \fnof{b}{\vectr{x}}\fnof{r}{\fnof{u}{\vectr{x},t}}+\fnof{s}{\vectr{x},t}=0
  }
  \label{eqn:GeneralisedReactionDiffusionEquation}
\end{equation}
where $\vectr{x}\in\Omega$, $t\in[0,\infty)$, $\fnof{u}{\vectr{x},t}$
  is the vector quatity that diffuses and reacts,
  $\fnof{r}{\fnof{u}{\vectr{x},t}}$ is the reaction vector,
  $\fnof{a}{\vectr{x}}$ and $\fnof{b}{\vectr{x}}$ are material
  coefficients, $\fnof{\diffusivitytensor}{\vectr{x}}$ is the
  conductivity/diffusivity tensor throughout the domain and
  $\fnof{s}{\vectr{x},t}$ is a source term in the domain.
  
A specialisation of the reaction-diffusion equation is \emph{Fisher's
equation}\footnote{named after
\link{https://en.wikipedia.org/wiki/Ronald_Fisher}{Sir Ronald Aylmer
  Fisher} (1890-1962), a British polymath.}\index{Fisher's equation}, which can be used to
model population growth. Fisher's equation is
\begin{equation}
  \addtolength{\fboxsep}{5pt}
  \boxed{
    \delby{\fnof{u}{x,t}}{t}-D\deltwosqby{\fnof{u}{x,t}}{x}=c.\fnof{u}{x,t}\pbrac{1-\fnof{u}{x,t}}
  }
  \label{eqn:FishersEquation}
\end{equation}
where $c\geq 2$.

Fisher's equation is a special case of the generalised
reaction-diffusion equation with $a=1$, $\diffusivitytensorsymbol=-D$,
$b=-c$, and $\fnof{r}{u}=u\pbrac{1-u}$ and $s\equiv 0$. Fisher's
equation is also known as the Kolmogorov\footnote{named after
\link{https://en.wikipedia.org/wiki/Andrey_Kolmogorov}{Andrey
  Kolmogorov} (1903-1987), a Russian
mathematician.}–Petrovsky\footnote{named after
\link{https://en.wikipedia.org/wiki/Ivan_Petrovsky}{Ivan G. Petrovsky} (1901-1973), a Russian
mathematician.}–Piskunov\footnote{named after
\link{https://en.wikipedia.org/wiki/Nikolai_Piskunov}{Nikolai Piskunov} (1908-1977), a Russian
mathematician.} (KPP) equation or the Fisher-KPP
equation.

The \emph{generalised Fisher's equation}\index{Fisher's
  equation!generalised} can be found by extending
\eqnref{eqn:FishersEquation} to multiple dimensions \ie
\begin{equation}
  \addtolength{\fboxsep}{5pt}
  \boxed{
    \fnof{a}{\vectr{x}}\delby{\fnof{u}{\vectr{x},t}}{t}+
    \divergence{}{\pbrac{\fnof{\diffusivitytensor}{\vectr{x}}\gradient{}{\fnof{u}{\vectr{x},t}}}}
    +\fnof{b}{\vectr{x}}\fnof{u}{\vectr{x},t}+\fnof{c}{\vectr{x}}\sqbrac{\fnof{u}{\vectr{x},t}}^{2}
    +\fnof{s}{\vectr{x},t}=0
  }
  \label{eqn:GeneralisedFishersEquation}
\end{equation}

Appropriate boundary conditions conditions for the reaction-diffusion equation
are specification of Dirichlet boundary conditions on the solution,
$\fnof{g}{\vectr{x},t}$, \ie
\begin{equation}
  \fnof{u}{\vectr{x},t} = \fnof{g}{\vectr{x},t} \quad \vectr{x}\in\Gamma_{D},t\in[0,\infty)
  \label{eqn:ReactionDiffusionDirichletBC} 
\end{equation}
and/or Neumann conditions in terms of the solution flux in the normal
direction, $\fnof{h}{\vectr{x},t}$ \ie
\begin{equation}
  \fnof{q}{\vectr{x},t} = \dotprod{\pbrac{\fnof{\diffusivitytensor}{\vectr{x}}
      \gradient{}{\fnof{u}{\vectr{x},t}}}}{\fnof{\normal}{\vectr{x}}} =
  \fnof{h}{\vectr{x},t} \quad \vectr{x}\in\Gamma_{N},t\in[0,\infty)
  \label{eqn:ReactionDiffusionNeumannBC} 
\end{equation}
where $\fnof{q}{\vectr{x},t}$, is the flux in the normal direction, $\fnof{\normal}{\vectr{x}}$ is the normal
vector to the boudary and $\Gamma=\union{\Gamma_{D}}{\Gamma_{N}}$.

Appropriate initial conditions for the diffusion equation are the
specification of an initial value of the solution, $\fnof{i}{\vectr{x}}$, \ie
\begin{equation}
  \fnof{u}{\vectr{x},0} = \fnof{i}{\vectr{x}} \quad \vectr{x}\in\Omega.
  \label{eqn:ReactionDiffusionInitialC} 
\end{equation}

\subsection{Weak formulation}
\label{subsec:ReactionDiffusionWeakForm}

The corresponding weak form of \eqnref{eqn:GeneralisedReactionDiffusionEquation} can be found by
integrating over the spatial domain with a test function \ie
\begin{equation}
  \gint{\Omega}{}{\pbrac{a\delby{u}{t}+\divergence{}{\pbrac{\diffusivitytensor\gradient{}{u}}}+b\fnof{r}{u}+s}w}{\Omega}=0
  \label{eqn:ReactionDiffusionWeakForm1}
\end{equation}
where $w$ is a suitable spatial test function.

Applying the divergence theorem to \eqnref{eqn:ReactionDiffusionWeakForm1} gives
\begin{equation}
  \gint{\Omega}{}{\pbrac{a\delby{u}{t}}w}{\Omega}-
      \gint{\Omega}{}{\dotprod{\diffusivitytensor\gradient{}{u}}{\gradient{}{w}}}{\Omega}
      +\gint{\Gamma}{}{\pbrac{\dotprod{\diffusivitytensor\gradient{}{u}}{\normal}}w}{\Gamma}
      +\gint{\Omega}{}{b\fnof{r}{u}w}{\Omega}
      +\gint{\Omega}{}{sw}{\Omega}=0
  \label{eqn:ReactionDiffusionWeakForm2}
\end{equation}

\subsection{Tensor notation}
\label{subsec:ReactionDiffusionTensorNotation}

\Eqnref{eqn:ReactionDiffusionWeakForm2} can be written in tensor notation as
\begin{equation}
  \gint{\Omega}{}{a\dot{u}w}{\Omega}
  -\gint{\Omega}{}{G^{jk}\diffusivitytensorsymbol^{i}_{j}\covarderiv{u}{i}\covarderiv{w}{k}}{\Omega}
  +\gint{\Gamma}{}{G^{jk}\diffusivitytensorsymbol^{i}_{j}\covarderiv{u}{i}n_{k}w}{\Gamma}
  +\gint{\Omega}{}{b\fnof{r}{u}w}{\Omega}
  +\gint{\Omega}{}{sw}{\Omega}=0
  \label{eqn:ReactionDiffusionTensorForm1}
\end{equation}
or
\begin{equation}
  \gint{\Omega}{}{a\dot{u}w}{\Omega}
  -\gint{\Omega}{}{G^{jk}\diffusivitytensorsymbol^{i}_{j}\covarderiv{u}{i}\covarderiv{w}{k}}{\Omega}
  +\gint{\Gamma}{}{qw}{\Gamma}
  +\gint{\Omega}{}{b\fnof{r}{u}w}{\Omega}
  +\gint{\Omega}{}{sw}{\Omega}=0
  \label{eqn:ReactionDiffusionTensorForm1}
\end{equation}

\subsection{Finite element formulation}
\label{subsec:ReactionDiffusionFEMFormulation}

We can now discretise the spatial domain into finite elements \ie $\Omega=
\displaystyle{\bigcup_{e=1}^{E}}\Omega_{e}$ with
$\Gamma=\displaystyle{\bigcup_{f=1}^{F}}\Gamma_{f}$. 
\Eqnref{eqn:ReactionDiffusionTensorForm2} becomes
\begin{equation}
  \dsum_{e=1}^{E}\gint{\Omega_{e}}{}{a\dot{u}w}{\Omega}-
  \dsum_{e=1}^{E}\gint{\Omega_{e}}{}{G^{jk}\diffusivitytensorsymbol^{i}_{j}
    \covarderiv{u}{i}\covarderiv{w}{k}}{\Omega}+
  \dsum_{f=1}^{F}\gint{\Gamma_{f}}{}{qw}{\Gamma} +
  \dsum_{e=1}^{E}\gint{\Omega_{e}}{}{b\fnof{r}{u}w}{\Omega}=0
  \dsum_{e=1}^{E}\gint{\Omega_{e}}{}{sw}{\Omega}=0
  \label{eqn:ReactionDiffusionFEMForm}
\end{equation}

If we now assume that the dependent variable $u$ can be interpolated
separately in space and in time we can write
\begin{equation}
  \fnof{u}{\vectr{x},t}=\gbfn{n}{}{\vectr{x}}\fnof{\nodept{u}{n}}{t}
\end{equation}
or, in standard interpolation notation within an element,
\begin{equation}
  \fnof{u}{\vectr{\xi},t}=\gbfn{n}{\beta}{\vectr{\xi}}
  \fnof{\nodedof{u}{n}{\beta}}{t}\gsf{n}{\beta}
\end{equation}
where $\fnof{\nodedof{u}{n}{\beta}}{t}$ are the time varying nodal
degrees-of-freedom for node $n$, global derivative $\beta$,
$\gbfn{n}{\beta}{\vectr{\xi}}$ are the corresponding basis functions 
and $\gsf{n}{\beta}$ are the scale factors. 

We can also interpolate the other variables in a similar manner \ie
\begin{equation}
  \begin{split}
    \fnof{q}{\vectr{\xi},t} &= \gbfn{o}{\gamma}{\vectr{\xi}}\fnof{\nodedof{q}{o}{\gamma}}{t}\gsf{o}{\gamma} \\
    \fnof{s}{\vectr{\xi},t} &= \gbfn{p}{\delta}{\vectr{\xi}}\fnof{\nodedof{s}{p}{\delta}}{t}\gsf{p}{\delta} \\
    \fnof{\tilde{\diffusivitytensor}}{\vectr{\xi}} &=\gbfn{l}{\epsilon}{\vectr{\xi}}
    \nodedof{\tilde{\diffusivitytensor}}{l}{\epsilon}\gsf{l}{\epsilon} \\
    \fnof{a}{\vectr{\xi}} &=\gbfn{v}{\zeta}{\vectr{\xi}}\nodedof{a}{v}{\zeta}\gsf{v}{\zeta} \\
    \fnof{b}{\vectr{\xi}} &=\gbfn{w}{\eta}{\vectr{\xi}}\nodedof{b}{w}{\eta}\gsf{w}{\eta} 
  \end{split}
\end{equation}
where $\fnof{\nodedof{q}{o}{\gamma}}{t}$,
$\fnof{\nodedof{s}{p}{\delta}}{t}$,
$\nodedof{\tilde{\diffusivitytensor}}{l}{\epsilon}$,
$\nodedof{a}{v}{\zeta}$, and $\nodedof{b}{w}{\eta}$ are the nodal
degrees-of-freedom for the flux, source, diffusivity, and material
variables.

For a Galerkin finite element formulation we also choose the spatial weighting
function $w$ to be equal to the basis fucntions \ie
\begin{equation}
  \fnof{w}{\vectr{\xi}}=\gbfn{m}{\alpha}{\vectr{\xi}}\gsf{m}{\alpha}
\end{equation}

\subsection{Spatial integration}
\label{subsec:ReactionDiffusionSpatialIntegration}

Adopting the standard integration notation we can write the spatial
integration term in \eqnref{eqn:ReactionDiffusionFEMForm} as
\begin{multline}
  \dsum_{e=1}^{E}\gint{\vectr{0}}{\vectr{1}}{\fnof{a}{\vectr{\xi}}
    \delby{\pbrac{\gbfn{n}{\beta}{\vectr{\xi}}\fnof{\nodedof{u}{\beta}{n}}{t}
        \gsf{n}{\beta}}}{t}\gbfn{m}{\alpha}{\vectr{\xi}}\gsf{m}{\alpha}
    \abs{\fnof{\matr{J}}{\vectr{\xi}}}}{\vectr{\xi}} \\
  -\dsum_{e=1}^{E}\dintl{\vectr{0}}{\vectr{1}}G^{jk}\delby{x^{i}}{\xi^{d}}\delby{\xi^{e}}{x^{j}}
  \delby{\xi^{d}}{\nu^{a}}\delby{\nu^{b}}{\xi^{e}}\fnof{{\tilde{\diffusivitytensorsymbol}}^{a}_{.b}}{\vectr{\xi}}
  \delby{\xi^{s}}{x^{i}}
  \quad\quad\quad\quad\quad\quad\quad\quad\quad\quad \\ \quad\quad\quad\quad\quad\quad\quad\quad\quad\quad
  \delby{\pbrac{\gbfn{\beta}{n}{\vectr{\xi}}\fnof{\nodedof{u}{\beta}{n}}{t}
      \gsf{\beta}{n}}}{\xi^{s}}\delby{\xi^{r}}{x^{k}}\delby{
    \pbrac{\gbfn{\alpha}{m}{\vectr{\xi}}\gsf{\alpha}{m}}}{\xi^{r}}
  \abs{\fnof{\matr{J}}{\vectr{\xi}}}d\vectr{\xi} \\
  +\dsum_{f=1}^{F}\gint{\vectr{0}}{\vectr{1}}{\gbfn{o}{\gamma}{\vectr{\xi}}
    \fnof{\nodedof{q}{o}{\gamma}}{t}\gsf{o}{\gamma}\gbfn{m}{\alpha}{\vectr{\xi}}
    \gsf{m}{\alpha}\abs{\fnof{\matr{J}}{\vectr{\xi}}}}{\vectr{\xi}} \\
  +\dsum_{e=1}^{E}\gint{\Omega_{e}}{}{\fnof{b}{\vectr{\xi}}\fnof{r}{\fnof{u}{\vectr{\xi},t}}\gbfn{m}{\alpha}{\vectr{\xi}}
    \gsf{m}{\alpha}\abs{\fnof{\matr{J}}{\vectr{\xi}}}}{\vectr{\xi}} \\
  +\dsum_{e=1}^{E}\gint{\vectr{0}}{\vectr{1}}{\gbfn{p}{\delta}{\vectr{\xi}}\fnof{\nodedof{s}{p}{\delta}}{t}\gsf{p}{\delta}\gbfn{m}{\alpha}{\vectr{\xi}}
    \gsf{m}{\alpha}\abs{\fnof{\matr{J}}{\vectr{\xi}}}}{\vectr{\xi}} = 0
  \label{eqn:ReactionDiffusionSpatialIntegration1}
\end{multline}
where $\fnof{\matr{J}}{\vectr{\xi}}$ is the \emph{Jacobian} of the
transformation from the integration $\vectr{x}$ to $\vectr{\xi}$ coordinates.

Taking values that are constant over the integration interval outside the
integration gives
\begin{multline}
  \dsum_{e=1}^{E}\fnof{\nodedof{\dot{u}}{\beta}{n}}{t}\gsf{m}{\alpha}\gsf{n}{\beta}
  \gint{\vectr{0}}{\vectr{1}}{\fnof{a}{\vectr{\xi}}
    \gbfn{m}{\alpha}{\vectr{\xi}}\gbfn{n}{\beta}{\vectr{\xi}}
    \abs{\fnof{\matr{J}}{\vectr{\xi}}}}{\vectr{\xi}} \\
  -\dsum_{e=1}^{E}\fnof{\nodedof{u}{\beta}{n}}{t}\gsf{\alpha}{m}\gsf{\beta}{n}
  \gint{\vectr{0}}{\vectr{1}}{\delby{\gbfn{\alpha}{m}{\vectr{\xi}}}{\xi^{r}}
  \delby{\gbfn{\beta}{n}{\vectr{\xi}}}{\xi^{s}}\fnof{\gamma^{rs}}{\vectr{\xi}}
  \abs{\fnof{\matr{J}}{\vectr{\xi}}}}{\vectr{\xi}} \\
  +\dsum_{f=1}^{F}\fnof{\nodedof{q}{o}{\gamma}}{t}\gsf{m}{\alpha}\gsf{o}{\gamma}
  \gint{\vectr{0}}{\vectr{1}}{\gbfn{m}{\alpha}{\vectr{\xi}}\gbfn{o}{\gamma}{\vectr{\xi}}
    \abs{\fnof{\matr{J}}{\vectr{\xi}}}}{\vectr{\xi}} \\
  +\dsum_{e=1}^{E}\gbfn{m}{\alpha}{\vectr{\xi}}
  \gsf{m}{\alpha}\gint{\Omega_{e}}{}{\fnof{b}{\vectr{\xi}}\fnof{r}{\fnof{u}{\vectr{\xi},t}}
    \abs{\fnof{\matr{J}}{\vectr{\xi}}}}{\vectr{\xi}} \\
  +\dsum_{e=1}^{E}\fnof{\nodedof{s}{p}{\delta}}{t}\gsf{m}{\alpha}\gsf{p}{\delta}
  \gint{\vectr{0}}{\vectr{1}}{\gbfn{m}{\alpha}{\vectr{\xi}}\gbfn{p}{\delta}{\vectr{\xi}}
    \abs{\fnof{\matr{J}}{\vectr{\xi}}}}{\vectr{\xi}} = 0
  \label{eqn:ReactionDiffusionSpatialIntegration2}
\end{multline}
where $\fnof{\gamma^{rs}}{\vectr{\xi}}$ is defined in 
\eqnthrurefs{eqn:DiffusionOperatorGammaDefinition1}{eqn:DiffusionOperatorGammaDefinition3}.

This is an equation of the form
\begin{equation}
  \matr{C}\fnof{\dot{\vect{u}}}{t}+\matr{K}\fnof{\vect{u}}{t}+\fnof{\vectr{g}}{\fnof{\vect{u}}{t}}+\fnof{\vect{f}}{t}=\vect{0}
  \label{eqn:ReactionDiffusionVectorEquations}
\end{equation}
where
\begin{equation}
  \fnof{\vect{f}}{t}=\matr{N}\fnof{\vect{q}}{t}+\matr{R}\fnof{\vect{s}}{t}
\end{equation}

The elemental damping matrix, $C^{\alpha\beta}_{mn}$, is given by
\begin{equation}
  C^{\alpha\beta}_{mn} =
  \gsf{m}{\alpha}\gsf{n}{\beta}\gint{\vectr{0}}{\vectr{1}}{\fnof{a}{\vectr{\xi}}
    \gbfn{m}{\alpha}{\vectr{\xi}}\gbfn{n}{\beta}{\vectr{\xi}}
    \abs{\fnof{\matr{J}}{\vectr{\xi}}}}{\vectr{\xi}}
\end{equation}

The elemental stiffness matrix, $K^{\alpha\beta}_{mn}$, is given by
\begin{equation}
  K^{\alpha\beta}_{mn} = -\gsf{m}{\alpha}\gsf{n}{\beta}\gint{\vectr{0}}{\vectr{1}}{
    \delby{\gbfn{\alpha}{m}{\vectr{\xi}}}{\xi^{r}}\delby{\gbfn{\beta}{m}{\vectr{\xi}}}{\xi^{s}}
    \fnof{\gamma^{rs}}{\vectr{\xi}}\abs{\fnof{\matr{J}}{\vectr{\xi}}}}{\vectr{\xi}}
\end{equation}

The elemental residual vector, $g^{\alpha}_{m}$, is given by
\begin{equation}
  g^{\alpha}_{m} =
  \gsf{m}{\alpha}\gint{\vectr{0}}{\vectr{1}}{\fnof{b}{\vectr{\xi}}
    \fnof{r}{\fnof{u}{\vectr{\xi},t}}
    \gbfn{m}{\alpha}{\vectr{\xi}}
    \abs{\fnof{\matr{J}}{\vectr{\xi}}}}{\vectr{\xi}}
\end{equation}

The elemental flux matrix, $N^{\alpha\gamma}_{mo}$, is given by
\begin{equation}
  N^{\alpha\gamma}_{mo} =\gsf{m}{\alpha}\gsf{o}{\gamma}
  \gint{\vectr{0}}{\vectr{1}}{\gbfn{m}{\alpha}{\vectr{\xi}}\gbfn{o}{\gamma}{\vectr{\xi}}
    \abs{\fnof{\matr{J}}{\vectr{\xi}}}}{\vectr{\xi}}
\end{equation}
and the elemental source matrix, $R^{\alpha\delta}_{mp}$, 
\begin{equation}
  R^{\alpha\delta}_{mp}=\gsf{m}{\alpha}\gsf{p}{\delta}
  \gint{\vectr{0}}{\vectr{1}}{\gbfn{m}{\alpha}{\vectr{\xi}}\gbfn{p}{\delta}{\vectr{\xi}}
    \abs{\fnof{\matr{J}}{\vectr{\xi}}}}{\vectr{\xi}}
\end{equation}

\subsection{Operator splitting}
\label{subsec:ReactionDiffusionOperatorSplitting}

\Eqnref{eqn:ReactionDiffusionVectorEquations} can be solved
using standard (nonlinear) time integration solvers. An alternative
solution method for solving the reaction-diffusion problems is to use
operator splitting (see
\subsubsecref{subsubsec:DynamicOperatorSplitting}). Operator splitting
is particularily useful if the reaction term involves dynamics at a
different time-scale to that of the diffusion dynamics. Operator
splitting allows for the decoupling of the reaction dynamics for the
diffusion dynamics.

The generalised reaction-diffusion equation in
\eqnref{eqn:GeneralisedReactionDiffusionEquation} can be written as
the sum of two operators \ie
\begin{equation}
  \delby{\fnof{u}{\vectr{x},t}}{t}=\fnof{A}{\fnof{u}{\vectr{x},t}}+\fnof{B}{\fnof{u}{\vectr{x},t}}
  \label{eqn:ReactionDiffusionOperatorForm}
\end{equation}
where $\fnof{A}{\cdot}$ and $\fnof{B}{\cdot}$ are two operators given by
\begin{align}
  \fnof{A}{\fnof{u}{\vectr{x},t}}&=\dfrac{\divergence{}{\pbrac{\fnof{\diffusivitytensor}{\vectr{x}}
        \gradient{}{\fnof{u}{\vectr{x},t}}}}+\fnof{s}{\vectr{x},t}}{\fnof{a}{\vectr{x}}} \\
  \fnof{B}{\fnof{u}{\vectr{x},t}}&=\dfrac{\fnof{b}{\vectr{x}}\fnof{r}{\fnof{u}{\vectr{x},t}}}{\fnof{a}{\vectr{x}}}
\end{align}

To solve the initial value problem in
\eqnref{eqn:ReactionDiffusionOperatorForm} with the initial condition
$\fnof{u}{\vectr{x},0}=\fnof{u_{0}}{\vectr{x}}$ we need to
integrate. Condsider a single integration time step \ie from $t=0$ to
$t=\Delta t$. Operator splitting breaks this integration step into a
sequence of integration steps involving each operator.

For \emph{Gudunov splitting} 

\subsection{Analytic solutions}
\label{subsec:ReactionDiffusionAnalyticSolutions}

\subsubsection{One Dimensional Analytic Function 1}

This analytic solution is a solution to Fisher's equation as given in
\eqnref{eqn:FishersEquation} with $D=1$. From
\urllink{https://eqworld.ipmnet.ru/en/solutions/npde/npde1101.pdf} a
solution to
\begin{equation}
  \delby{u}{t}-\deltwosqby{u}{x}=c.u\pbrac{1-u}
  \label{eqn:AnalyticReactionDiffusionFishersEquation}
\end{equation}
is
\begin{equation}
  \fnof{u}{x,t}=\dfrac{1}{\sqbrac{1+Ae^{\frac{-5}{6}ct\pm\frac{1}{6}\sqrt{6c}x}}^{2}}
  \label{eqn:AnalyticReactionDiffusionEquationOneDim1}
\end{equation}
where $A$ is an arbitrary constant.

To prove the analytic solution we differentiate
\eqnref{eqn:AnalyticReactionDiffusionEquationOneDim1}. Consider
defining $\fnof{v}{w}=1+Ae^{\fnof{w}{x,t}}$, $\fnof{w}{x,t}=\alpha t +
\beta x$, $\alpha=\dfrac{-5c}{6}$, and
$\beta=\pm\dfrac{1}{6}\sqrt{6c}$. \Eqnref{eqn:AnalyticReactionDiffusionEquationOneDim1}
is thus $\fnof{u}{v}=\fnof{v}{w}^{-2}$ and so we have for the temporal derivative
\begin{align}
  \delby{\fnof{u}{x,t}}{t}&=\delby{\fnof{u}{v}}{v}\delby{\fnof{v}{w}}{w}\delby{\fnof{w}{x,t}}{t} \\
  &=\dfrac{-2}{\fnof{v}{w}^{3}}.Ae^{\fnof{w}{x,t}}.\alpha \\
  &=\dfrac{-2\alpha A e^{\fnof{w}{x,t}}}{\sqbrac{1+A e^{\fnof{w}{x,t}}}^{3}}
\end{align}

If we now consider
$\delby{\fnof{u}{w}}{t}=\dfrac{\fnof{y}{w}}{\fnof{z}{w}}$ where
$\fnof{y}{w}=-2\beta A e^{\fnof{w}{x,t}}$ and
$\fnof{z}{w}=\sqbrac{1+A e^{\fnof{w}{x,t}}}^{3}$ we thus have
\begin{align}
  \deltwosqby{\fnof{u}{x,t}}{t}&=\delby{}{t}\pbrac{\delby{\fnof{u}{x,t}}{x}}=\delby{}{t}\pbrac{\dfrac{\fnof{y}{w}}{\fnof{z}{w}}}\\
  &=\dfrac{\fnof{z}{w}\delby{\fnof{y}{w}}{w}\delby{\fnof{w}{x,t}}{t}-
    \fnof{y}{w}\delby{\fnof{z}{w}}{w}\delby{\fnof{w}{x,t}}{t}}{\fnof{z}{w}^{2}} \\
  &=\dfrac{\sqbrac{1+A e^{\fnof{w}{x,t}}}^{3}.-2\alpha A e^{\fnof{w}{x,t}}.\alpha-
    \pbrac{-2\alpha A e^{\fnof{w}{x,t}}}.3\sqbrac{1+A e^{\fnof{w}{x,t}}}^{2}.A e^{\fnof{w}{x,t}}.\alpha}{\sqbrac{1+A e^{\fnof{w}{x,t}}}^{6}} \\
  &=\dfrac{-2\alpha^{2}A e^{\fnof{w}{x,t}}\sqbrac{1+A e^{\fnof{w}{x,t}}}^{3}+
    6\alpha^{2}\pbrac{A e^{\fnof{w}{x,t}}}^{2}\sqbrac{1+A e^{\fnof{w}{x,t}}}^{2}}{\sqbrac{1+A e^{\fnof{w}{x,t}}}^{6}}\\
  &=\dfrac{2\alpha^{2}A e^{\fnof{w}{x,t}}\sqbrac{1+A e^{\fnof{w}{x,t}}}^{2}\pbrac{3A e^{\fnof{w}{x,t}}-
      \sqbrac{1+A e^{\fnof{w}{x,t}}}}}{\sqbrac{1+A e^{\fnof{w}{x,t}}}^{6}} \\
  &=\dfrac{2\alpha^{2}A e^{\fnof{w}{x,t}}\pbrac{2A e^{\fnof{w}{x,t}}-1}}{\sqbrac{1+A e^{\fnof{w}{x,t}}}^{4}}
\end{align}

For the spatial derivative we have
\begin{align}
  \delby{\fnof{u}{x,t}}{x}&=\delby{\fnof{u}{v}}{v}\delby{\fnof{v}{w}}{w}\delby{\fnof{w}{x,t}}{x} \\
  &=\dfrac{-2}{\fnof{v}{w}^{3}}.Ae^{\fnof{w}{x,t}}.\beta \\
  &=\dfrac{-2\beta A e^{\fnof{w}{x,t}}}{\sqbrac{1+A e^{\fnof{w}{x,t}}}^{3}}
\end{align}

If we now consider
$\delby{\fnof{u}{w}}{x}=\dfrac{\fnof{y}{w}}{\fnof{z}{w}}$ where
$\fnof{y}{w}=-2\beta A e^{\fnof{w}{x,t}}$ and
$\fnof{z}{w}=\sqbrac{1+A e^{\fnof{w}{x,t}}}^{3}$ we thus have
\begin{align}
  \deltwosqby{\fnof{u}{x,t}}{x}&=\delby{}{x}\pbrac{\delby{\fnof{u}{x,t}}{x}}=\delby{}{x}\pbrac{\dfrac{\fnof{y}{w}}{\fnof{z}{w}}}\\
  &=\dfrac{\fnof{z}{w}\delby{\fnof{y}{w}}{w}\delby{\fnof{w}{x,t}}{x}-
    \fnof{y}{w}\delby{\fnof{z}{w}}{w}\delby{\fnof{w}{x,t}}{x}}{\fnof{z}{w}^{2}} \\
  &=\dfrac{\sqbrac{1+A e^{\fnof{w}{x,t}}}^{3}.-2\beta A e^{\fnof{w}{x,t}}.\beta-
    \pbrac{-2\beta A e^{\fnof{w}{x,t}}}.3\sqbrac{1+A e^{\fnof{w}{x,t}}}^{2}.A e^{\fnof{w}{x,t}}.\beta}{\sqbrac{1+A e^{\fnof{w}{x,t}}}^{6}} \\
  &=\dfrac{-2\beta^{2}A e^{\fnof{w}{x,t}}\sqbrac{1+A e^{\fnof{w}{x,t}}}^{3}+
    6\beta^{2}\pbrac{A e^{\fnof{w}{x,t}}}^{2}\sqbrac{1+A e^{\fnof{w}{x,t}}}^{2}}{\sqbrac{1+A e^{\fnof{w}{x,t}}}^{6}}\\
  &=\dfrac{2\beta^{2}A e^{\fnof{w}{x,t}}\sqbrac{1+A e^{\fnof{w}{x,t}}}^{2}\pbrac{3A e^{\fnof{w}{x,t}}-
      \sqbrac{1+A e^{\fnof{w}{x,t}}}}}{\sqbrac{1+A e^{\fnof{w}{x,t}}}^{6}} \\
  &=\dfrac{2\beta^{2}A e^{\fnof{w}{x,t}}\pbrac{2A e^{\fnof{w}{x,t}}-1}}{\sqbrac{1+A e^{\fnof{w}{x,t}}}^{4}}
\end{align}

The reaction term is given by
\begin{align}
  c\fnof{u}{x,t}\pbrac{1-\fnof{u}{x,t}}&=c\fnof{u}{v}\pbrac{1-\fnof{u}{v}}=
  \dfrac{c}{\fnof{v}{w}^{2}}\pbrac{1-\dfrac{1}{\fnof{v}{w}^{2}}}=\dfrac{c\pbrac{\fnof{v}{w}^{2}-1}}{\fnof{v}{w}^{4}}\\
  &=\dfrac{c\pbrac{\sqbrac{1+A e^{\fnof{w}{x,t}}}^{2}-1}}{\sqbrac{1+A e^{\fnof{w}{x,t}}}^{4}}\\
  &=\dfrac{c\pbrac{1+2A e^{\fnof{w}{x,t}}+\pbrac{A e^{\fnof{w}{x,t}}}^{2}-1}}{\sqbrac{1+A e^{\fnof{w}{x,t}}}^{4}}\\
  &=\dfrac{cA e^{\fnof{w}{x,t}}\pbrac{2+A e^{\fnof{w}{x,t}}}}{\sqbrac{1+A e^{\fnof{w}{x,t}}}^{4}}
\end{align}

In summary, the derivatives of
\eqnref{eqn:AnalyticReactionDiffusionEquationOneDim1} are thus given
by
\begin{equation}
  \begin{aligned}
    \delby{u}{t}&=\dfrac{-2\alpha A e^{\fnof{w}{x,t}}}{\sqbrac{1+A e^{\fnof{w}{x,t}}}^{3}}\\
    &=\dfrac{5 c A e^{\frac{-5}{6}ct\pm\frac{1}{6}\sqrt{6c}x}}{3\sqbrac{1+A e^{\frac{-5}{6}ct\pm\frac{1}{6}\sqrt{6c}x}}^{3}}\\
    \deltwosqby{u}{t}&=\dfrac{2\alpha^{2}A e^{\fnof{w}{x,t}}\pbrac{2A e^{\fnof{w}{x,t}}-1}}{\sqbrac{1+A e^{\fnof{w}{x,t}}}^{4}}\\
    &=\dfrac{25 c^{2} A e^{\frac{-5}{6}ct\pm\frac{1}{6}\sqrt{6c}x}\pbrac{2A e^{\frac{-5}{6}ct\pm\frac{1}{6}\sqrt{6c}x}-1}}{18\sqbrac{1+
        A e^{\frac{-5}{6}ct\pm\frac{1}{6}\sqrt{6c}x}}^{3}}\\
    \delby{u}{x}&=\dfrac{-2\beta A e^{\fnof{w}{x,t}}}{\sqbrac{1+A e^{\fnof{w}{x,t}}}^{3}}\\
    &=\dfrac{\mp\sqrt{6c} A e^{\frac{-5}{6}ct\pm\frac{1}{6}\sqrt{6c}x}}{3\sqbrac{1+A e^{\frac{-5}{6}ct\pm\frac{1}{6}\sqrt{6c}x}}^{3}}\\
    \deltwosqby{u}{x}&=\dfrac{2\beta^{2}A e^{\fnof{w}{x,t}}\pbrac{2A e^{\fnof{w}{x,t}}-1}}{\sqbrac{1+A e^{\fnof{w}{x,t}}}^{4}}\\
    &=\dfrac{c A e^{\frac{-5}{6}ct\pm\frac{1}{6}\sqrt{6c}x}\pbrac{2A e^{\frac{-5}{6}ct\pm\frac{1}{6}\sqrt{6c}x}-1}}{3\sqbrac{1+
        A e^{\frac{-5}{6}ct\pm\frac{1}{6}\sqrt{6c}x}}^{3}}
  \end{aligned}
  \label{eqn:AnalyticReactionDiffusionOneDim1Derivatives}
\end{equation}
and the reaction term in \eqnref{eqn:AnalyticReactionDiffusionFishersEquation} is thus
given by
\begin{equation}
  \begin{aligned}
    c\fnof{u}{x,t}\pbrac{1-\fnof{u}{x,t}}&=\dfrac{cA e^{\fnof{w}{x,t}}\pbrac{2+A e^{\fnof{w}{x,t}}}}{\sqbrac{1+A e^{\fnof{w}{x,t}}}^{4}}\\
    &=\dfrac{cA e^{\frac{-5}{6}ct\pm\frac{1}{6}\sqrt{6c}x}\pbrac{2+A e^{\frac{-5}{6}ct\pm\frac{1}{6}\sqrt{6c}x}}}{\sqbrac{1+
        A e^{\frac{-5}{6}ct\pm\frac{1}{6}\sqrt{6c}x}}^{4}}
  \end{aligned}
  \label{eqn:AnalyticReactionDiffusionOneDim1Reaction}
\end{equation}

Substiting
\eqnrefs{eqn:AnalyticReactionDiffusionOneDim1Derivatives}{eqn:AnalyticReactionDiffusionOneDim1Reaction}
into \eqnref{eqn:AnalyticReactionDiffusionFishersEquation} gives
\begin{align}
  \delby{u}{t}-\deltwosqby{u}{x}-c.u\pbrac{1-u}&= \\
  \dfrac{-2\alpha A e^{\fnof{w}{x,t}}}{\sqbrac{1+A e^{\fnof{w}{x,t}}}^{3}}-
  \dfrac{2\beta^{2}A e^{\fnof{w}{x,t}}\pbrac{2A e^{\fnof{w}{x,t}}-1}}{\sqbrac{1+A e^{\fnof{w}{x,t}}}^{4}}
  -\dfrac{cA e^{\fnof{w}{x,t}}\pbrac{2+A e^{\fnof{w}{x,t}}}}{\sqbrac{1+A e^{\fnof{w}{x,t}}}^{4}}&=0\\
  \dfrac{A e^{\fnof{w}{x,t}}}{\sqbrac{1+A e^{\fnof{w}{x,t}}}^{4}}\pbrac{-2\alpha\pbrac{1+A e^{\fnof{w}{x,t}}}-
    2\beta^{2}\pbrac{2A e^{\fnof{w}{x,t}}-1}-c\pbrac{2+A e^{\fnof{w}{x,t}}}}&=0\\
  \dfrac{A e^{\fnof{w}{x,t}}}{6\sqbrac{1+A e^{\fnof{w}{x,t}}}^{4}}\pbrac{10 c\pbrac{1+A e^{\fnof{w}{x,t}}}-
    2c\pbrac{2A e^{\fnof{w}{x,t}}-1}-6c\pbrac{2+A e^{\fnof{w}{x,t}}}}&=0\\
  \dfrac{A e^{\fnof{w}{x,t}}}{6\sqbrac{1+A e^{\fnof{w}{x,t}}}^{4}}\pbrac{10 c+10 cA e^{\fnof{w}{x,t}}-
    4cA e^{\fnof{w}{x,t}}+2c-12c-6cA e^{\fnof{w}{x,t}}}&=0\\
  \dfrac{A e^{\fnof{w}{x,t}}}{6\sqbrac{1+A e^{\fnof{w}{x,t}}}^{4}}\pbrac{0}&=0
\end{align}
which thus proves that \eqnref{eqn:AnalyticReactionDiffusionEquationOneDim1} is an analytic solution to \eqnref{eqn:AnalyticReactionDiffusionFishersEquation}.

The analytic field component definitions in \OpenCMISS are shown in \tabref{tab:OpenCMISSAnalyticFieldReactionDiffusionOneDim1}.

\begin{table}[htb] \centering
  \begin{tabular}{|c|c|} \hline
    \emph{Analytic constant} & \emph{Analytic field component} \\ \hline \hline
    $A$ & $1$ \\ 
    $c$ & $2$ \\ \hline
  \end{tabular}
  \caption{\OpenCMISS analytic field components for the \oned reaction-diffusion equation
    analytic function 1.}
  \label{tab:OpenCMISSAnalyticFieldReactionDiffusionOneDim1}
\end{table}


\subsection{Worked example}
\label{subsec:ReactionDiffusionWorkedExample}

Consider Fisher's equation
\begin{equation}
  \delby{\fnof{u}{x,t}}{t}-D\deltwosqby{\fnof{u}{x,t}}{x}=c.\fnof{u}{x,t}\pbrac{1-\fnof{u}{x,t}}
  \label{eqn:FishersExampleEquation}
\end{equation}
on the \onedal domain of length $L$ and with $E$ elements and $N=E+1$ nodes as shown in \figref{fig:Fishers1DDomain}.

\epstexfigure{ClassicalField/svgs/FishersOneDDomain.eps_tex}{One
  dimensional domain for Fisher's equation.}{One
  dimensional domain for Fisher's equation. The domain
  of length $L$ has $E$ elements and $N=E+1$ nodes. The length of each
  element is $L/E$.}{fig:Fishers1DDomain}{1.0}

The weighted residual form of \eqnref{eqn:FishersExampleEquation} is given by
\begin{equation}
  \gint{\Omega}{}{\pbrac{\delby{\fnof{u}{x,t}}{t}-D\deltwosqby{\fnof{u}{x,t}}{x}-
      c.\fnof{u}{x,t}\pbrac{1-\fnof{u}{x,t}}}\fnof{w}{x}}{\Omega}=0
  \label{eqn:FishersExampleWeightedResidual}
\end{equation}

Applying the divergence theorem to \eqnref{eqn:FishersExampleWeightedResidual} gives
\begin{equation}
  \begin{split}
    \gint{\Omega}{}{\pbrac{\delby{\fnof{u}{x,t}}{t}+D\delby{\fnof{u}{x,t}}{x}\delby{\fnof{w}{x}}{x}-
        c.\fnof{u}{x,t}\pbrac{1-\fnof{u}{x,t}}}\fnof{w}{x}}{\Omega}\\
    -\gint{\Gamma}{}{\pbrac{\dotprod{D\delby{\fnof{u}{x,t}}{x}}{n}}\fnof{w}{x}}{\Gamma}=0
  \end{split}
  \label{eqn:FishersExampleDivergence}
\end{equation}
or
\begin{equation}
  \begin{split}
    \gsum{e=1}{E}{\gint{0}{1}{\pbrac{\delby{\fnof{u}{\xi,t}}{t}\fnof{w}{\xi}
          +D\delby{\fnof{u}{\xi,t}}{\xi}\delby{\xi}{x}\delby{\fnof{w}{\xi}}{\xi}\delby{\xi}{x}
          +c\fnof{u}{\xi,t}\pbrac{1-\fnof{u}{\xi,t}}\fnof{w}{\xi}}\abs{\fnof{J}{\xi}}}{\xi}} \\
    -\gsum{f=1}{F}{\gint{0}{1}{\pbrac{\dotprod{D\delby{\fnof{u}{\xi,t}}{\xi}\delby{\xi}{x}}{n}}\fnof{w}{\xi}
        \abs{\fnof{J}{\xi}}}{\xi}}=0
  \end{split}
  \label{eqn:FishersExampleFEMForm}
\end{equation}

Upon integration the resultant system of equations can be written as
\begin{equation}
  \matr{C}\dotover{\vectr{u}}+\matr{K}\vectr{u}+\fnof{\vectr{g}}{\vectr{u}}=\vectr{f}
\end{equation}

Considering element $i$ we can derive expressions for the elemental
stiffness matrices. Assuming linear basis functions the interpolation
expression for $\fnof{u}{\xi}$ can be written as
\begin{equation}
  \fnof{u}{\xi}=\xi u_{i}+\pbrac{1-\xi}u_{i+1}
\end{equation}
also we have
\begin{align}
  \delby{x}{\xi}&=\dfrac{L}{E} \\
  \delby{\xi}{x}&=\dfrac{E}{L} \\
  \abs{\fnof{J}{\xi}}&=\dfrac{L}{E}
\end{align}

The element damping matrix can be written as
\begin{align}
  C_{i,i}&=\gint{0}{1}{\delby{\pbrac{\xi u_{i}+\pbrac{1-\xi}u_{i+1}}}{t}\xi\dfrac{L}{E}}{\xi} \\
  &=\dfrac{L}{E}\gint{0}{1}{\pbrac{\xi\dotover{u}_{i}+\pbrac{1-\xi}\dotover{u}_{i+1}}\xi}{\xi} \\
  &=\dfrac{L}{E}\gint{0}{1}{\pbrac{\xi^{2}\dotover{u}_{i}+\pbrac{\xi-\xi^{2}}\dotover{u}_{i+1}}}{\xi} \\
  &=\dfrac{L}{E}\sqbrac{\dfrac{\xi^{3}}{3}\dotover{u}_{i}+\pbrac{\dfrac{\xi^{2}}{2}-\dfrac{\xi^{3}}{3}}\dotover{u}_{i+1}}_{0}^{1}\\
  &=\dfrac{L}{E}\pbrac{\dfrac{\dotover{u}_{i}}{3}+\dfrac{\dotover{u}_{i+1}}{2}-\dfrac{\dotover{u}_{i+1}}{3}}\\
  &=\dfrac{L\pbrac{2\dotover{u}_{i}+\dotover{u}_{i+1}}}{6E}  
\end{align}

Similarily
\begin{align}
  C_{i+1,i}&=\gint{0}{1}{\delby{\pbrac{\xi u_{i}+\pbrac{1-\xi}u_{i+1}}}{t}\pbrac{1-\xi}\dfrac{L}{E}}{\xi} \\
  &=\dfrac{L}{E}\gint{0}{1}{\pbrac{\xi\dotover{u}_{i}+\pbrac{1-\xi}\dotover{u}_{i+1}}\pbrac{1-\xi}}{\xi} \\
  &=\dfrac{L}{E}\gint{0}{1}{\pbrac{\pbrac{\xi-\xi^{2}}\dotover{u}_{i}+\pbrac{1-2\xi+\xi^{2}}\dotover{u}_{i+1}}}{\xi} \\
  &=\dfrac{L}{E}\sqbrac{\pbrac{\dfrac{\xi^{2}}{2}-\dfrac{\xi^{3}}{3}}\dotover{u}_{i}+\pbrac{\xi-\xi^{2}+
      \dfrac{\xi^{3}}{3}}\dotover{u}_{i+1}}_{0}^{1}\\
  &=\dfrac{L}{E}\pbrac{\dfrac{\dotover{u}_{i}}{2}-\dfrac{\dotover{u}_{i}}{3}+\dotover{u}_{i+1}-\dotover{u}_{i+1}+
    \dfrac{\dotover{u}_{i+1}}{3}}\\
  &=\dfrac{L\pbrac{\dotover{u}_{i}+2\dotover{u}_{i+1}}}{6E}  
\end{align}

The element damping matrix for element $i$ is thus
\begin{equation}
  \begin{bmatrix}
    C_{i,i} & C_{i,i+1} \\
    C_{i+1,i} & C_{i+1,i+1}
  \end{bmatrix}\begin{bmatrix}
    \dotover{u}_{i} \\
    \dotover{u}_{i+1}
  \end{bmatrix} = \begin{bmatrix}
    \frac{L}{3E} & \frac{L}{6E} \\
    \frac{L}{6E} & \frac{L}{3E}
   \end{bmatrix}\begin{bmatrix}
    \dotover{u}_{i} \\
    \dotover{u}_{i+1}
  \end{bmatrix}
\end{equation}


The element stiffness matrix can be written as
\begin{align}
  K_{i,i}&=\gint{0}{1}{D\delby{\pbrac{\xi u_{i}+\pbrac{1-\xi}u_{i+1}}}{\xi}\dfrac{E}{L}\delby{\xi}{\xi}\dfrac{E}{L}\dfrac{L}{E}}{\xi} \\
  &=\dfrac{DE\pbrac{u_{i}-u_{i+1}}}{L}\gint{0}{1}{}{\xi}\\
  &=\dfrac{DE\pbrac{u_{i}-u_{i+1}}}{L}\sqbrac{\xi}_{0}^{1}\\
  &=\dfrac{DE\pbrac{u_{i}-u_{i+1}}}{L}
\end{align}

Similarily
\begin{align}
  K_{i+1,i}&=\gint{0}{1}{D\delby{\pbrac{\xi u_{i}+\pbrac{1-\xi}u_{i+1}}}{\xi}\dfrac{E}{L}\delby{\pbrac{1-\xi}}{\xi}\dfrac{E}{L}\dfrac{L}{E}}{\xi} \\
  &=\dfrac{-DE\pbrac{u_{i}-u_{i+1}}}{L}\gint{0}{1}{}{\xi}\\
  &=\dfrac{-DE\pbrac{u_{i}-u_{i+1}}}{L}\sqbrac{\xi}_{0}^{1}\\
  &=\dfrac{-DE\pbrac{u_{i}-u_{i+1}}}{L}
\end{align}

The element stiffness matrix for element $i$ is thus
\begin{equation}
  \begin{bmatrix}
    K_{i,i} & K_{i,i+1} \\
    K_{i+1,i} & K_{i+1,i+1}
  \end{bmatrix}\begin{bmatrix}
    u_{i} \\
    u_{i+1}
  \end{bmatrix} = \begin{bmatrix}
    \frac{DE}{L} & \frac{-DE}{L} \\
    \frac{-DE}{L} & \frac{DE}{L}
   \end{bmatrix}\begin{bmatrix}
    u_{i} \\
    u_{i+1}
  \end{bmatrix}
\end{equation}

The nonlinear vector can be written as
\begin{align}
  g_{i}&=\gint{0}{1}{c.\pbrac{\xi u_{i}+\pbrac{1-\xi}u_{i+1}}\pbrac{1-\pbrac{\xi u_{i}+\pbrac{1-\xi}u_{i+1}}}\xi\dfrac{L}{E}}{\xi} \\
  &=\dfrac{cL}{E}\gint{0}{1}{\pbrac{\pbrac{\xi-\xi^{2}}u^{2}_{i}-2\pbrac{\xi-\xi^{2}}u_{i}u_{i+1}+\pbrac{\xi-\xi^{2}}u^{2}_{i+1}}\xi}{\xi} \\
  &=\dfrac{cL}{E}\gint{0}{1}{\pbrac{\pbrac{\xi^{2}-\xi^{3}}u^{2}_{i}-2\pbrac{\xi^{2}-\xi^{3}}u_{i}u_{i+1}+\pbrac{\xi^{2}-\xi^{3}}u^{2}_{i+1}}}{\xi} \\
  &=\dfrac{cL}{E}\pbrac{\sqbrac{\dfrac{\xi^{3}}{3}-\dfrac{\xi^{4}}{4}}^{1}_{0}u^{2}_{i}-2\sqbrac{\dfrac{\xi^{3}}{3}-\dfrac{\xi^{4}}{4}}^{1}_{0}u_{i}u_{i+1}+\sqbrac{\dfrac{\xi^{3}}{3}-\dfrac{\xi^{4}}{4}}^{1}_{0}u^{2}_{i+1}}\\
  &=\dfrac{cL\pbrac{u_{i}^{2}-2u_{i}u_{i+1}+u_{i+1}^{2}}}{12E}
\end{align}

Similarily
\begin{align}
  g_{i+1}&=\gint{0}{1}{c.\pbrac{\xi u_{i}+\pbrac{1-\xi}u_{i+1}}\pbrac{1-\pbrac{\xi u_{i}+\pbrac{1-\xi}u_{i+1}}}\pbrac{1-\xi}\dfrac{L}{E}}{\xi} \\
  &=\dfrac{cL}{E}\gint{0}{1}{\pbrac{\pbrac{\xi-\xi^{2}}u^{2}_{i}-2\pbrac{\xi-\xi^{2}}u_{i}u_{i+1}+\pbrac{\xi-\xi^{2}}u^{2}_{i+1}}\pbrac{1-\xi}}{\xi} \\
  &=\dfrac{cL}{E}\gint{0}{1}{\pbrac{\pbrac{\xi-2\xi^{2}+\xi^{3}}u^{2}_{i}-2\pbrac{\xi-2\xi^{2}+\xi^{3}}u_{i}u_{i+1}+\pbrac{\xi-2\xi^{2}+\xi^{3}}u^{2}_{i+1}}}{\xi} \\
  &=\dfrac{cL}{E}\pbrac{\sqbrac{\dfrac{\xi^{2}}{2}-2\dfrac{\xi^{3}}{3}+\dfrac{\xi^{4}}{4}}^{1}_{0}u^{2}_{i}-2\sqbrac{\dfrac{\xi^{2}}{2}-2\dfrac{\xi^{3}}{3}+\dfrac{\xi^{4}}{4}}^{1}_{0}u_{i}u_{i+1}+\sqbrac{\dfrac{\xi^{2}}{2}-2\dfrac{\xi^{3}}{3}+\dfrac{\xi^{4}}{4}}^{1}_{0}u^{2}_{i+1}}\\
  &=\dfrac{-cL\pbrac{u_{i}^{2}-2u_{i}u_{i+1}+u_{i+1}^{2}}}{12E}
\end{align}


The element residual vector for element $i$ is thus
\begin{equation}
  \begin{bmatrix}
    g_{i}  \\
    g_{i+1} 
  \end{bmatrix} = \begin{bmatrix}
    \frac{cL\pbrac{u_{i}^{2}-2u_{i}u_{i+1}+u_{i+1}^{2}}}{12E} \\
    \frac{-cL\pbrac{u_{i}^{2}-2u_{i}u_{i+1}+u_{i+1}^{2}}}{12E}
   \end{bmatrix}
\end{equation}

The element Jacobian matrix for element $i$ is thus
\begin{equation}
  \begin{bmatrix}
    J_{i,i} & J_{i,i+1} \\
    J_{i+1,i} & J_{i+1,i+1} 
  \end{bmatrix} = \begin{bmatrix}
    \delby{g_{i}}{u_{i}} & \delby{g_{i}}{u_{i+1}} \\
    \delby{g_{i+1}}{u_{i}} & \delby{g_{i+1}}{u_{i+1}}
   \end{bmatrix} = \begin{bmatrix}
    \frac{cL\pbrac{u_{i}-u_{i+1}}}{6E} & \frac{-cL\pbrac{u_{i}-u_{i+1}}}{6E} \\
    \frac{-cL\pbrac{u_{i}-u_{i+1}}}{6E} & \frac{cL\pbrac{u_{i}-u_{i+1}}}{6E}
   \end{bmatrix}
\end{equation}

The matrix system for the domain shown in \figref{fig:Fishers1DDomain} with no flux conditions is thus
\begin{multline}
  \begin{bmatrix}
    \frac{L}{3E} & \frac{L}{6E} & 0 & 0 & 0 & \cdots & 0 & 0 \\
    \frac{L}{6E} & \frac{2L}{3E} &  \frac{L}{6E} & 0 & 0 & \cdots & 0 & 0 \\
    0 & \frac{L}{6E} & \frac{2L}{3E} &  \frac{L}{6E} & 0 & \cdots & 0 & 0 \\
    0 & 0 & \frac{L}{6E} & \frac{2L}{3E} &  \frac{L}{6E} & \cdots & 0 & 0 \\
    0 & 0 & 0 & \frac{L}{6E} & \frac{2L}{3E} & \cdots & 0 & 0 \\
    \vdots & \vdots & \vdots & \vdots & \vdots & \ddots & \vdots & \vdots \\
    0 & 0 & 0 & 0 & 0 & \cdots & \frac{2L}{3E} & \frac{L}{6E} \\
    0 & 0 & 0 & 0 & 0 & \cdots & \frac{L}{6E} & \frac{L}{3E} \\
  \end{bmatrix} \begin{bmatrix}
    \dotover{u}_{1} \\
    \dotover{u}_{2} \\
    \dotover{u}_{3} \\
    \dotover{u}_{4} \\
    \dotover{u}_{5} \\
    \vdots \\
    \dotover{u}_{N-1} \\
    \dotover{u}_{N} \\
  \end{bmatrix} \\
  + \begin{bmatrix}
    \frac{DE}{L} & \frac{-DE}{L} & 0 & 0 & 0 & \cdots & 0 & 0 \\
    \frac{-DE}{L} & \frac{2DE}{L} &  \frac{-DE}{L} & 0 & 0 & \cdots & 0 & 0 \\
    0 & \frac{-DE}{L} & \frac{2DE}{L} &  \frac{-DE}{L} & 0 & \cdots & 0 & 0 \\
    0 & 0 & \frac{-DE}{L} & \frac{2DE}{L} &  \frac{-DE}{L} & \cdots & 0 & 0 \\
    0 & 0 & 0 & \frac{-DE}{L} & \frac{2DE}{L} & \cdots & 0 & 0 \\
    \vdots & \vdots & \vdots & \vdots & \vdots & \ddots & \vdots & \vdots \\
    0 & 0 & 0 & 0 & 0 & \cdots & \frac{2DE}{L} & \frac{-DE}{L} \\
    0 & 0 & 0 & 0 & 0 & \cdots & \frac{-DE}{L} & \frac{DE}{L} \\
  \end{bmatrix} \begin{bmatrix}
    u_{1} \\
    u_{2} \\
    u_{3} \\
    u_{4} \\
    u_{5} \\
    \vdots \\
    u_{N-1} \\
    u_{N}
  \end{bmatrix} \\
  + \begin{bmatrix}
    \frac{cL\pbrac{u_{1}^{2}-2u_{1}u_{2}+u_{2}^{2}}}{12E} \\
    \frac{cL\pbrac{-u_{1}^{2}+2u_{1}u_{2}-2u_{2}u_{3}+u_{3}^{2}}}{12E} \\
    \frac{cL\pbrac{-u_{2}^{2}+2u_{2}u_{3}-2u_{3}u_{4}+u_{4}^{2}}}{12E} \\
    \frac{cL\pbrac{-u_{3}^{2}+2u_{3}u_{4}-2u_{4}u_{5}+u_{5}^{2}}}{12E} \\
    \frac{cL\pbrac{-u_{4}^{2}+2u_{4}u_{5}-2u_{5}u_{6}+u_{6}^{2}}}{12E} \\
    \vdots \\
    \frac{cL\pbrac{-u_{N-2}^{2}+2u_{N-2}u_{N-1}-2u_{N-1}u_{N}+u_{N}^{2}}}{12E} \\
    \frac{cL\pbrac{-u_{N-1}^{2}+2u_{N-1}u_{N}-u_{N}^{2}}}{12E}    
  \end{bmatrix} = \begin{bmatrix}
    0 \\
    0 \\
    0 \\
    0 \\
    0 \\
    \vdots \\
    0 \\
    0 \\
  \end{bmatrix}
\end{multline}

And the Jacobian matrix is
\begin{equation}
  \begin{bmatrix}
    \frac{cL\pbrac{u_{1}-u_{2}}}{6E} & \frac{-cL\pbrac{u_{1}-u_{2}}}{6E} & 0 & 0 & 0 & \cdots & 0 & 0 \\
    \frac{-cL\pbrac{u_{1}-u_{2}}}{6E} & \frac{cL\pbrac{u_{1}-u_{3}}}{6E} &  \frac{-cL\pbrac{u_{2}-u_{3}}}{6E} & 0 & 0 & \cdots & 0 & 0 \\
    0 & \frac{-cL\pbrac{u_{2}-u_{3}}}{6E} & \frac{cL\pbrac{u_{2}-u_{4}}}{6E} &  \frac{-cL\pbrac{u_{3}-u_{4}}}{6E} & 0 & \cdots & 0 & 0 \\
    0 & 0 & \frac{-cL\pbrac{u_{3}-u_{4}}}{6E} & \frac{cL\pbrac{u_{3}-u_{5}}}{6E} &  \frac{-cL\pbrac{u_{4}-u_{5}}}{6E} & \cdots & 0 & 0 \\
    0 & 0 & 0 & \frac{-cL\pbrac{u_{4}-u_{5}}}{6E} & \frac{cL\pbrac{u_{4}-u_{6E}}}{6E} & \cdots & 0 & 0 \\
    \vdots & \vdots & \vdots & \vdots & \vdots & \ddots & \vdots & \vdots \\
    0 & 0 & 0 & 0 & 0 & \cdots & \frac{cL\pbrac{u_{N-2}-u_{N}}}{6E} & \frac{-cL\pbrac{u_{N-1}-u_{N}}}{6E} \\
    0 & 0 & 0 & 0 & 0 & \cdots & \frac{-cL\pbrac{u_{N-1}-u_{N}}}{6E} & \frac{cL\pbrac{u_{N-1}-u_{N}}}{6E}
  \end{bmatrix}
\end{equation}
