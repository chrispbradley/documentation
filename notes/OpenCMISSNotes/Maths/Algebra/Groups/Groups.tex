\subsection{Groups}
\label{subsec:MathsAlgebraGroups}

In mathematics a group is a set, $G$, and a binary operation that acts
on members of the set to produce a member of the set. If $a$ and $b$
are members of the set the the group operation is denoted as $a.b$
which is an element of $G$. A group has three requirements or
\emph{group axioms}, namely:
\begin{itemize}
\item \emph{Associativity}: For all $a,b,c\in G$ then $(a.b).c = a.(b.c)$.
\item \emph{Identity element}: There exists an element $i\in G$ such that for each $a\in G$ then $i.a=a$ and $a.i=a$.
\item \emph{Inverse element}: For each element $a\in G$ there exists another element $b\in G$ such that $a.b=i$ and $b.a=i$. $b$ is known as the \emph{inverse element} of $a$. .
\end{itemize}

In addition to the group axioms listed above we can further categorise
a group by considering wether or not the group is commutative under
the group operation \ie
\begin{itemize}
  \item \emph{Commutivity}: For all $a,b\in G$ then $a.b=b.a$
\end{itemize}

In the group operation is commutative then the group is known as a
\emph{commutative group} or \emph{Abelian group}\footnote{named after
\link{https://en.wikipedia.org/wiki/Niels_Henrik_Abel}{Niels Henrik
  Abel} (1802-1829), a Norwegian mathematician.}. If the group
operation is not commutative then the group is known as a
\emph{non-commutative group} or \emph{non-Abelian group}.

The set of numbers together with the addition operation form an
example of a group. Groups play important roles in symmetry and
geometric transformations.

