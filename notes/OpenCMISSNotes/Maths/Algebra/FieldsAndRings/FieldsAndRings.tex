\subsection{Fields and Rings}
\label{subsec:MathsAlgebraFieldsRings}

In mathematics a field is a set, $F$, together with two binary operations
(\cf groups that have one operation) that act on members of the set to
produce another member of the set. One of the binary operations is an
\emph{addition} operation and the second operation is a
\emph{multiplication} operation. The result of the addition operation
on two elements of the set $a$ and $b$ is called the \emph{sum} and is
denoted as $a+b$. The result of the multiplication operation on two
elements of the set $a$ and $b$ is called the \emph{product} and is
denoted as $a.b$. A field must satisity the \emph{field axioms}, namely if $a,b,c\in F$ then:
\begin{itemize}
\item \emph{Associativity}: Both the addition and multiplication operations are associative \ie $a+(b+c)=(a+b)+c$ and $a.(b.c)=(a.b).c$.
\item \emph{Commutativity}: Both the addition and multiplication operations are commutative \ie $a+b=b+a$ and $a.b=b.a$.
\item \emph{Distributivity}: The multiplication operator is distributive over addition \ie $a.(b+c)=(a.b)+(b.c)$.
\item \emph{Additive identity element}: There exists an additive identity element $0\in F$ such that $a+0=a$ for all $a\in F$.
\item \emph{Multiplicative identity element}: There exists a multiplicative identity element $1\in F$ such that $a.1=a$ for all $a\in F$.
\item \emph{Additive inverse element}: For every element $a\in F$ there exists an element, $-a\in F$, called the additive inverse of $a$ such that $a+(-a)=0$.
\item \emph{Multiplicative inverse element}: For every element $a\in F$ there exists an element, $\inverse{a}$, called the multiplicative inverse of $a$ such that $a.\inverse{a}=1$.
\end{itemize}

Fields, thus, form a Abelian group under addition with $0$ as the
identity element, and the non-zero elements of the field also form
another Abelian group under multiplication with $1$ as the identity
element.

Examples of fields include the well known real numbers, $\realnums$,
and complex numbers, $\complexnums$, with the usual addition and
multiplication operations.

If the number of elements in a field's set is finite then the field is
known as a \emph{finite field}. An example of a finite field is the
binary numbers which have two elements in the set, $0$ and $1$. The
binary numbers form a field with $0$ the additive identity element,
$1$ the multiplicative identity element, logical exclusive or,
$\oplus$, is the addition operation, and logical and, $.$ is the
multiplicative operation.

A field is a specialisation of a more general mathemetical structure called a \emph{ring}. 


