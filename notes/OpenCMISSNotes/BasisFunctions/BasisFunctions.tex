\clearemptydoublepage
\chapter{Basis Function and Interpolation}
\label{cha:BasisFunctions}

\section{Representing a One-Dimensional Field}
\label{sec:reponedfield}

Consider the problem of finding a mathematical expression $\fnof{u}{x}$ to
represent a one-dimensional field \eg measurements of temperature $u$
against distance $x$ along a bar, as shown in \figref{fig:1Dfield}.

\epstexfigure{BasisFunctions/svgs/onedimfield.eps_tex}{One dimensional field
  measurements.}  {Temperature distribution $\fnof{u}{x}$ along a bar.
  The points are the measured temperatures.}{fig:1Dfield}{0.3}
\epstexfigure{BasisFunctions/svgs/onedimfieldline.eps_tex}{One dimensional
  field measurements with fitted line.}{A least-squares polynomial fit to the
  data, showing the unacceptable oscillation between data
  points}{fig:1Dfieldline}{0.3}

One approach would be to use a polynomial expression $\fnof{u}{x} = a + bx +
cx^{2} + dx^{3} + \ldots$ and to estimate the values of the parameters
$a$, $b$, $c$ and $d$ from a least-squares fit to the data.  As the degree of the
polynomial is increased the data points are fitted with increasing accuracy
and polynomials provide a very convenient form of expression because they can
be differentiated and integrated readily. For low degree polynomials this is a
satisfactory approach, but if the polynomial order is increased further to
improve the accuracy of fit a problem arises: the polynomial can be made to
fit the data accurately, but it oscillates unacceptably between the data
points, as shown in \figref{fig:1Dfieldline}.  

\section{Basis Functions and Interpolation}
\label{sec:basisfunctions}

Both the finite element method (FEM) and the boundary element method (BEM) use
interpolation in finding a field solution \ie the methods find the solution at
a number of points in the domain of interest and then approximate the solution
between these points using interpolation. The points at which the solution is
found are known as \emph{nodes}. \emph{Basis functions} are used to
interpolate the field between nodes within a subregion of the domain known as
an \emph{element}. Interpolation is achieved by mapping the field coordinate
onto a \emph{local parametric}, or $\elementcoordinatesymbol$, coordinate (which varies from $0$ to
$1$) within each element. The global nodes which make up each element are also
mapped onto local element nodes and the basis functions are chosen (in terms
of polynomials of the local parametric coordinate) such that the interpolated
field is equal to the known nodal values at each node and is thus continuous
between elements. A schematic of this
scheme is shown in \figref{fig:nodesandelements}.

\epstexfigure{BasisFunctions/svgs/nodesandelements.eps_tex}
{A schematic of the relationship between local and global nodes, elements and the parametric elemental $\elementcoordinatesymbol$
  coordinate.}{Relationship between local and global nodes and elements.}{fig:nodesandelements}{0.3}

\subsection{Lagrangian Basis Functions}
\label{sec:lagrangebasisfunctions}

One important family of basis functions are the Lagrange\footnote{named after
\link{https://en.wikipedia.org/wiki/Joseph-Louis_Lagrange}{Joseph-Louis Lagrange}
(1736-1813), an Italian-born, French mathematician and physicist.} basis functions. This
family has one basis function for each of the local element nodes and are
defined such that, at a particular node, only one basis function is non-zero
and has the value of one. In this sense a basis function can be thought of as
being associated with a local node and serves to weight the interpolated
solution in terms of the field value at that node. Lagrange basis functions
hence provide only $C^{0}$ continuity of the field variable across element
boundaries.

\subsubsection{Linear Lagrange basis functions}

The simplest basis functions of the Lagrange family are the \onedal linear
Lagrange basis functions. These basis functions involve two local nodes and
are defined as
\begin{equation}
  \begin{split}
    \lbfn{1}{\elementcoordinatesymbol}&=1-\elementcoordinatesymbol \\
    \lbfn{2}{\elementcoordinatesymbol}&=\elementcoordinatesymbol
  \end{split}
  \label{eqn:linearlbfuns}
\end{equation}

The two \onedal linear Lagrange basis functions are gshown in \figref{fig:linlagrangebfuns}.

\pstexfigure{BasisFunctions/plots/linlagrangebfuns.pstex}{Linear Lagrange basis functions.}
{Linear Lagrange basis functions.}{fig:linlagrangebfuns}

The interpolation of a field variable, $u$, using these basis functions is
given by
\begin{equation}
  \begin{split}
    \fnof{u}{\elementcoordinatesymbol}&=\lbfn{1}{\elementcoordinatesymbol}\nodept{u}{1}+\lbfn{2}{\elementcoordinatesymbol}\nodept{u}{2} \\
    &=\pbrac{1-\elementcoordinatesymbol}\nodept{u}{1}+\elementcoordinatesymbol\nodept{u}{2}
  \end{split}
\end{equation}
where $\nodept{u}{1}$ and $\nodept{u}{2}$ are the values of the field variable at
the first and second local nodes respectively. These basis functions hence
provide a linear variation between the local nodal values with the local
element coordinate, $\elementcoordinatesymbol$.

\subsubsection{Quadratic Lagrange basis functions}

Lagrange basis functions can also be used to provide higher order variations,
for example the one-dimensional quadratic Lagrange basis functions involve
three local nodes and can provide a quadratic variation of field parameter
with $\elementcoordinatesymbol$. They are defined as
\begin{equation}
  \begin{split}
    \lbfn{1}{\elementcoordinatesymbol}&=2\pbrac{\elementcoordinatesymbol-\frac12}\pbrac{\elementcoordinatesymbol-1} \\
    &=2\elementcoordinatesymbol^{2}-3\elementcoordinatesymbol+1\\
    \lbfn{2}{\elementcoordinatesymbol}&=4\elementcoordinatesymbol\pbrac{1-\elementcoordinatesymbol} \\
    &=-4\elementcoordinatesymbol^{2}+4\elementcoordinatesymbol\\
    \lbfn{3}{\elementcoordinatesymbol}&=2\elementcoordinatesymbol\pbrac{\elementcoordinatesymbol-\frac12}\\
    &=2\elementcoordinatesymbol^{2}-\elementcoordinatesymbol
  \end{split}
  \label{eqn:quadraticlbfuns}
\end{equation}

The three \onedal quadratic Lagrange basis functions are shown in \figref{fig:quadlagrangebfuns}.

\pstexfigure{BasisFunctions/plots/quadlagrangebfuns.pstex}{Quadratic Lagrange basis functions.}
{Quadratic Lagrange basis functions.}{fig:quadlagrangebfuns}

The interpolation formula is
\begin{equation}
  \begin{split}
    \fnof{u}{\elementcoordinatesymbol}&=\lbfn{1}{\elementcoordinatesymbol}\nodept{u}{1}+\lbfn{2}{\elementcoordinatesymbol}\nodept{u}{2}+
    \lbfn{3}{\elementcoordinatesymbol}\nodept{u}{3}\\
    &=2\pbrac{\elementcoordinatesymbol-\frac12}\pbrac{\elementcoordinatesymbol-1}\nodept{u}{1}+
    4\elementcoordinatesymbol\pbrac{1-\elementcoordinatesymbol}\nodept{u}{2}+2\elementcoordinatesymbol\pbrac{\elementcoordinatesymbol-\frac12}\nodept{u}{3}
  \end{split}
\end{equation}

\subsubsection{Cubic Lagrange basis functions}

One-dimensional cubic Lagrange basis functions involve
four local nodes and can provide a cubic variation of field parameter
with $\elementcoordinatesymbol$. They are defined as
\begin{equation}
  \begin{split}
    \lbfn{1}{\elementcoordinatesymbol}&=\frac12\pbrac{3\elementcoordinatesymbol-1}\pbrac{3\elementcoordinatesymbol-2}\pbrac{1-\elementcoordinatesymbol} \\
    &=\frac12\pbrac{-9\elementcoordinatesymbol^{3}+18\elementcoordinatesymbol^{2}-11\elementcoordinatesymbol+2}\\
    \lbfn{2}{\elementcoordinatesymbol}&=\frac92\elementcoordinatesymbol\pbrac{3\elementcoordinatesymbol-2}\pbrac{\elementcoordinatesymbol-1} \\
    &=\frac92\pbrac{3\elementcoordinatesymbol^{3}-5\elementcoordinatesymbol^{2}+2\elementcoordinatesymbol}\\
    \lbfn{3}{\elementcoordinatesymbol}&=\frac92\elementcoordinatesymbol\pbrac{3\elementcoordinatesymbol-1}\pbrac{1-\elementcoordinatesymbol} \\
    &=\frac92\pbrac{-3\elementcoordinatesymbol^{3}+4\elementcoordinatesymbol^{2}-\elementcoordinatesymbol}\\
    \lbfn{4}{\elementcoordinatesymbol}&=\frac12\elementcoordinatesymbol\pbrac{3\elementcoordinatesymbol-1}\pbrac{3\elementcoordinatesymbol-2} \\
    &=\frac12\pbrac{9\elementcoordinatesymbol^{3}-9\elementcoordinatesymbol^{2}+2\elementcoordinatesymbol}
  \end{split}
  \label{eqn:cubiclbfuns}
\end{equation}

The four \onedal cubic Lagrange basis functions are shown in \figref{fig:cublagrangebfuns}.

\pstexfigure{BasisFunctions/plots/cublagrangebfuns.pstex}{Cubic Lagrange basis functions.}
{Cubic Lagrange basis functions.}{fig:cublagrangebfuns}

The interpolation formula is
\begin{equation}
  \begin{split}
    \fnof{u}{\elementcoordinatesymbol}&=\lbfn{1}{\elementcoordinatesymbol}\nodept{u}{1}+\lbfn{2}{\elementcoordinatesymbol}\nodept{u}{2}+
    \lbfn{3}{\elementcoordinatesymbol}\nodept{u}{3}+\lbfn{4}{\elementcoordinatesymbol}\nodept{u}{4}\\
    &=\frac12\pbrac{3\elementcoordinatesymbol-1}\pbrac{3\elementcoordinatesymbol-2}\pbrac{1-\elementcoordinatesymbol}\nodept{u}{1}+
    \frac92\elementcoordinatesymbol\pbrac{3\elementcoordinatesymbol-2}\pbrac{\elementcoordinatesymbol-1}\nodept{u}{2} \\
    &\quad+\frac92\elementcoordinatesymbol\pbrac{3\elementcoordinatesymbol-1}\pbrac{1-\elementcoordinatesymbol}\nodept{u}{3}+
    \frac12\elementcoordinatesymbol\pbrac{3\elementcoordinatesymbol-1}\pbrac{3\elementcoordinatesymbol-2}\nodept{u}{4}
  \end{split}
\end{equation}

\subsubsection{General Lagrange basis functions}

In general the interpolation formula for the Lagrange family of basis
functions is, using \index{Einstein summation notation}\emph{Einstein
  summation notation}, given by
\begin{equation}
  \fnof{u}{\elementcoordinatesymbol}=\lbfn{\alpha}{\elementcoordinatesymbol}\nodept{u}{\alpha}\quad \alpha=1,\ldots,n_{e}
  \label{eqn:lagrangeinterpolation}
\end{equation}
where $n_{e}$ is the number of local nodes in the element. Einstein summation
notation uses a repeated index in a product expression to imply summation. For
example \eqnref{eqn:lagrangeinterpolation} is equivalent to
\begin{equation}
  \fnof{u}{\elementcoordinatesymbol}=\gsum{\alpha=1}{n_{e}}{\lbfn{\alpha}{\elementcoordinatesymbol}\nodept{u}{\alpha}}
\end{equation}

\subsubsection{Bilinear Lagrange basis functions}

Multi-dimensional Lagrange basis functions can be constructed from the tensor,
or outer, products of the one-dimensional Lagrange basis functions. For
example the two-dimensional bilinear Lagrange basis functions have four local
nodes with the basis functions given by
\begin{equation}
  \begin{split}
    \lbfn{1}{\elemcoordone,\elemcoordtwo}&=\lbfn{1}{\elemcoordone}\lbfn{1}{\elemcoordtwo}=
    \pbrac{1-\elemcoordone}\pbrac{1-\elemcoordtwo}\\
    \lbfn{2}{\elemcoordone,\elemcoordtwo}&=\lbfn{2}{\elemcoordone}\lbfn{1}{\elemcoordtwo}=
    \elemcoordone\pbrac{1-\elemcoordtwo}\\
    \lbfn{3}{\elemcoordone,\elemcoordtwo}&=\lbfn{1}{\elemcoordone}\lbfn{2}{\elemcoordtwo}=
    \pbrac{1-\elemcoordone}\elemcoordtwo \\
    \lbfn{4}{\elemcoordone,\elemcoordtwo}&=\lbfn{2}{\elemcoordone}\lbfn{2}{\elemcoordtwo}=
    \elemcoordone\elemcoordtwo
  \end{split}
\end{equation}

The four \twodal bilinear Lagrange basis functions are shown in \figref{fig:bilinlagrangebfuns}.

%\pstexfigure{BasisFunctions/plots/bilinlagrangebfuns.pstex}{Bilinear Lagrange basis functions.}
%            {Bilinear Lagrange basis functions.}{fig:bilinlagrangebfuns}

\begin{figure}[hbtp]
   \centering
   \gnuplotsubfigure{BasisFunctions/gnuplots/bilinlagrangebfuns1.gnuplot}{Bilinear Lagrange basis function $1$.}
     {Bilinear Lagrange basis function $1$, $\lbfn{1}{\elemcoordone,\elemcoordtwo}$.}{subfig:bilinlagrangebfuns1}{0.45\linewidth}{10}{3}{}
   \hfil
   \gnuplotsubfigure{BasisFunctions/gnuplots/bilinlagrangebfuns2.gnuplot}{Bilinear Lagrange basis function $2$.}
     {Bilinear Lagrange basis function $2$, $\lbfn{2}{\elemcoordone,\elemcoordtwo}$.}{subfig:bilinlagrangebfuns2}{0.45\linewidth}{10}{3}{}
   \gnuplotsubfigure{BasisFunctions/gnuplots/bilinlagrangebfuns3.gnuplot}{Bilinear Lagrange basis function $3$.}
     {Bilinear Lagrange basis function $3$, $\lbfn{3}{\elemcoordone,\elemcoordtwo}$.}{subfig:bilinlagrangebfuns3}{0.45\linewidth}{10}{3}{}
   \hfil  
   \gnuplotsubfigure{BasisFunctions/gnuplots/bilinlagrangebfuns4.gnuplot}{Bilinear Lagrange basis function $4$.}
     {Bilinear Lagrange basis function $4$, $\lbfn{4}{\elemcoordone,\elemcoordtwo}$.}{subfig:bilinlagrangebfuns4}{0.45\linewidth}{10}{3}{}
   \caption[Bilinear Lagrange basis functions.]{Bilinear Lagrange basis functions.}
   \label{fig:bilinlagrangebfuns}
\end{figure}
            
The multi-dimensional interpolation formula is still a sum of the products of
the nodal basis function and the field value at the node. For example the
interpolated geometric position vector within an element is given by
\begin{equation}
  \begin{split}
    \fnof{\coordinatevector}{\elemcoordone,\elemcoordtwo}&=\lbfn{\alpha}{\elemcoordone,\elemcoordtwo}
    \nodept{\coordinatevector}{\alpha}\\
    &=\lbfn{1}{\elemcoordone,\elemcoordtwo}\nodept{\coordinatevector}{1}+\lbfn{2}{\elemcoordone,\elemcoordtwo}
    \nodept{\coordinatevector}{2}+\lbfn{3}{\elemcoordone,\elemcoordtwo}\nodept{\coordinatevector}{3}+
    \lbfn{4}{\elemcoordone,\elemcoordtwo}\nodept{\coordinatevector}{4}
  \end{split}
\end{equation}
where, for the vector field, each component is interpolated separately using
the given basis functions.

\subsubsection{Biquadratic Lagrange basis functions}

The two-dimensional biquadratic Lagrange basis functions have nine local nodes
with the basis functions given by
\begin{equation}
  \begin{split}
    \lbfn{1}{\elemcoordone,\elemcoordtwo}&=\lbfn{1}{\elemcoordone}\lbfn{1}{\elemcoordtwo}=
    4\pbrac{\elemcoordone-\frac12}\pbrac{\elemcoordone-1}\pbrac{\elemcoordtwo-\frac12}\pbrac{\elemcoordtwo-1} \\
    \lbfn{2}{\elemcoordone,\elemcoordtwo}&=\lbfn{2}{\elemcoordone}\lbfn{1}{\elemcoordtwo}=
    8\elemcoordone\pbrac{1-\elemcoordone}\pbrac{\elemcoordtwo-\frac12}\pbrac{\elemcoordtwo-1} \\
    \lbfn{3}{\elemcoordone,\elemcoordtwo}&=\lbfn{3}{\elemcoordone}\lbfn{1}{\elemcoordtwo}=
    4\elemcoordone\pbrac{\elemcoordone-\frac12}\pbrac{\elemcoordtwo-\frac12}\pbrac{\elemcoordtwo-1} \\
    \lbfn{4}{\elemcoordone,\elemcoordtwo}&=\lbfn{1}{\elemcoordone}\lbfn{2}{\elemcoordtwo}=
    8\pbrac{\elemcoordone-\frac12}\pbrac{\elemcoordone-1}\elemcoordtwo\pbrac{1-\elemcoordtwo} \\
    \lbfn{5}{\elemcoordone,\elemcoordtwo}&=\lbfn{2}{\elemcoordone}\lbfn{2}{\elemcoordtwo}=
    16\elemcoordone\pbrac{1-\elemcoordone}\elemcoordtwo\pbrac{1-\elemcoordtwo} \\
    \lbfn{6}{\elemcoordone,\elemcoordtwo}&=\lbfn{3}{\elemcoordone}\lbfn{2}{\elemcoordtwo}=
    8\elemcoordone\pbrac{\elemcoordone-\frac12}\elemcoordtwo\pbrac{1-\elemcoordtwo} \\
    \lbfn{7}{\elemcoordone,\elemcoordtwo}&=\lbfn{1}{\elemcoordone}\lbfn{3}{\elemcoordtwo}=
    4\pbrac{\elemcoordone-\frac12}\pbrac{\elemcoordone-1}\elemcoordtwo\pbrac{\elemcoordtwo-\frac12} \\
    \lbfn{8}{\elemcoordone,\elemcoordtwo}&=\lbfn{2}{\elemcoordone}\lbfn{3}{\elemcoordtwo}=
    8\elemcoordone\pbrac{1-\elemcoordone}\elemcoordtwo\pbrac{\elemcoordtwo-\frac12} \\
    \lbfn{9}{\elemcoordone,\elemcoordtwo}&=\lbfn{3}{\elemcoordone}\lbfn{3}{\elemcoordtwo}=
    4\elemcoordone\pbrac{\elemcoordone-\frac12}\elemcoordtwo\pbrac{\elemcoordtwo-\frac12}
  \end{split}
\end{equation}

The nine \twodal biquadratic Lagrange basis functions are shown in \figref{fig:biquadlagrangebfuns}.

%\pstexfigure{BasisFunctions/plots/biquadlagrangebfuns.pstex}{Biquadratic Lagrange basis functions.}
%            {Biquadratic Lagrange basis functions.}{fig:biquadlagrangebfuns}

\begin{figure}[hbtp]
   \centering
   \gnuplotsubfigure{BasisFunctions/gnuplots/biquadlagrangebfuns1.gnuplot}{Biquadratic Lagrange basis function $1$.}
     {Biquadratic Lagrange basis function $1$, $\lbfn{1}{\elemcoordone,\elemcoordtwo}$.}{subfig:biquadlagrangebfuns1}{0.30\linewidth}{9}{2.5}{}
   \hfil
   \gnuplotsubfigure{BasisFunctions/gnuplots/biquadlagrangebfuns2.gnuplot}{Biquadratic Lagrange basis function $2$.}
     {Biquadratic Lagrange basis function $2$, $\lbfn{2}{\elemcoordone,\elemcoordtwo}$.}{subfig:biquadlagrangebfuns2}{0.30\linewidth}{9}{2.5}{}
   \hfil
   \gnuplotsubfigure{BasisFunctions/gnuplots/biquadlagrangebfuns3.gnuplot}{Biquadratic Lagrange basis function $3$.}
     {Biquadratic Lagrange basis function $3$, $\lbfn{3}{\elemcoordone,\elemcoordtwo}$.}{subfig:biquadlagrangebfuns3}{0.30\linewidth}{9}{2.5}{}
   \gnuplotsubfigure{BasisFunctions/gnuplots/biquadlagrangebfuns4.gnuplot}{Biquadratic Lagrange basis function $4$.}
     {Biquadratic Lagrange basis function $4$, $\lbfn{4}{\elemcoordone,\elemcoordtwo}$.}{subfig:biquadlagrangebfuns4}{0.30\linewidth}{9}{2.5}{}
   \hfil
   \gnuplotsubfigure{BasisFunctions/gnuplots/biquadlagrangebfuns5.gnuplot}{Biquadratic Lagrange basis function $5$.}
     {Biquadratic Lagrange basis function $5$, $\lbfn{5}{\elemcoordone,\elemcoordtwo}$.}{subfig:biquadlagrangebfuns5}{0.30\linewidth}{9}{2.5}{}
   \hfil
   \gnuplotsubfigure{BasisFunctions/gnuplots/biquadlagrangebfuns6.gnuplot}{Biquadratic Lagrange basis function $6$.}
     {Biquadratic Lagrange basis function $6$, $\lbfn{6}{\elemcoordone,\elemcoordtwo}$.}{subfig:biquadlagrangebfuns6}{0.30\linewidth}{9}{2.5}{}
   \gnuplotsubfigure{BasisFunctions/gnuplots/biquadlagrangebfuns7.gnuplot}{Biquadratic Lagrange basis function $7$.}
     {Biquadratic Lagrange basis function $7$, $\lbfn{7}{\elemcoordone,\elemcoordtwo}$.}{subfig:biquadlagrangebfuns7}{0.30\linewidth}{9}{2.5}{}
   \hfil
   \gnuplotsubfigure{BasisFunctions/gnuplots/biquadlagrangebfuns8.gnuplot}{Biquadratic Lagrange basis function $8$.}
     {Biquadratic Lagrange basis function $8$, $\lbfn{8}{\elemcoordone,\elemcoordtwo}$.}{subfig:biquadlagrangebfuns8}{0.30\linewidth}{9}{2.5}{}
   \hfil
   \gnuplotsubfigure{BasisFunctions/gnuplots/biquadlagrangebfuns9.gnuplot}{Biquadratic Lagrange basis function $9$.}
     {Biquadratic Lagrange basis function $9$, $\lbfn{9}{\elemcoordone,\elemcoordtwo}$.}{subfig:biquadlagrangebfuns9}{0.30\linewidth}{9}{2.5}{}
   \caption[Biquadratic Lagrange basis functions.]{Biquadratic Lagrange basis functions.}
   \label{fig:biquadlagrangebfuns}
\end{figure}


\subsubsection{Bicubic Lagrange basis functions}

The two-dimensional bicubic Lagrange basis functions have sixteen local nodes
with the basis functions given by
\begin{equation}
  \begin{split}
    \lbfn{1}{\elemcoordone,\elemcoordtwo}&=\lbfn{1}{\elemcoordone}\lbfn{1}{\elemcoordtwo}=
    \frac14\pbrac{3\elemcoordone-1}\pbrac{3\elemcoordone-2}\pbrac{1-\elemcoordone}\pbrac{3\elemcoordtwo-1}\pbrac{3\elemcoordtwo-2}\pbrac{1-\elemcoordtwo} \\
    \lbfn{2}{\elemcoordone,\elemcoordtwo}&=\lbfn{2}{\elemcoordone}\lbfn{1}{\elemcoordtwo}=
    \frac94\elemcoordone\pbrac{3\elemcoordone-2}\pbrac{\elemcoordone-1}\pbrac{3\elemcoordtwo-1}\pbrac{3\elemcoordtwo-2}\pbrac{1-\elemcoordtwo} \\
    \lbfn{3}{\elemcoordone,\elemcoordtwo}&=\lbfn{3}{\elemcoordone}\lbfn{1}{\elemcoordtwo}=
    \frac94\elemcoordone\pbrac{3\elemcoordone-1}\pbrac{1-\elemcoordone}\pbrac{3\elemcoordtwo-1}\pbrac{3\elemcoordtwo-2}\pbrac{1-\elemcoordtwo} \\
    \lbfn{4}{\elemcoordone,\elemcoordtwo}&=\lbfn{4}{\elemcoordone}\lbfn{1}{\elemcoordtwo}=
    \frac14\elemcoordone\pbrac{3\elemcoordone-1}\pbrac{3\elemcoordone-2}\pbrac{3\elemcoordtwo-1}\pbrac{3\elemcoordtwo-2}\pbrac{1-\elemcoordtwo} \\
    \lbfn{5}{\elemcoordone,\elemcoordtwo}&=\lbfn{1}{\elemcoordone}\lbfn{2}{\elemcoordtwo}=
    \frac94\pbrac{3\elemcoordone-1}\pbrac{3\elemcoordone-2}\pbrac{1-\elemcoordone}\elemcoordtwo\pbrac{3\elemcoordtwo-2}\pbrac{\elemcoordtwo-1} \\
    \lbfn{6}{\elemcoordone,\elemcoordtwo}&=\lbfn{2}{\elemcoordone}\lbfn{2}{\elemcoordtwo}=
    \frac{81}{4}\elemcoordone\pbrac{3\elemcoordone-2}\pbrac{\elemcoordone-1}\elemcoordtwo\pbrac{3\elemcoordtwo-2}\pbrac{\elemcoordtwo-1} \\
    \lbfn{7}{\elemcoordone,\elemcoordtwo}&=\lbfn{3}{\elemcoordone}\lbfn{2}{\elemcoordtwo}=
    \frac{81}{4}\elemcoordone\pbrac{3\elemcoordone-1}\pbrac{1-\elemcoordone}\elemcoordtwo\pbrac{3\elemcoordtwo-2}\pbrac{\elemcoordtwo-1} \\
    \lbfn{8}{\elemcoordone,\elemcoordtwo}&=\lbfn{4}{\elemcoordone}\lbfn{2}{\elemcoordtwo}=
    \frac94\elemcoordone\pbrac{3\elemcoordone-1}\pbrac{3\elemcoordone-2}\elemcoordtwo\pbrac{3\elemcoordtwo-2}\pbrac{\elemcoordtwo-1} \\
    \lbfn{9}{\elemcoordone,\elemcoordtwo}&=\lbfn{1}{\elemcoordone}\lbfn{3}{\elemcoordtwo}=
    \frac94\pbrac{3\elemcoordone-1}\pbrac{3\elemcoordone-2}\pbrac{1-\elemcoordone}\elemcoordtwo\pbrac{3\elemcoordtwo-2}\pbrac{1-\elemcoordtwo} \\
    \lbfn{10}{\elemcoordone,\elemcoordtwo}&=\lbfn{2}{\elemcoordone}\lbfn{3}{\elemcoordtwo}=
    \frac{81}{4}\elemcoordone\pbrac{3\elemcoordone-2}\pbrac{\elemcoordone-1}\elemcoordtwo\pbrac{3\elemcoordtwo-2}\pbrac{1-\elemcoordtwo} \\
    \lbfn{11}{\elemcoordone,\elemcoordtwo}&=\lbfn{3}{\elemcoordone}\lbfn{3}{\elemcoordtwo}=
    \frac{81}{4}\elemcoordone\pbrac{3\elemcoordone-1}\pbrac{1-\elemcoordone}\elemcoordtwo\pbrac{3\elemcoordtwo-2}\pbrac{1-\elemcoordtwo} \\
    \lbfn{12}{\elemcoordone,\elemcoordtwo}&=\lbfn{4}{\elemcoordone}\lbfn{3}{\elemcoordtwo}=
    \frac94\elemcoordone\pbrac{3\elemcoordone-1}\pbrac{3\elemcoordone-2}\elemcoordtwo\pbrac{3\elemcoordtwo-2}\pbrac{1-\elemcoordtwo} \\
    \lbfn{13}{\elemcoordone,\elemcoordtwo}&=\lbfn{1}{\elemcoordone}\lbfn{4}{\elemcoordtwo}=
    \frac14\pbrac{3\elemcoordone-1}\pbrac{3\elemcoordone-2}\pbrac{1-\elemcoordone}\elemcoordtwo\pbrac{3\elemcoordtwo-1}\pbrac{3\elemcoordtwo-2}\\
    \lbfn{14}{\elemcoordone,\elemcoordtwo}&=\lbfn{2}{\elemcoordone}\lbfn{4}{\elemcoordtwo}=
    \frac94\elemcoordone\pbrac{3\elemcoordone-2}\pbrac{\elemcoordone-1}\elemcoordtwo\pbrac{3\elemcoordtwo-1}\pbrac{3\elemcoordtwo-2} \\
    \lbfn{15}{\elemcoordone,\elemcoordtwo}&=\lbfn{3}{\elemcoordone}\lbfn{4}{\elemcoordtwo}=
    \frac94\elemcoordone\pbrac{3\elemcoordone-1}\pbrac{1-\elemcoordone}\elemcoordtwo\pbrac{3\elemcoordtwo-1}\pbrac{3\elemcoordtwo-2} \\
    \lbfn{16}{\elemcoordone,\elemcoordtwo}&=\lbfn{4}{\elemcoordone}\lbfn{4}{\elemcoordtwo}=
    \frac14\elemcoordone\pbrac{3\elemcoordone-1}\pbrac{3\elemcoordone-2}\elemcoordtwo\pbrac{3\elemcoordtwo-1}\pbrac{3\elemcoordtwo-2}
  \end{split}
\end{equation}

The sixteen \twodal bicubic Lagrange basis functions are shown in \figref{fig:bicubiclagrangebfuns}.

%\pstexfigure{BasisFunctions/plots/bicubiclagrangebfuns.pstex}{Bicubic Lagrange basis functions.}
%            {Bicubic Lagrange basis functions.}{fig:bicubiclagrangebfuns}
            
\begin{figure}[hbtp]
   \centering
   \gnuplotsubfigure{BasisFunctions/gnuplots/bicubiclagrangebfuns1.gnuplot}{Bicubic Lagrange basis function $1$.}
     {Bicubic Lagrange basis function $1$, $\lbfn{1}{\elemcoordone,\elemcoordtwo}$.}{subfig:bicubiclagrangebfuns1}{0.20\linewidth}{6}{1.75}{1.5}
   \hfil
   \gnuplotsubfigure{BasisFunctions/gnuplots/bicubiclagrangebfuns2.gnuplot}{Bicubic Lagrange basis function $2$.}
     {Bicubic Lagrange basis function $2$, $\lbfn{2}{\elemcoordone,\elemcoordtwo}$.}{subfig:bicubiclagrangebfuns2}{0.20\linewidth}{6}{1.75}{1.5}
   \hfil
   \gnuplotsubfigure{BasisFunctions/gnuplots/bicubiclagrangebfuns3.gnuplot}{Bicubic Lagrange basis function $3$.}
     {Bicubic Lagrange basis function $3$, $\lbfn{3}{\elemcoordone,\elemcoordtwo}$.}{subfig:bicubiclagrangebfuns3}{0.20\linewidth}{6}{1.75}{1.5}
   \hfil
   \gnuplotsubfigure{BasisFunctions/gnuplots/bicubiclagrangebfuns4.gnuplot}{Bicubic Lagrange basis function $4$.}
     {Bicubic Lagrange basis function $4$, $\lbfn{4}{\elemcoordone,\elemcoordtwo}$.}{subfig:bicubiclagrangebfuns4}{0.20\linewidth}{6}{1.75}{1.5}
   \gnuplotsubfigure{BasisFunctions/gnuplots/bicubiclagrangebfuns5.gnuplot}{Bicubic Lagrange basis function $5$.}
     {Bicubic Lagrange basis function $5$, $\lbfn{5}{\elemcoordone,\elemcoordtwo}$.}{subfig:bicubiclagrangebfuns5}{0.20\linewidth}{6}{1.75}{1.5}
   \hfil
   \gnuplotsubfigure{BasisFunctions/gnuplots/bicubiclagrangebfuns6.gnuplot}{Bicubic Lagrange basis function $6$.}
     {Bicubic Lagrange basis function $6$, $\lbfn{6}{\elemcoordone,\elemcoordtwo}$.}{subfig:bicubiclagrangebfuns6}{0.20\linewidth}{6}{1.75}{1.5}
   \hfil
   \gnuplotsubfigure{BasisFunctions/gnuplots/bicubiclagrangebfuns7.gnuplot}{Bicubic Lagrange basis function $7$.}
     {Bicubic Lagrange basis function $7$, $\lbfn{7}{\elemcoordone,\elemcoordtwo}$.}{subfig:bicubiclagrangebfuns7}{0.20\linewidth}{6}{1.75}{1.5}
   \hfil
   \gnuplotsubfigure{BasisFunctions/gnuplots/bicubiclagrangebfuns8.gnuplot}{Bicubic Lagrange basis function $8$.}
     {Bicubic Lagrange basis function $8$, $\lbfn{8}{\elemcoordone,\elemcoordtwo}$.}{subfig:bicubiclagrangebfuns8}{0.20\linewidth}{6}{1.75}{1.5}
   \gnuplotsubfigure{BasisFunctions/gnuplots/bicubiclagrangebfuns9.gnuplot}{Bicubic Lagrange basis function $9$.}
     {Bicubic Lagrange basis function $9$, $\lbfn{9}{\elemcoordone,\elemcoordtwo}$.}{subfig:bicubiclagrangebfuns9}{0.20\linewidth}{6}{1.75}{1.5}
   \hfil
   \gnuplotsubfigure{BasisFunctions/gnuplots/bicubiclagrangebfuns10.gnuplot}{Bicubic Lagrange basis function $10$.}
     {Bicubic Lagrange basis function $10$, $\lbfn{10}{\elemcoordone,\elemcoordtwo}$.}{subfig:bicubiclagrangebfuns10}{0.20\linewidth}{6}{1.75}{1.5}
   \hfil
   \gnuplotsubfigure{BasisFunctions/gnuplots/bicubiclagrangebfuns11.gnuplot}{Bicubic Lagrange basis function $11$.}
     {Bicubic Lagrange basis function $11$, $\lbfn{11}{\elemcoordone,\elemcoordtwo}$.}{subfig:bicubiclagrangebfuns11}{0.20\linewidth}{6}{1.75}{1.5}
   \hfil
   \gnuplotsubfigure{BasisFunctions/gnuplots/bicubiclagrangebfuns12.gnuplot}{Bicubic Lagrange basis function $12$.}
     {Bicubic Lagrange basis function $12$, $\lbfn{12}{\elemcoordone,\elemcoordtwo}$.}{subfig:bicubiclagrangebfuns12}{0.20\linewidth}{6}{1.75}{1.5}
   \gnuplotsubfigure{BasisFunctions/gnuplots/bicubiclagrangebfuns13.gnuplot}{Bicubic Lagrange basis function $13$.}
     {Bicubic Lagrange basis function $13$, $\lbfn{13}{\elemcoordone,\elemcoordtwo}$.}{subfig:bicubiclagrangebfuns13}{0.20\linewidth}{6}{1.75}{1.5}
   \hfil
   \gnuplotsubfigure{BasisFunctions/gnuplots/bicubiclagrangebfuns14.gnuplot}{Bicubic Lagrange basis function $14$.}
     {Bicubic Lagrange basis function $14$, $\lbfn{14}{\elemcoordone,\elemcoordtwo}$.}{subfig:bicubiclagrangebfuns14}{0.20\linewidth}{6}{1.75}{1.5}
   \hfil
   \gnuplotsubfigure{BasisFunctions/gnuplots/bicubiclagrangebfuns15.gnuplot}{Bicubic Lagrange basis function $15$.}
     {Bicubic Lagrange basis function $15$, $\lbfn{15}{\elemcoordone,\elemcoordtwo}$.}{subfig:bicubiclagrangebfuns15}{0.20\linewidth}{6}{1.75}{1.5}
   \hfil
   \gnuplotsubfigure{BasisFunctions/gnuplots/bicubiclagrangebfuns16.gnuplot}{Bicubic Lagrange basis function $16$.}
     {Bicubic Lagrange basis function $16$, $\lbfn{16}{\elemcoordone,\elemcoordtwo}$.}{subfig:bicubiclagrangebfuns16}{0.20\linewidth}{6}{1.75}{1.5}
   \caption[Bicubic Lagrange basis functions.]{Bicubic Lagrange basis functions.}
   \label{fig:bicubiclagrangebfuns}
\end{figure}

\subsection{Hermitian Basis Functions}
\label{sec:Hermitianbasisfunctions}

Hermitian\footnote{named after
\link{https://en.wikipedia.org/wiki/Charles_Hermite}{Charles Hermite}
(1822-1901), a French mathematician.} basis functions preserve
continuity of the derivative of the interpolating variable \ie $C^{1}$
continuity, with respect to $\elementcoordinatesymbol$ across element
boundaries by defining additional nodal derivative parameters. Like
Lagrange bases, Hermitian basis functions are also chosen so that, at
a particular node, only one basis function is non-zero and equal to
one. They also are chosen so that, at a particular node,
the \emph{derivative} of only one of four basis functions is non-zero
and is equal to one. Hermitian basis functions hence serve to weight
the interpolated solution in terms of the field value and derivative
of the field value at nodes.

\subsubsection{Cubic Hermite basis functions}

\Cubicherm basis functions are the simplest of the Hermitian family and
involve two local nodes per element. The interpolation within each element is
in terms of $\nodept{\coordinatevector}{\alpha}$ and \evalat{\dby{\coordinatevector}{\elementcoordinatesymbol}}{\alpha}
and is given by \index{cubic Hermite basis!$\elementcoordinatesymbol$ interpolation formula}
\begin{equation}
  \fnof{\coordinatevector}{\elementcoordinatesymbol}=\chbfn{1}{0}{\elementcoordinatesymbol}\nodept{\coordinatevector}{1}+\chbfn{1}{1}{\elementcoordinatesymbol}
  \evalat{\dby{\coordinatevector}{\elementcoordinatesymbol}}{1}+\chbfn{2}{0}{\elementcoordinatesymbol}\nodept{\coordinatevector}{2}+
  \chbfn{2}{1}{\elementcoordinatesymbol}\evalat{\dby{\coordinatevector}{\elementcoordinatesymbol}}{2}
  \label{eqn:chxiinterpolation}
\end{equation}
where the four \onedal \cubicherm basis functions are given in 
\eqnref{eqn:chbfuns} and shown in \figref{fig:chbfuns}.
\index{cubic Hermite basis!basis functions formulae}
\begin{equation}
  \begin{split}
    \chbfn{1}{0}{\elementcoordinatesymbol} &= 1-3\elementcoordinatesymbol^{2}+2\elementcoordinatesymbol^{3} \\
    \chbfn{1}{1}{\elementcoordinatesymbol} &= \elementcoordinatesymbol\pbrac{\elementcoordinatesymbol-1}^{2} \\
    \chbfn{2}{0}{\elementcoordinatesymbol} &= \elementcoordinatesymbol^{2}\pbrac{3-2\elementcoordinatesymbol} \\
    \chbfn{2}{1}{\elementcoordinatesymbol} &= \elementcoordinatesymbol^{2}\pbrac{\elementcoordinatesymbol-1}
  \end{split}
  \label{eqn:chbfuns}
\end{equation}
\pstexfigure{BasisFunctions/plots/chbfuns.pstex}{Cubic Hermite basis functions.}
{Cubic Hermite basis functions.}{fig:chbfuns}

\subsubsection{Scaling}

One further step is required to make \cubicherm basis functions useful in
practice.  Consider the two \cubicherm elements shown in
\figref{fig:chelements}.

\epstexfigure{BasisFunctions/svgs/cubichermiteelem.eps_tex}{Two
  cubic Hermite elements formed from three nodes.}{Two cubic Hermite elements
  (denoted by $\mathit{1}$ and $\mathit{2}$) formed from three nodes (shown as
  a $\bullet$ and denoted by $\mathbf{1}, \mathbf{2}$ and $\mathbf{3}$) and
  having \arclens $\arclengthcoordinate{1}$ and $\arclengthcoordinate{2}$ respectively.}{fig:chelements}{0.50}

The derivative $\evalat{\dby{\coordinatevector}{\elementcoordinatesymbol}}{\alpha}$ defined at local node
$\alpha$ is dependent upon the local element \xicoord and is therefore, in
general, different in the two adjacent elements. Interpretation of the
derivative is hence difficult as two derivatives with the same magnitude in
different parts of the mesh might represent two completely different physical
derivatives. This is problematic for modelling and computation if the interpretation of the
magnitude of the derivative (or \emph{scaling}) is unknown \eg we cannot
assign physical units. If the scaling varies throughout the mesh then a
derivative at a node that has a magnitude of, say, 5 will be different from
another derivative at another node that also has the magnitude of 5. Thus, a
numerical solver that is given a vector of derivative values would assume that
the scalings are the same and interpret the magnitudes identically. This would
mean that algorithms may fail \eg if, say, we needed to compute the
norm of a vector of derivatives then by assuming the same scaling the wrong
result would be computed.

In order to the have a consistent interpretation of the derivative
throughout the mesh it is better to base the interpolation on a physical
coordinate. Whilst we are free to choose the physical coordinate to be
anything the optimum choice is arc length as this is what physical processes
are based on. However, arc-length is extremely difficult to use as an
interpolation parameter as the inherent nonlinearity involved in its
calculation makes conversion to and from coordinates non trivial.
The solution is to find a parameter that scales
in the same way as arc-length or as close to it as we can. 

Consider then basing the derivatives on an \arclen coordinate at nodes,
$\dby{\nodept{\coordinatevector}{\alpha}}{s}$, with
\begin{equation}
  \begin{split}
    \evalat{\dby{\coordinatevector}{\elementcoordinatesymbol}}{\alpha}&=\dby{\nodept{\coordinatevector}{
        \fnof{\Delta}{\alpha,e}}}{s}\pbrac{\dby{s}{\elementcoordinatesymbol}}_{e} \\ &=
    \dby{\nodept{\coordinatevector}{\fnof{\Delta}{\alpha,e}}}{s}\esfone{e}
  \end{split}
  \label{eqn:xitosch}
\end{equation}
used to determine $\evalat{\dby{\coordinatevector}{\elementcoordinatesymbol}}{\alpha}$. Here
$\dby{\coordinatevector}{s}$ is a physical \arclen derivative,
$\fnof{\Delta}{\alpha,e}$ is the global node number of local node $\alpha$ in
element $e$, $\pbrac{\dby{s}{\elementcoordinatesymbol}}_{e}$ is an \index{element scale
  factor}element \emph{scale factor}, denoted by $\esfone{e}$, which scales
the \arclen derivative to the \xicoord derivative.  Thus $\dby{\coordinatevector}{s}$
is constrained to be continuous across element boundaries rather than
$\dby{\coordinatevector}{\elementcoordinatesymbol}$. The \cubicherm interpolation formula now becomes
\begin{equation}
  \fnof{\coordinatevector}{\elementcoordinatesymbol}=\chbfn{1}{0}{\elementcoordinatesymbol}\nodept{\coordinatevector}{1}+\chbfn{1}{1}{\elementcoordinatesymbol}
  \dby{\nodept{\coordinatevector}{1}}{s}\esfone{e}+\chbfn{2}{0}{\elementcoordinatesymbol}\nodept{\coordinatevector}{2}+
  \chbfn{2}{1}{\elementcoordinatesymbol}\dby{\nodept{\coordinatevector}{2}}{s}\esfone{e}
  \label{eqn:chseinterpolation}
\end{equation}

By interpolating with respect to $\arclengthcoordinatesymbol$ rather than with respect to $\elementcoordinatesymbol$ there is some
liberty as to the choice of the element scale factor, $\esfone{e}$. The choice
of the scale factor will, however, affect how $\elementcoordinatesymbol$ changes with $\arclengthcoordinatesymbol$.  It is
computationally desirable to have a relatively uniform change of $\elementcoordinatesymbol$ with
$\arclengthcoordinatesymbol$ (for example not biasing the Gaussian quadrature -- see later -- scheme to
one end of the element). For this reason the element scale factor is chosen as
some function of the \arclen of the element, $s_{e}$. The simplest linear
function that can be chosen is the \arclen itself. This type of scaling is
called \index{arc-length scaling}\emph{\arclen scaling}.

To calculate the \arclen for a particular element an iterative process is
needed. The \arclen for a \onedal element in \twods is defined as
\index{arc-length definition}
\begin{equation}
  \text{\arclen, }s_{e}=\gint{0}{1}{\norm{\dby{\fnof{\coordinatevector}{\elementcoordinatesymbol}}{\elementcoordinatesymbol}}}
  {\elementcoordinatesymbol}=\gint{0}{1}{\sqrt{\pbrac{\dby{\fnof{x}{\elementcoordinatesymbol}}{\elementcoordinatesymbol}}^{2}+
      \pbrac{\dby{\fnof{y}{\elementcoordinatesymbol}}{\elementcoordinatesymbol}}^{2}}}{\elementcoordinatesymbol}
  \label{eqn:arclendef}
\end{equation}

However, since the interpolation of $\fnof{\coordinatevector}{\elementcoordinatesymbol}$, as defined in
\eqnref{eqn:chseinterpolation}, uses the \arclen in the calculation of the
scaling factor, an iterative root finding technique is needed to obtain the
\arclen.

Thus, for an element $e$, the \onedal \cubicherm interpolation
formula in \eqnref{eqn:chseinterpolation} becomes
\begin{equation}
  \fnof{\coordinatevector}{\elementcoordinatesymbol}=\chbfn{\alpha}{u}{\elementcoordinatesymbol}\nodept{\coordinatevector}{\alpha}_{,u}
  \esftwo{e}{u}
  \label{eqn:chsfinterpolation}
\end{equation}
where $\alpha$ varies from $1$ to $2$, $u$ varies from $0$ to $1$,
$\nodept{\coordinatevector}{\alpha}_{,0}=\nodept{\coordinatevector}{\alpha}$,
$\nodept{\coordinatevector}{\alpha}_{,1}= \dby{\nodept{\coordinatevector}{\alpha}}{s}$,
$\esftwo{e}{0}=1$ and $\esftwo{e}{1}=\esfone{e}=s_{e}$. \Eqnref{eqn:chsfinterpolation} is equivalent to
\begin{equation}
  \fnof{\coordinatevector}{\elementcoordinatesymbol}=\chbfn{1}{0}{\elementcoordinatesymbol}\nodept{\coordinatevector}{1}_{,0}\esftwo{e}{0}
  +\chbfn{1}{1}{\elementcoordinatesymbol}\nodept{\coordinatevector}{1}_{,1}\esftwo{e}{1}+
  \chbfn{2}{0}{\elementcoordinatesymbol}\nodept{\coordinatevector}{2}_{,0}\esftwo{e}{0}
  +\chbfn{2}{1}{\elementcoordinatesymbol}\nodept{\coordinatevector}{2}_{,1}\esftwo{e}{1}
\end{equation}
\ie there is an implied sum with $\alpha$ and $u$ for $\chbfn{\alpha}{u}{\elementcoordinatesymbol}$
and $\nodept{\coordinatevector}{\alpha}_{,u}$ but not for $\esftwo{e}{u}$.

There is one final condition that must be placed on the $\elementcoordinatesymbol$ to \arclen
transformation to ensure \arclen derivatives. This condition is based on the
geometric defintion of \arclen which is given by Pythagorus \ie for \twods in
rectangular cartesian coordinate we have
\begin{equation}
  ds^{2}=dx^{2}+dy^{2}
  \label{eqn:arclengthpythagorus}
\end{equation}
or, in general coordinates,
\begin{equation}
  ds^{2}=g_{ij}dx^{i}dx^{j}
  \label{eqn:genarclengthpythagorus}
\end{equation}
where $g_{ij}$ are the components of the metric tensor.

Rearranging \eqnref{eqn:arclengthpythagorus} we find that the \arclen derivative vector at a
node for geometric like fields, for rectangular cartesian coordinates, must
have unit magnitude. Thus for global node $A$ we have
\begin{equation}
  \norm{\dby{\nodept{\coordinatevector}{A}}{s}}=1
  \label{eqn:chnormconstraint}
\end{equation}

In general coordinates this condition becomes
\begin{equation}
  \norm{\delby{\nodept{\coordinatevector}{A}}{\arclengthcoordinate{k}}}=\sqrt{\det{\generalmetrictensor}}
  \label{eqn:genchnormconstraint}
\end{equation}
where $\arclengthcoordinate{k}$ is the \nth{k} global arc-length direction and $\generalmetrictensor$ is the metric tensor.

The use of this constraint on \arclen derivative magnitude ensures that there is continuity with respect to a physical parameter,
$\arclengthcoordinatesymbol$, rather than with respect to a mathematical parameter $\elementcoordinatesymbol$. The set of
mesh parameters, $\vect{u}$, for \cubicherm interpolation hence contains the
set of nodal values (or positions), the set of nodal \arclen derivatives and
the set of scale factors.

\subsubsection{Bicubic Hermite Basis Functions}

\Bicubicherm basis functions are the \twodal extension of the \onedal
\cubicherm basis functions. They are formed from the tensor (or outer) product
of two of the \onedal cubic Hermite basis functions defined in
\eqnref{eqn:chbfuns}. The $16$ \bicubicherm basis functions are given in \eqnref{eqn:BicubicHermBFuns}.

\begin{equation}
  \begin{split}
    \chbfn{1}{0}{\elemcoordone,\elemcoordtwo}&=\chbfn{1}{0}{\elemcoordone}\chbfn{1}{0}{\elemcoordtwo}=\pbrac{1-3\elemcoordonesq+2\elemcoordonecube}\pbrac{1-3\elemcoordtwosq+2\elemcoordtwocube}\\
    \chbfn{2}{0}{\elemcoordone,\elemcoordtwo}&=\chbfn{2}{0}{\elemcoordone}\chbfn{1}{0}{\elemcoordtwo}=\pbrac{\elemcoordone(\elemcoordone-1)^{2}}\pbrac{1-3\elemcoordtwosq+2\elemcoordtwocube}\\
    \chbfn{3}{0}{\elemcoordone,\elemcoordtwo}&=\chbfn{1}{0}{\elemcoordone}\chbfn{2}{0}{\elemcoordtwo}=\pbrac{1-3\elemcoordonesq+2\elemcoordonecube}\pbrac{\elemcoordtwosq\pbrac{3-2\elemcoordtwo}}\\
    \chbfn{4}{0}{\elemcoordone,\elemcoordtwo}&=\chbfn{2}{0}{\elemcoordone}\chbfn{2}{0}{\elemcoordtwo}=\pbrac{\elemcoordonesq\pbrac{3-2\elemcoordone}}\pbrac{\elemcoordtwosq\pbrac{3-2\elemcoordtwo}}\\
    \chbfn{1}{1}{\elemcoordone,\elemcoordtwo}&=\chbfn{1}{1}{\elemcoordone}\chbfn{1}{0}{\elemcoordtwo}=\pbrac{\elemcoordone\pbrac{\elemcoordone-1}^{2}}\pbrac{1-3\elemcoordtwosq+2\elemcoordtwocube}\\
    \chbfn{2}{1}{\elemcoordone,\elemcoordtwo}&=\chbfn{2}{1}{\elemcoordone}\chbfn{1}{0}{\elemcoordtwo}=\pbrac{\elemcoordonesq\pbrac{\elemcoordone-1}}\pbrac{1-3\elemcoordtwosq+2\elemcoordtwocube}\\
    \chbfn{3}{1}{\elemcoordone,\elemcoordtwo}&=\chbfn{1}{1}{\elemcoordone}\chbfn{2}{0}{\elemcoordtwo}=\pbrac{\elemcoordone\pbrac{\elemcoordone-1}^{2}}\pbrac{\elemcoordtwosq\pbrac{3-2\elemcoordtwo}}\\
    \chbfn{4}{1}{\elemcoordone,\elemcoordtwo}&=\chbfn{2}{1}{\elemcoordone}\chbfn{2}{0}{\elemcoordtwo}=\pbrac{\elemcoordonesq\pbrac{\elemcoordone-1}}\pbrac{\elemcoordtwosq\pbrac{3-2\elemcoordtwo}}\\
    \chbfn{1}{2}{\elemcoordone,\elemcoordtwo}&=\chbfn{1}{0}{\elemcoordone}\chbfn{1}{1}{\elemcoordtwo}=\pbrac{1-3\elemcoordonesq+2\elemcoordonecube}\pbrac{\elemcoordtwo\pbrac{\elemcoordtwo-1}^{2}}\\
    \chbfn{2}{2}{\elemcoordone,\elemcoordtwo}&=\chbfn{2}{0}{\elemcoordone}\chbfn{1}{1}{\elemcoordtwo}=\pbrac{\elemcoordonesq\pbrac{3-2\elemcoordone}}\pbrac{\elemcoordtwo\pbrac{\elemcoordtwo-1}^{2}}\\
    \chbfn{3}{2}{\elemcoordone,\elemcoordtwo}&=\chbfn{1}{0}{\elemcoordone}\chbfn{2}{1}{\elemcoordtwo}=\pbrac{1-3\elemcoordonesq+2\elemcoordonecube}\pbrac{\elemcoordtwosq\pbrac{\elemcoordtwo-1}}\\
    \chbfn{4}{2}{\elemcoordone,\elemcoordtwo}&=\chbfn{2}{0}{\elemcoordone}\chbfn{2}{1}{\elemcoordtwo}=\pbrac{\elemcoordonesq\pbrac{3-2\elemcoordone}}\pbrac{\elemcoordtwosq\pbrac{\elemcoordtwo-1}}\\
    \chbfn{1}{3}{\elemcoordone,\elemcoordtwo}&=\chbfn{1}{1}{\elemcoordone}\chbfn{1}{1}{\elemcoordtwo}=\pbrac{\elemcoordone\pbrac{\elemcoordone-1}^{2}}\pbrac{\elemcoordtwo\pbrac{\elemcoordtwo-1}^{2}}\\
    \chbfn{2}{3}{\elemcoordone,\elemcoordtwo}&=\chbfn{2}{1}{\elemcoordone}\chbfn{1}{1}{\elemcoordtwo}=\pbrac{\elemcoordonesq\pbrac{\elemcoordone-1}}\pbrac{\elemcoordtwo\pbrac{\elemcoordtwo-1}^{2}}\\
    \chbfn{3}{3}{\elemcoordone,\elemcoordtwo}&=\chbfn{1}{1}{\elemcoordone}\chbfn{2}{1}{\elemcoordtwo}=\pbrac{\elemcoordone\pbrac{\elemcoordone-1}^{2}}\pbrac{\elemcoordtwosq\pbrac{\elemcoordtwo-1}}\\
    \chbfn{4}{3}{\elemcoordone,\elemcoordtwo}&=\chbfn{2}{1}{\elemcoordone}\chbfn{2}{1}{\elemcoordtwo}=\pbrac{\elemcoordonesq\pbrac{\elemcoordone-1}}\pbrac{\elemcoordtwosq\pbrac{\elemcoordtwo-1}}    
  \end{split}
  \label{eqn:BicubicHermBFuns}
\end{equation}

The interpolation formula for the point
$\fnof{\coordinatevector}{\elemcoordone,\elemcoordtwo}$ within an element is obtained from the
\bicubicherm interpolation formula \citep{nielsen:1991a}, \index{bicubic
  Hermite basis!$\elementcoordinatesymbol$ interpolation formula}
\begin{equation}
  \begin{split}
    \fnof{\coordinatevector}{\elemcoordone,\elemcoordtwo} &=
    \chbfn{1}{0}{\elemcoordone,\elemcoordtwo}\nodept{\coordinatevector}{1} +
    \chbfn{2}{0}{\elemcoordone,\elemcoordtwo}\nodept{\coordinatevector}{2} +
    \chbfn{3}{0}{\elemcoordone,\elemcoordtwo}\nodept{\coordinatevector}{3} +
    \chbfn{4}{0}{\elemcoordone,\elemcoordtwo}\nodept{\coordinatevector}{4} + \\
    & \chbfn{1}{1}{\elemcoordone,\elemcoordtwo}\evalat{\delby{\coordinatevector}{\elemcoordone}}{1}+
    \chbfn{2}{1}{\elemcoordone,\elemcoordtwo}\evalat{\delby{\coordinatevector}{\elemcoordone}}{2}+ 
    \chbfn{3}{1}{\elemcoordone,\elemcoordtwo}\evalat{\delby{\coordinatevector}{\elemcoordone}}{3}+
    \chbfn{4}{1}{\elemcoordone,\elemcoordtwo}\evalat{\delby{\coordinatevector}{\elemcoordone}}{4} + \\ 
    & \chbfn{1}{2}{\elemcoordone,\elemcoordtwo}\evalat{\delby{\coordinatevector}{\elemcoordtwo}}{1}+
    \chbfn{2}{2}{\elemcoordone,\elemcoordtwo}\evalat{\delby{\coordinatevector}{\elemcoordtwo}}{2} + 
    \chbfn{3}{2}{\elemcoordone,\elemcoordtwo}\evalat{\delby{\coordinatevector}{\elemcoordtwo}}{3}+
    \chbfn{4}{2}{\elemcoordone,\elemcoordtwo}\evalat{\delby{\coordinatevector}{\elemcoordtwo}}{4} + \\ 
    & \chbfn{1}{3}{\elemcoordone,\elemcoordtwo}\evalat{\deltwoby{\coordinatevector}{\elemcoordone}{\elemcoordtwo}}{1} +
    \chbfn{2}{3}{\elemcoordone,\elemcoordtwo}\evalat{\deltwoby{\coordinatevector}{\elemcoordone}{\elemcoordtwo}}{2} +
    \chbfn{3}{3}{\elemcoordone,\elemcoordtwo}\evalat{\deltwoby{\coordinatevector}{\elemcoordone}{\elemcoordtwo}}{3} + 
    \chbfn{4}{3}{\elemcoordone,\elemcoordtwo}\evalat{ \deltwoby{\coordinatevector}{\elemcoordone}{\elemcoordtwo}}{4}    
  \end{split}
  \label{eqn:bichxiinterp}
\end{equation}

The sixteen \twodal bicubic Hermite basis functions are shown in \figref{fig:BicubicHermiteBFuns}.
            
\begin{figure}[hbtp]
   \centering
   \gnuplotsubfigure{BasisFunctions/gnuplots/bicubichermitebfuns1.gnuplot}{Bicubic Hermite basis function $1$.}
     {Bicubic Hermite basis function $1$, $\chbfn{1}{0}{\elemcoordone,\elemcoordtwo}$.}{subfig:bicubichermitebfuns1}{0.20\linewidth}{6}{1.75}{1.5}
   \hfil
   \gnuplotsubfigure{BasisFunctions/gnuplots/bicubichermitebfuns2.gnuplot}{Bicubic Hermite basis function $2$.}
     {Bicubic Hermite basis function $2$, $\chbfn{2}{0}{\elemcoordone,\elemcoordtwo}$.}{subfig:bicubichermitebfuns2}{0.20\linewidth}{6}{1.75}{1.5}
   \hfil
   \gnuplotsubfigure{BasisFunctions/gnuplots/bicubichermitebfuns3.gnuplot}{Bicubic Hermite basis function $3$.}
     {Bicubic Hermite basis function $3$, $\chbfn{3}{0}{\elemcoordone,\elemcoordtwo}$.}{subfig:bicubichermitebfuns3}{0.20\linewidth}{6}{1.75}{1.5}
   \hfil
   \gnuplotsubfigure{BasisFunctions/gnuplots/bicubichermitebfuns4.gnuplot}{Bicubic Hermite basis function $4$.}
     {Bicubic Hermite basis function $4$, $\chbfn{4}{0}{\elemcoordone,\elemcoordtwo}$.}{subfig:bicubichermitebfuns4}{0.20\linewidth}{6}{1.75}{1.5}
   \gnuplotsubfigure{BasisFunctions/gnuplots/bicubichermitebfuns5.gnuplot}{Bicubic Hermite basis function $5$.}
     {Bicubic Hermite basis function $5$, $\chbfn{1}{1}{\elemcoordone,\elemcoordtwo}$.}{subfig:bicubichermitebfuns5}{0.20\linewidth}{6}{1.75}{1.5}
   \hfil
   \gnuplotsubfigure{BasisFunctions/gnuplots/bicubichermitebfuns6.gnuplot}{Bicubic Hermite basis function $6$.}
     {Bicubic Hermite basis function $6$, $\chbfn{2}{1}{\elemcoordone,\elemcoordtwo}$.}{subfig:bicubichermitebfuns6}{0.20\linewidth}{6}{1.75}{1.5}
   \hfil
   \gnuplotsubfigure{BasisFunctions/gnuplots/bicubichermitebfuns7.gnuplot}{Bicubic Hermite basis function $7$.}
     {Bicubic Hermite basis function $7$, $\chbfn{3}{1}{\elemcoordone,\elemcoordtwo}$.}{subfig:bicubichermitebfuns7}{0.20\linewidth}{6}{1.75}{1.5}
   \hfil
   \gnuplotsubfigure{BasisFunctions/gnuplots/bicubichermitebfuns8.gnuplot}{Bicubic Hermite basis function $8$.}
     {Bicubic Hermite basis function $8$, $\chbfn{4}{1}{\elemcoordone,\elemcoordtwo}$.}{subfig:bicubichermitebfuns8}{0.20\linewidth}{6}{1.75}{1.5}
   \gnuplotsubfigure{BasisFunctions/gnuplots/bicubichermitebfuns9.gnuplot}{Bicubic Hermite basis function $9$.}
     {Bicubic Hermite basis function $9$, $\chbfn{1}{2}{\elemcoordone,\elemcoordtwo}$.}{subfig:bicubichermitebfuns9}{0.20\linewidth}{6}{1.75}{1.5}
   \hfil
   \gnuplotsubfigure{BasisFunctions/gnuplots/bicubichermitebfuns10.gnuplot}{Bicubic Hermite basis function $10$.}
     {Bicubic Hermite basis function $10$, $\chbfn{2}{2}{\elemcoordone,\elemcoordtwo}$.}{subfig:bicubichermitebfuns10}{0.20\linewidth}{6}{1.75}{1.5}
   \hfil
   \gnuplotsubfigure{BasisFunctions/gnuplots/bicubichermitebfuns11.gnuplot}{Bicubic Hermite basis function $11$.}
     {Bicubic Hermite basis function $11$, $\chbfn{3}{2}{\elemcoordone,\elemcoordtwo}$.}{subfig:bicubichermitebfuns11}{0.20\linewidth}{6}{1.75}{1.5}
   \hfil
   \gnuplotsubfigure{BasisFunctions/gnuplots/bicubichermitebfuns12.gnuplot}{Bicubic Hermite basis function $12$.}
     {Bicubic Hermite basis function $12$, $\chbfn{4}{2}{\elemcoordone,\elemcoordtwo}$.}{subfig:bicubichermitebfuns12}{0.20\linewidth}{6}{1.75}{1.5}
   \gnuplotsubfigure{BasisFunctions/gnuplots/bicubichermitebfuns13.gnuplot}{Bicubic Hermite basis function $13$.}
     {Bicubic Hermite basis function $13$, $\chbfn{1}{3}{\elemcoordone,\elemcoordtwo}$.}{subfig:bicubichermitebfuns13}{0.20\linewidth}{6}{1.75}{1.5}
   \hfil
   \gnuplotsubfigure{BasisFunctions/gnuplots/bicubichermitebfuns14.gnuplot}{Bicubic Hermite basis function $14$.}
     {Bicubic Hermite basis function $14$, $\chbfn{2}{3}{\elemcoordone,\elemcoordtwo}$.}{subfig:bicubichermitebfuns14}{0.20\linewidth}{6}{1.75}{1.5}
   \hfil
   \gnuplotsubfigure{BasisFunctions/gnuplots/bicubichermitebfuns15.gnuplot}{Bicubic Hermite basis function $15$.}
     {Bicubic Hermite basis function $15$, $\chbfn{3}{3}{\elemcoordone,\elemcoordtwo}$.}{subfig:bicubichermitebfuns15}{0.20\linewidth}{6}{1.75}{1.5}
   \hfil
   \gnuplotsubfigure{BasisFunctions/gnuplots/bicubichermitebfuns16.gnuplot}{Bicubic Hermite basis function $16$.}
     {Bicubic Hermite basis function $16$, $\chbfn{4}{3}{\elemcoordone,\elemcoordtwo}$.}{subfig:bicubichermitebfuns16}{0.20\linewidth}{6}{1.75}{1.5}
   \caption[Bicubic Hermite basis functions.]{Bicubic Hermite basis functions.}
   \label{fig:BicubicHermiteBFuns}
\end{figure}

As with \onedal \cubicherm elements, the derivatives with respect to $\elementcoordinatesymbol$ in
the \twodal interpolation formula above are expressed as the product of a
nodal \arclen derivative and a scale factor. This is, however, complicated by
the fact that there are now multiple $\elementcoordinatesymbol$ directions at each node. From the
product rule the transformation from an $\elementcoordinatesymbol$ based derivative to an \arclen
based derivative is given by,
\begin{equation}
  \delby{\coordinatevector}{\elementcoordinate{l}}=\delby{\coordinatevector}{\arclengthcoordinate{1}}\delby{\arclengthcoordinate{1}}{\elementcoordinate{l}}+
  \delby{\coordinatevector}{\arclengthcoordinate{2}}\delby{\arclengthcoordinate{2}}{\elementcoordinate{l}}
  \label{eqn:xitosproductrule}
\end{equation}

Now, by definition, the $\nth{l}$ \arclen direction is only a function of the
$\nth{l}$ $\elementcoordinatesymbol$ direction, hence the derivative at local node $\alpha$ is
\begin{equation}
  \evalat{\delby{\coordinatevector}{\elementcoordinate{l}}}{\alpha}=\delby{\nodept{\coordinatevector}{
      \fnof{\Delta}{\alpha,e}}}{\arclengthcoordinate{l}}\esftwo{e}{l}
  \label{eqn:xitosbich}
\end{equation}
and the cross-derivative is
\begin{equation}
  \evalat{\deltwoby{\coordinatevector}{\elemcoordone}{\elemcoordtwo}}{\alpha}=
  \deltwoby{\nodept{\coordinatevector}{\fnof{\Delta}{\alpha,e}}}{\arclengthcoordinate{1}}{\arclengthcoordinate{2}}\esftwo{e}{1}
  \esftwo{e}{2}
  \label{eqn:xitosbichcd}
\end{equation}

Unlike the \onedal \cubicherm case a condition must be placed on
this transformation in order to maintain $C^{1}$ continuity across element
boundaries. 

Consider the line between global nodes $\nodenumber{1}$ and $\nodenumber{2}$ in the
two \bicubicherm elements shown in \figref{fig:bichelementcont}.
\epstexfigure{BasisFunctions/svgs/C1bicubicHermite.eps_tex}{Continuity of two bicubic
  Hermite elements.}{Two bicubic Hermite elements (denoted by $\elementnumber{1}$ and
  $\elementnumber{2}$). The global node numbers are given in boldface, the local node
  numbers in normal text and the element scale factors used along each line
  are denoted by $\esfone{l}$.}{fig:bichelementcont}{0.35}

For $C^{1}$ continuity, as opposed to $G^{1}$ continuity, between these
elements the derivative with respect to $\elemcoordone$, that is
\delby{\fnof{\coordinatevector}{\elemcoordtwo}}{\elemcoordone}, must be continuous\footnote{For
  $C^{1}$ continuity the normals either side of an element boundary must be in
  the same direction \emph{and} have the same magnitude. For $G^{1}$
  continuity the normals must only have the same direction.}. The formula for
this derivative in element $\elementnumber{1}$ along the boundary between elements
$\elementnumber{1}$ and $\elementnumber{2}$ is
\begin{equation}
  \delby{\fnof{\coordinatevector}{1,\elemcoordtwo}}{\elemcoordone}=\chbfn{0}{1}{\elemcoordtwo}\evalat{
    \delby{\coordinatevector}{\elemcoordone}}{2}+\chbfn{0}{2}{\elemcoordtwo}\evalat{
    \delby{\coordinatevector}{\elemcoordone}}{4}+\chbfn{1}{1}{\elemcoordtwo}\evalat{
    \deltwoby{\coordinatevector}{\elemcoordone}{\elemcoordtwo}}{2}+\chbfn{1}{2}{\elemcoordtwo}\evalat{
    \deltwoby{\coordinatevector}{\elemcoordone}{\elemcoordtwo}}{4}
  \label{eqn:c1contelem1}
\end{equation}
and for element $\mathit{2}$ is
\begin{equation}
  \delby{\fnof{\coordinatevector}{0,\elemcoordtwo}}{\elemcoordone}=\chbfn{0}{1}{\elemcoordtwo}\evalat{
    \delby{\coordinatevector}{\elemcoordone}}{1}+\chbfn{0}{2}{\elemcoordtwo}\evalat{
    \delby{\coordinatevector}{\elemcoordone}}{3}+\chbfn{1}{1}{\elemcoordtwo}\evalat{
    \deltwoby{\coordinatevector}{\elemcoordone}{\elemcoordtwo}}{1}+\chbfn{1}{2}{\elemcoordtwo}\evalat{
    \deltwoby{\coordinatevector}{\elemcoordone}{\elemcoordtwo}}{3}
  \label{eqn:c1contelem2}
\end{equation}

Now substituting \eqnrefs{eqn:xitosbich}{eqn:xitosbichcd} into the
above equations yields for element $\elementnumber{1}$
\begin{equation}
  \begin{split}
    \delby{\fnof{\coordinatevector}{1,\elemcoordtwo}}{\elemcoordone} &=
    \chbfn{0}{1}{\elemcoordtwo}\delby{\nodept{\coordinatevector}{2}}{\arclengthcoordinate{1}}\esfone{2}+
    \chbfn{0}{2}{\elemcoordtwo}\delby{\nodept{\coordinatevector}{4}}{\arclengthcoordinate{1}}\esfone{5}+ \\
    &\quad\chbfn{1}{1}{\elemcoordtwo}\deltwoby{\nodept{\coordinatevector}{2}}{\arclengthcoordinate{1}}{\arclengthcoordinate{2}}
    \esfone{2}\esfone{4}+
    \chbfn{1}{2}{\elemcoordtwo}\deltwoby{\nodept{\coordinatevector}{4}}{\arclengthcoordinate{1}}{\arclengthcoordinate{2}}
    \esfone{5}\esfone{4}
  \end{split}
\end{equation}
and for element $\elementnumber{2}$
\begin{equation}
  \begin{split}
    \delby{\fnof{\coordinatevector}{0,\elemcoordtwo}}{\elemcoordone} &=
    \chbfn{0}{1}{\elemcoordone}\delby{\nodept{\coordinatevector}{1}}{\arclengthcoordinate{1}}\esfone{3}+
    \chbfn{0}{2}{\elemcoordtwo}\delby{\nodept{\coordinatevector}{3}}{\arclengthcoordinate{1}}\esfone{6}+ \\
    &\quad\chbfn{1}{1}{\elemcoordtwo}\deltwoby{\nodept{\coordinatevector}{1}}{\arclengthcoordinate{1}}{\arclengthcoordinate{2}}
    \esfone{3}\esfone{4}+ 
    \chbfn{1}{2}{\elemcoordtwo}\deltwoby{\nodept{\coordinatevector}{3}}{\arclengthcoordinate{1}}{\arclengthcoordinate{2}}
    \esfone{6}\esfone{4}
  \end{split}
\end{equation}

Now local node $2$ in element $\elementnumber{1}$ and local node $1$ in element
$\elementnumber{2}$ is the same as global node $\nodenumber{1}$ and local node $4$ in
element $\elementnumber{1}$ and local node $3$ in element $\elementnumber{2}$ is the same as
global node $\nodenumber{2}$. Hence for a given $\elemcoordtwo$ the condition for $C^{1}$
continuity across the element boundary is
\begin{multline}
  \chbfn{0}{1}{\elemcoordtwo}\delby{\nodept{\coordinatevector}{\mathbf{1}}}{\arclengthcoordinate{1}}\esfone{2}+
  \chbfn{0}{2}{\elemcoordtwo}\delby{\nodept{\coordinatevector}{\mathbf{2}}}{\arclengthcoordinate{1}}\esfone{5}+ 
  \chbfn{1}{1}{\elemcoordtwo}\deltwoby{\nodept{\coordinatevector}{\mathbf{1}}}{\arclengthcoordinate{1}}{\arclengthcoordinate{2}}
  \esfone{2}\esfone{4} \\
  +\chbfn{1}{2}{\elemcoordtwo}\deltwoby{\nodept{\coordinatevector}{\mathbf{2}}}{\arclengthcoordinate{1}}{\arclengthcoordinate{2}}
  \esfone{5}\esfone{4} = \chbfn{0}{1}{\elemcoordtwo}\delby{\nodept{\coordinatevector}
    {\mathbf{1}}}{\arclengthcoordinate{1}}\esfone{3}+\chbfn{0}{2}{\elemcoordtwo}\delby{\nodept{\coordinatevector}
    {\mathbf{2}}}{\arclengthcoordinate{1}}\esfone{6} \\
  +\chbfn{1}{1}{\elemcoordtwo}\deltwoby{\nodept{\coordinatevector}{\mathbf{1}}}{\arclengthcoordinate{1}}{\arclengthcoordinate{2}}
  \esfone{3}\esfone{4}+\chbfn{1}{2}{\elemcoordtwo}\deltwoby{\nodept{\coordinatevector}
    {\mathbf{2}}}{\arclengthcoordinate{1}}{\arclengthcoordinate{2}}\esfone{6}\esfone{4}
\end{multline}
or
\begin{multline}
  \pbrac{\esfone{2}-\esfone{3}}\pbrac{\chbfn{0}{1}{\elemcoordtwo}
    \delby{\nodept{\coordinatevector}{\mathbf{1}}}{\arclengthcoordinate{1}}+\chbfn{1}{1}{\elemcoordtwo}
    \deltwoby{\nodept{\coordinatevector}{\mathbf{1}}}{\arclengthcoordinate{1}}{\arclengthcoordinate{2}}\esfone{4}} = \\
  \pbrac{\esfone{6}-\esfone{5}}\pbrac{\chbfn{0}{2}{\elemcoordtwo}
    \delby{\nodept{\coordinatevector}{\mathbf{2}}}{\arclengthcoordinate{1}}+\chbfn{1}{2}{\elemcoordtwo}
    \deltwoby{\nodept{\coordinatevector}{\mathbf{2}}}{\arclengthcoordinate{1}}{\arclengthcoordinate{2}}\esfone{4}}
  \label{eqn:bchc1condition}
\end{multline}

Now by choosing the scale factors to be equal on either side of node
$\nodenumber{1}$ and $\nodenumber{2}$ (\ie $\esfone{2}=\esfone{3}=\nsfone{\nodenumber{1}}$
and $\esfone{5}=\esfone{6}=\nsfone{\nodenumber{2}}$), that is nodal based scale
factors, \eqnref{eqn:bchc1condition} is satisfied for any choice of the scale
factors.  Hence nodal scale factors are a sufficient condition to ensure
$C^{1}$ continuity. If it is desired that the scale factors be different
either side of the node then \eqnref{eqn:bchc1condition} must be satisfied to
ensure continuity.

The choice of the scale factors again determines the $\elementcoordinatesymbol$
to $\arclengthcoordinatesymbol$ spacing. We have a number of choices for the scale factor depending on
whether or not the $\elementcoordinatesymbol$ to $\arclengthcoordinatesymbol$ spacing should favour bigger or smaller
elements. One choice which equally favours both elements either side of the
node is for the scale factors to be chosen to be nodally based and equal to the average \arclen on either side
of the node for each $\elementcoordinatesymbol$ direction \ie for the $\nth{l}$ direction
\begin{equation}
  \nsftwo{A}{l}=\dfrac{\fnof{\arclengthcoordinate{l}}{\fnof{A_{\ominus}}{l}}+ 
    \fnof{\arclengthcoordinate{l}}{\fnof{A_{\oplus}}{l}}}{2}
  \label{eqn:arithmeanarclenscale}
\end{equation}
where $\nsftwo{A}{l}$ is the nodal scale factor in the $\nth{l}$ $\elementcoordinatesymbol$
direction at global node $A$, $\fnof{A_{\ominus}}{l}$ is the element
immediately preceding (in the \nth{l} direction) node $A$, and
$\fnof{A_{\oplus}}{l}$ is the element immediately after (in the \nth{l}
direction) node $A$ and $\fnof{\arclengthcoordinate{l}}{e}$ is the \arclen in the \nth{l} $\elementcoordinatesymbol$
direction from node $A$ in element $e$. This type of scaling is known as
\emph{arithmetic mean \arclen scaling}.

Other means can be used \ie \emph{geometric mean \arclen scaling}
\begin{equation}
  \nsftwo{A}{l}=\sqrt{\fnof{\arclengthcoordinate{l}}{\fnof{A_{\ominus}}{l}}
    \fnof{\arclengthcoordinate{l}}{\fnof{A_{\oplus}}{l}}}
  \label{eqn:geomeanarclenscale}
\end{equation}
or \emph{harmonic mean \arclen scaling}
\begin{equation}
  \nsftwo{A}{l}=\dfrac{\fnof{\arclengthcoordinate{l}}{\fnof{A_{\ominus}}{l}}
    \fnof{\arclengthcoordinate{l}}{\fnof{A_{\oplus}}{l}}}{\fnof{\arclengthcoordinate{l}}{\fnof{A_{\ominus}}{l}}+
    \fnof{\arclengthcoordinate{l}}{\fnof{A_{\oplus}}{l}}}
  \label{eqn:harmonicmeanarclenscale}
\end{equation}

Other possible scalings include \emph{minimum \arclen scaling}
\begin{equation}
  \nsftwo{A}{l}=\min\pbrac{\fnof{\arclengthcoordinate{l}}{\fnof{A_{\ominus}}{l}},
    \fnof{\arclengthcoordinate{l}}{\fnof{A_{\oplus}}{l}}}
  \label{eqn:minimumarclenscale}
\end{equation}
and \emph{maximum \arclen scaling}
\begin{equation}
  \nsftwo{A}{l}=\max\pbrac{\fnof{\arclengthcoordinate{l}}{\fnof{A_{\ominus}}{l}},
    \fnof{\arclengthcoordinate{l}}{\fnof{A_{\oplus}}{l}}}
  \label{eqn:maximumarclenscale}
\end{equation}
and \emph{Root Mean Squared (RMS) \arclen scaling}
\begin{equation}
  \nsftwo{A}{l}=\sqrt{\dfrac{\fnof{\arclengthcoordinate{l}}{\fnof{A_{\ominus}}{l}}^{2}+
    \fnof{\arclengthcoordinate{l}}{\fnof{A_{\oplus}}{l}}^{2}}{2}}
  \label{eqn:rmsarclenscale}
\end{equation}

To investigate the effect of different types of scaling consider the following
test problem. Consider a fuction $\fnof{u}{x}=A\cos x$ in a \oned domain from
$x\in\sqbrac{0,L}$. The domain is modelled with a mesh consisting of three
cubic Hermite elements. The first of the four nodes is located at $x=0$, the
second at $x=a$, the third at $x=b$ and the final node at $x=L$. The analytic
first derivative would be $\fnof{u^{'}}{x}=-A\sin x$ and the analytic second
derivative would be $\fnof{u^{''}}{x}=-A\cos x$. The nodal values and
derivatives are set to their analytic values.

As a specific example of this test problem let us first consider unevenly
spaced elements with $L=2\pi$, $A=2$, $a=1.5$ and
$b=3.5$ is given in \Figref{fig:functionscaling} for the function,
\Figref{fig:firstdscaling} for the first derivative and
\Figref{fig:seconddscaling} for the second derivative. Note that the first
derivative is discontinuous at element boundaries for \arclen scaling.

\pstexfigure{BasisFunctions/plots/funcscaling.pstex}{Function scaling for
  uneven element spacing.}{Function scaling for uneven element
  spacing.}{fig:functionscaling}{1.5}
\pstexfigure{BasisFunctions/plots/firstdscaling.pstex}{First derivative
  scaling for uneven element spacing.}{First derivative scaling for uneven
  element spacing.}{fig:firstdscaling}{1.5}
\pstexfigure{BasisFunctions/plots/seconddscaling.pstex}{Second derivative
  scaling for uneven element spacing.}{Second derivative scaling for uneven
  element spacing.}{fig:seconddscaling}{1.5}

Now consider this problem with evenly spaced elements \ie with
$a=\frac{2\pi}{3}$ and $b=\frac{4\pi}{3}$. The interpolations are given in
\Figref{fig:functionscaling2} for the function, \Figref{fig:firstdscaling2}
for the first derivative and \Figref{fig:seconddscaling2} for the second
derivative. Note that the value of the function and the first derivative is
continuous at element boundaries for all scaling types.

\pstexfigure{BasisFunctions/plots/funcscaling2.pstex}{Function scaling for
  even element spacing.}{Function scaling for even element
  spacing.}{fig:functionscaling2}{1.5}
\pstexfigure{BasisFunctions/plots/firstdscaling2.pstex}{First derivative
  scaling for even element spacing.}{First derivative scaling for even element
  spacing.}{fig:firstdscaling2}{1.5}
\pstexfigure{BasisFunctions/plots/seconddscaling2.pstex}{Second derivative
  scaling for even element spacing.}{Second derivative scaling for even
  element spacing.}{fig:seconddscaling2}{1.5}

Let us now consider the influence of scaling type on geometric variables which
require normalised \arclen derivatives. For this problem let us consider
modelling a circle with three cubic Hermite elements. The analytic description
is given by $\fnof{x}{\theta}=A\cos\theta$ and $\fnof{y}{\theta}=A\sin\theta$. And
the derivatives are given by $\delby{\fnof{x}{\theta}}{s}=-\sin\theta$,
$\delby{\fnof{y}{\theta}}{s}=\cos\theta$,
$\deltwosqby{\fnof{x}{\theta}}{s}=-\cos\theta$ and
$\deltwosqby{\fnof{y}{\theta}}{s}=-\sin\theta$. The analytic value of the
\arclen for an element is given by $s=A\Delta\theta$ where $\Delta\theta$ is
the angle subtended by the element. 

First let us consider unevenly spaced elements with $A=2$, $a=1.5$ and
$b=3.5$. The nodal positions and derivatives are set to their analytic values
and the element \arclens are set to their analytic value. The interpolation is
given in \Figref{fig:geometricscaling} for the function,
\Figref{fig:geomfirstdscaling} for the first derivative and
\Figref{fig:geomseconddscaling} for the second derivative. Note that the first
derivative is discontinuous at element boundaries for \arclen scaling.

\pstexfigure{BasisFunctions/plots/geometricscaling.pstex}{Geometric circle
  scaling for uneven element spacing.}{Geometric circle scaling for uneven
  element spacing.}{fig:geometricscaling}{1.5}
\pstexfigure{BasisFunctions/plots/geomfirstdscaling.pstex}{Geometric circle
  first derivative scaling for uneven element spacing.}{Geometric circle first
  derivative scaling for uneven element spacing.}{fig:geomfirstdscaling}{1.5}
\pstexfigure{BasisFunctions/plots/geomseconddscaling.pstex}{Geometric circle
  second derivative scaling for uneven element spacing.}{Geometric circle
  second derivative scaling for uneven element
  spacing.}{fig:geomseconddscaling}{1.5}

Now consider this problem with evenly spaced elements \ie with
$a=\frac{2\pi}{3}$ and $b=\frac{4\pi}{3}$. The interpolations are given in
\Figref{fig:geometricscaling2} for the function, \Figref{fig:geomfirstdscaling2}
for the first derivative and \Figref{fig:geomseconddscaling2} for the second
derivative. Note that the value of the function and the first derivative is
continuous at element boundaries for all scaling types.

\pstexfigure{BasisFunctions/plots/geometricscaling2.pstex}{Geometric circle
  scaling for even element spacing.}{Geometric circle scaling for even element
  spacing.}{fig:geometricscaling2}{1.5}
\pstexfigure{BasisFunctions/plots/geomfirstdscaling2.pstex}{Geometric circle
  first derivative scaling for even element spacing.}{Geometric circle first
  derivative scaling for even element spacing.}{fig:geomfirstdscaling2}{1.5}
\pstexfigure{BasisFunctions/plots/geomseconddscaling2.pstex}{Geometric circle
  second derivative scaling for even element spacing.}{Geometric circle second
  derivative scaling for even element spacing.}{fig:geomseconddscaling2}{1.5}

\subsubsection{Hermite-sector elements}
\label{sec:hselements}

One problem that arises when using quadrilateral elements (such as
\bicubicherm elements) to describe a surface is that it is impossible to
'close the surface' in three-dimensions whilst maintaining consistent $\elemcoordone$
and $\elemcoordtwo$ directions throughout the mesh. This is important as $C^{1}$
continuity requires either consistent $\elementcoordinatesymbol$ directions or a transformation at
each node to take into account the inconsistent directions \citep{petera:1994}.

One solution to this problem is to \emph{collapse} a \bicubicherm element.
This entails placing one of the four local nodes of the element at the same
geometric location as another local node of the element and results in a
triangular element from which it is possible to close the surface. There are
two main problems with this solution.  The first is that one of the two $\elementcoordinatesymbol$
directions at the collapsed node is undefined.  The second is that the
distance between the two nodes at the same location is zero.  Numerical
problems can result from this zero distance.  An alternative strategy has
been developed in which special elements, called ``Hermite-sector''
elements\index{Hermite-sector elements}, are used to close a \bicubicherm
surface in three-dimensions. There are two types of elements depending on
whether the $\elementcoordinatesymbol$ (or $\arclengthcoordinatesymbol$) directions come together at local node one or local
node three.  These two elements are shown in \figref{fig:hermitesectors}.

\epstexfigure{BasisFunctions/svgs/hermitesectors.eps_tex}{Hermite-sector elements.}
{Hermite-sector elements. (a) Apex node one element. (b) Apex node three
  element.}{fig:hermitesectors}{0.3}

From \figref{fig:hermitesectors} it can be seen that the $\arclengthcoordinate{2}$ direction is
not unique at the apex nodes. This gives us two choices for the interpolation
within the element: ignore the $\arclengthcoordinate{2}$ derivative when interpolating or set
the $\arclengthcoordinate{2}$ derivative identically to zero.

\textbf{Ignore $\arclengthcoordinate{2}$ apex derivative}: For this case it can be seen from
\figref{fig:hermitesectors} that the interpolation in the $\elemcoordone$ direction
is just the standard cubic Hermite interpolation. The interpolation in the
$\elemcoordtwo$ direction is now a little different in that the nodal \arclen
derivative has been dropped as it is no longer defined at the apex node.  For
an apex node one element shown in \figref{fig:hermitesectors}(a) the
interpolation for the line between local node one and local node $n$ is now
quadratic and is given by
\begin{equation}
  \fnof{\coordinatevector}{\elemcoordtwo}=\hsonebfn{1}{\elemcoordtwo}\nodept{\coordinatevector}{1}+
  \hsonebfn{2}{\elemcoordtwo}\nodept{\coordinatevector}{n}+
  \hsonebfn{3}{\elemcoordtwo}\evalat{\delby{\coordinatevector}{\elemcoordtwo}}{n}
  \label{eqn:hsapex1xiinterp}
\end{equation}
with the basis functions given by
\index{quadratic Hermite basis!apex node one!basis functions formulae}
\begin{equation}
  \begin{split}
    \hsonebfn{1}{\xi}&=\pbrac{\xi-1}^{2} \\ 
    \hsonebfn{2}{\xi}&=2\xi-\xi^{2} \\
    \hsonebfn{3}{\xi}&=\xi^{2}-\xi
  \end{split}
  \label{eqn:hsapex1bfuns}
\end{equation}

For the apex node three element shown in \figref{fig:hermitesectors}(b) the
interpolation for the line connecting local node $n$ with local node three is
given by
\begin{equation}
  \fnof{\coordinatevector}{\elemcoordtwo}=\hsthreebfn{1}{\elemcoordtwo}\nodept{\coordinatevector}{3}+
  \hsthreebfn{2}{\elemcoordtwo}\nodept{\coordinatevector}{n}+
  \hsthreebfn{3}{\elemcoordtwo}\evalat{\delby{\coordinatevector}{\elemcoordtwo}}{n}
  \label{eqn:hsapex3xiinterp}
\end{equation}
with the basis functions given by
\index{quadratic Hermite basis!apex node three!basis functions formulae}
\begin{equation}
  \begin{split}
    \hsthreebfn{1}{\xi}&=\xi^{2} \\ 
    \hsthreebfn{2}{\xi}&=1-\xi^{2} \\ 
    \hsthreebfn{3}{\xi}&=\xi-\xi^{2}
  \end{split}
  \label{eqn:hsapex3Bfuns}
\end{equation}
 
The full interpolation formula for the sector element can then be found by
taking the tensor product of the interpolation in the $\elemcoordone$ direction,
given in \eqnref{eqn:chxiinterpolation}, with the interpolation in the
$\elemcoordtwo$ direction (given by either Equations \bref{eqn:hsapex1xiinterp} or
\bref{eqn:hsapex3xiinterp}). The interpolation formula can be converted from
nodal $\elementcoordinatesymbol$ derivatives to nodal \arclen derivatives using the procedure
outlined for the \bicubicherm case. For example, the interpolation formulae for
an apex node one element is \index{Hermite-sector basis!apex node one!\arclen
  interpolation formula}
\begin{equation}
  \begin{split}
    \fnof{\coordinatevector}{\elemcoordone,\elemcoordtwo} &=
    \hsonebfn{1}{\elemcoordtwo}\nodept{\coordinatevector}{1}+\chbfn{1}{0}{\elemcoordone}\hsonebfn{2}
    {\elemcoordtwo}\nodept{\coordinatevector}{2}+\chbfn{2}{0}{\elemcoordone}\hsonebfn{2}{\elemcoordtwo}
    \nodept{\coordinatevector}{3} + \\
    & \chbfn{1}{1}{\elemcoordone}\hsonebfn{2}{\elemcoordtwo}\delby{\nodept{\coordinatevector}{2}}{\arclengthcoordinate{1}}
    \nsftwo{2}{1}+\chbfn{2}{1}{\elemcoordone}\hsonebfn{2}{\elemcoordtwo}\delby{\nodept{\coordinatevector}
      {3}}{\arclengthcoordinate{1}}\nsftwo{3}{1} + \\
    & \chbfn{1}{0}{\elemcoordone}\hsonebfn{3}{\elemcoordtwo}\delby{\nodept{\coordinatevector}{2}}{\arclengthcoordinate{2}}
    \nsftwo{2}{2} + \chbfn{2}{0}{\elemcoordone}\hsonebfn{3}{\elemcoordtwo}
    \delby{\nodept{\coordinatevector}{3}}{\arclengthcoordinate{2}}\nsftwo{3}{2} + \\
    & \chbfn{1}{1}{\elemcoordone}\hsonebfn{3}{\elemcoordtwo}\deltwoby{\nodept{\coordinatevector}{2}}
    {\arclengthcoordinate{1}}{\arclengthcoordinate{2}}\nsftwo{2}{1}\nsftwo{2}{2} + 
    \chbfn{2}{1}{\elemcoordone}\hsonebfn{3}{\elemcoordtwo}\deltwoby{\nodept{\coordinatevector}{3}}
    {\arclengthcoordinate{1}}{\arclengthcoordinate{2}}\nsftwo{3}{1}\nsftwo{3}{2}    
  \end{split}
  \label{eqn:hsapex1sinterp}
\end{equation}

Care must be taken when using Hermite-sector elements for rapidly changing
surfaces. Consider an apex node one element with undefined $\arclengthcoordinate{2}$ apex
derivatives. The rate of change of $\coordinatevector$ with respect to
$\elemcoordone$ along the line from node one to node three (\ie $\elemcoordone=1$) is
\begin{equation}
  \begin{split}
    \delby{\fnof{\coordinatevector}{1,\elemcoordtwo}}{\elemcoordone} &= \hsonebfn{2}{\elemcoordtwo}\delby{
      \nodept{\coordinatevector}{3}}{\arclengthcoordinate{1}}\nsftwo{3}{1}+\hsonebfn{3}{\elemcoordtwo}\deltwoby{
      \nodept{\coordinatevector}{3}}{\arclengthcoordinate{1}}{\arclengthcoordinate{2}}\nsftwo{3}{1}\nsftwo{3}{2} \\
    &= \nsftwo{3}{1}\pbrac{\pbrac{2\elemcoordtwo-\elemcoordtwosq}\delby{\nodept{\coordinatevector}{3}}
      {\arclengthcoordinate{1}}+\pbrac{\elemcoordtwosq-\elemcoordtwo}\deltwoby{\nodept{\coordinatevector}{3}}{\arclengthcoordinate{1}}{\arclengthcoordinate{2}}
      \nsftwo{3}{2}}
  \end{split}
\end{equation}

Taking the dot product of $\delby{\fnof{\coordinatevector}{1,\elemcoordtwo}}{\elemcoordone}$ with 
$\delby{\nodept{\coordinatevector}{3}}{\arclengthcoordinate{1}}$ gives
\begin{equation}
  \dotprod{\delby{\fnof{\coordinatevector}{1,\elemcoordtwo}}{\elemcoordone}}{\delby{\nodept{\coordinatevector}{3}}
    {\arclengthcoordinate{1}}} = \nsftwo{3}{1}
  \pbrac{\pbrac{2\elemcoordtwo-\elemcoordtwosq}\dotprod{\delby{\nodept{\coordinatevector}{3}}{\arclengthcoordinate{1}}}
    {\delby{\nodept{\coordinatevector}{3}}{\arclengthcoordinate{1}}}+\pbrac{\elemcoordtwosq-\elemcoordtwo}\nsftwo{3}{2}
    \dotprod{\deltwoby{\nodept{\coordinatevector}{3}}{\arclengthcoordinate{1}}{\arclengthcoordinate{2}}}
    {\delby{\nodept{\coordinatevector}{3}}{\arclengthcoordinate{1}}}}
  \label{eqn:hsonedirectiondotprod}
\end{equation}

The normality constraint for \arclen derivatives means that
$\dotprod{\delby{\nodept{\coordinatevector}{3}}{\arclengthcoordinate{1}}}{\delby{\nodept{\coordinatevector}{3}}
  {\arclengthcoordinate{1}}}=1$ and thus the right hand side of
\eqnref{eqn:hsonedirectiondotprod} divided by $\nsftwo{3}{1}$ (\ie normalised
by $\nsftwo{3}{1}$) is the quadratic
\begin{equation*}
  \pbrac{2\elemcoordtwo-\elemcoordtwosq}+\pbrac{\elemcoordtwosq-\elemcoordtwo}\nsftwo{3}{2}
  \dotprod{\deltwoby{\nodept{\coordinatevector}{3}}{\arclengthcoordinate{1}}{\arclengthcoordinate{2}}}{\delby{\nodept{\coordinatevector}
      {3}}{\arclengthcoordinate{1}}} 
\end{equation*}
or
\begin{equation*}
  \pbrac{\nsftwo{3}{2}\dotprod{\deltwoby{\nodept{\coordinatevector}{3}}{\arclengthcoordinate{1}}{\arclengthcoordinate{2}}}
    {\delby{\nodept{\coordinatevector}{3}}{\arclengthcoordinate{1}}} -1}\elemcoordtwosq+
  \pbrac{2-\nsftwo{3}{2}\dotprod{\deltwoby{\nodept{\coordinatevector}{3}}{\arclengthcoordinate{1}}{\arclengthcoordinate{2}}}
    {\delby{\nodept{\coordinatevector}{3}}{\arclengthcoordinate{1}}}}\elemcoordtwo
  \label{eqn:hsonedirectionpolynomial}
\end{equation*}

This quadratic is $1$ at $\elemcoordtwo=1$ and always has a root at $\elemcoordtwo=0$.
Consider the case of this quadratic having its second root in the interval
$(0,1)$. This would mean that at some point in the interval $(0,1)$ the dot
product of \delby{\fnof{\coordinatevector}{1,\elemcoordtwo}}{\elemcoordone} and
\delby{\nodept{\coordinatevector}{3}}{\arclengthcoordinate{1}} would go from zero to negative and then
positive as $\elemcoordtwo$ changed from $0$ to $1$ \ie the angle between
$\delby{\fnof{\coordinatevector}{1,\elemcoordtwo}}{\elemcoordone}$ and
$\delby{\nodept{\coordinatevector}{3}}{\arclengthcoordinate{1}}$ would, at some stage, be greater than
ninety degrees. As the direction of the normal to the surface along the line
between local node one and three is given by the cross product of
$\delby{\fnof{\coordinatevector}{1,\elemcoordtwo}}{\elemcoordone}$ and
$\delby{\fnof{\coordinatevector}{1,\elemcoordtwo}}{\elemcoordtwo}$ then, if the quadratic became
sufficiently negative, the normal to the surface could reverse direction from
an outward to an inward normal as $\elemcoordtwo$ changed from $0$ to $1$. This is
clearly undesirable. In fact even if the quadratic is only slightly negative
the resulting surface would be grossly deformed.

To avoid these effects the second root of the quadratic must be outside the
interval $(0,1)$. From the quadratic formula the conditions for this are
\begin{equation}
  \dfrac{\nsftwo{3}{2}\dotprod{\deltwoby{\nodept{\coordinatevector}{3}}{\arclengthcoordinate{1}}{\arclengthcoordinate{2}}}
    {\delby{\nodept{\coordinatevector}{3}}{\arclengthcoordinate{1}}}-2}{\nsftwo{3}{2}\dotprod{
      \deltwoby{\nodept{\coordinatevector}{3}}{\arclengthcoordinate{1}}{\arclengthcoordinate{2}}}{\delby{\nodept{\coordinatevector}{3}}
      {\arclengthcoordinate{1}}}-1}<0
\end{equation}
and 
\begin{equation}
  \dfrac{\nsftwo{3}{2}\dotprod{\deltwoby{\nodept{\coordinatevector}{3}}{\arclengthcoordinate{1}}{\arclengthcoordinate{2}}}
    {\delby{\nodept{\coordinatevector}{3}}{\arclengthcoordinate{1}}}-2}{\nsftwo{3}{2}\dotprod{
      \deltwoby{\nodept{\coordinatevector}{3}}{\arclengthcoordinate{1}}{\arclengthcoordinate{2}}}{\delby{\nodept{\coordinatevector}{3}}
      {\arclengthcoordinate{1}}}-1}>1
\end{equation}
that is (for the line from local node one to local node $n$) 
\index{Hermite-sector basis!apex node one!cross-derivative condition}
\begin{equation}
  \dotprod{\deltwoby{\nodept{\coordinatevector}{n}}{\arclengthcoordinate{1}}{\arclengthcoordinate{2}}}{\delby{\nodept{\coordinatevector}
      {n}}{\arclengthcoordinate{1}}}<\dfrac{2}{\nsftwo{n}{2}}
\end{equation}

The simplest way to interpret this constraint is that if the element is large
(\ie $\nsftwo{n}{2}$ is large) then $\dotprod{\deltwoby{\nodept{\coordinatevector}{n}}
  {\arclengthcoordinate{1}}{\arclengthcoordinate{2}}}{\delby{\nodept{\coordinatevector}{n}}{\arclengthcoordinate{1}}}$ must be small. The
simplest way for this to happen is to ensure the magnitude of the components of
$\deltwoby{\nodept{\coordinatevector}{n}}{\arclengthcoordinate{1}}{\arclengthcoordinate{2}}$ are small (or of opposite sign to
the comparable components of $\delby{\nodept{\coordinatevector}{n}}{\arclengthcoordinate{1}}$).

The equivalent interpolation formula to \eqnref{eqn:hsapex1sinterp} for an
apex node three Hermite-sector element is 
\index{Hermite-sector basis!apex node three!\arclen interpolation formula}
\begin{equation}
  \begin{split}
    \fnof{\coordinatevector}{\elemcoordone,\elemcoordtwo} &=
    \chbfn{1}{0}{\elemcoordone}\hsthreebfn{2}{\elemcoordtwo}\nodept{\coordinatevector}{1}+
    \chbfn{2}{0}{\elemcoordone}\hsthreebfn{2}{\elemcoordtwo}\nodept{\coordinatevector}{2}+
    \hsthreebfn{1}{\elemcoordtwo}\nodept{\coordinatevector}{3}+ \\
    & \chbfn{1}{1}{\elemcoordone}\hsthreebfn{2}{\elemcoordtwo}\delby{\nodept{\coordinatevector}{1}}
    {\arclengthcoordinate{1}}\nsftwo{1}{1}+\chbfn{2}{1}{\elemcoordone}\hsthreebfn{2}{\elemcoordtwo}
    \delby{\nodept{\coordinatevector}{2}}{\arclengthcoordinate{1}}\nsftwo{2}{1} + \\
    & \chbfn{1}{0}{\elemcoordone}\hsthreebfn{3}{\elemcoordtwo}\delby{\nodept{\coordinatevector}{1}}
    {\arclengthcoordinate{2}}\nsftwo{1}{2}+\chbfn{2}{0}{\elemcoordone}\hsthreebfn{3}{\elemcoordtwo}
    \delby{\nodept{\coordinatevector}{2}}{\arclengthcoordinate{2}}\nsftwo{2}{2} + \\
    & \chbfn{1}{1}{\elemcoordone}\hsthreebfn{3}{\elemcoordtwo}\deltwoby{\nodept{\coordinatevector}{1}}
    {\arclengthcoordinate{1}}{\arclengthcoordinate{2}}\nsftwo{1}{1}\nsftwo{1}{2} + 
    \chbfn{2}{1}{\elemcoordone}\hsthreebfn{3}{\elemcoordtwo}\deltwoby{\nodept{\coordinatevector}{2}}
    {\arclengthcoordinate{1}}{\arclengthcoordinate{2}}\nsftwo{2}{1}\nsftwo{2}{2}    
  \end{split}
  \label{eqn:hsapex3sinterp}
\end{equation}
and the equivalent constraint for apex node three Hermite-sector elements (for
the line from local node $n$ to local node three) is
\index{Hermite-sector basis!apex node three!cross-derivative condition}
\begin{equation}
  \dotprod{\deltwoby{\nodept{\coordinatevector}{n}}{\arclengthcoordinate{1}}{\arclengthcoordinate{2}}}{\delby{\nodept{\coordinatevector}
      {n}}{\arclengthcoordinate{1}}}>\dfrac{-2}{\nsftwo{n}{2}}
\end{equation}

\textbf{Zero $\arclengthcoordinate{2}$ apex derivative}: For this case the sector basis
functions are just the cubic Hermite basis functions. The corresponding
interpolation formulae for an apex node one element is hence
\begin{equation}
  \begin{split}
    \fnof{\coordinatevector}{\elemcoordone,\elemcoordtwo} &= \chbfn{1}{0}{\elemcoordtwo}
    \nodept{\coordinatevector}{1} + 
     \chbfn{1}{0}{\elemcoordone}\chbfn{2}{0}{\elemcoordtwo}\nodept{\coordinatevector}{2} +
    \chbfn{2}{0}{\elemcoordone}\chbfn{2}{0}{\elemcoordtwo}\nodept{\coordinatevector}{3} + \\
    & \chbfn{1}{1}{\elemcoordone}\chbfn{2}{0}{\elemcoordtwo}\delby{\nodept{\coordinatevector}{2}}{\arclengthcoordinate{1}}
    \nsftwo{2}{1}+
    \chbfn{2}{1}{\elemcoordone}\chbfn{2}{0}{\elemcoordtwo}\delby{\nodept{\coordinatevector}{3}}{\arclengthcoordinate{1}}
    \nsftwo{3}{1} + \\ 
    & \chbfn{1}{0}{\elemcoordone}\chbfn{2}{1}{\elemcoordtwo}\delby{\nodept{\coordinatevector}{2}}{\arclengthcoordinate{2}}
    \nsftwo{2}{2}+
    \chbfn{2}{0}{\elemcoordone}\chbfn{2}{1}{\elemcoordtwo}\delby{\nodept{\coordinatevector}{2}}{\arclengthcoordinate{2}}
    \nsftwo{3}{2} + \\ 
    & \chbfn{1}{1}{\elemcoordone}\chbfn{2}{1}{\elemcoordtwo}\deltwoby{\nodept{\coordinatevector}{2}}
      {\arclengthcoordinate{1}}{\arclengthcoordinate{2}}\nsftwo{2}{1}\nsftwo{2}{2} + 
    \chbfn{2}{1}{\elemcoordone}\chbfn{2}{1}{\elemcoordtwo}\deltwoby{\nodept{\coordinatevector}{3}}
      {\arclengthcoordinate{1}}{\arclengthcoordinate{2}}\nsftwo{3}{1}\nsftwo{3}{2}
  \end{split}
\end{equation}
and the condition to avoid reversal of the normal is
\begin{equation}
  \dotprod{\deltwoby{\nodept{\coordinatevector}{n}}{\arclengthcoordinate{1}}{\arclengthcoordinate{2}}}{\delby{\nodept{\coordinatevector}
      {n}}{\arclengthcoordinate{1}}}<\dfrac{3}{\nsftwo{n}{2}}
\end{equation}
and for the apex node three element the interpolation formula is
\begin{equation}
  \begin{split}
    \fnof{\coordinatevector}{\elemcoordone,\elemcoordtwo} &=
    \chbfn{1}{0}{\elemcoordone}\chbfn{2}{0}{\elemcoordtwo}\nodept{\coordinatevector}{1} +
    \chbfn{2}{0}{\elemcoordone}\chbfn{2}{0}{\elemcoordtwo}\nodept{\coordinatevector}{2} + 
    \chbfn{2}{0}{\elemcoordtwo}\nodept{\coordinatevector}{3} + \\
    & \chbfn{1}{1}{\elemcoordone}\chbfn{1}{0}{\elemcoordtwo}\delby{\nodept{\coordinatevector}{1}}{\arclengthcoordinate{1}}
    \nsftwo{1}{1}+
    \chbfn{2}{1}{\elemcoordone}\chbfn{1}{0}{\elemcoordtwo}\delby{\nodept{\coordinatevector}{2}}{\arclengthcoordinate{1}}
    \nsftwo{2}{1} + \\ 
    & \chbfn{1}{0}{\elemcoordone}\chbfn{1}{1}{\elemcoordtwo}\delby{\nodept{\coordinatevector}{1}}{\arclengthcoordinate{2}}
    \nsftwo{1}{2}+
    \chbfn{2}{0}{\elemcoordone}\chbfn{1}{1}{\elemcoordtwo}\delby{\nodept{\coordinatevector}{2}}{\arclengthcoordinate{2}}
    \nsftwo{2}{2} + \\ 
    & \chbfn{1}{1}{\elemcoordone}\chbfn{1}{1}{\elemcoordtwo}\deltwoby{\nodept{\coordinatevector}{1}}
      {\arclengthcoordinate{1}}{\arclengthcoordinate{2}}\nsftwo{1}{1}\nsftwo{1}{2} + 
    \chbfn{2}{1}{\elemcoordone}\chbfn{1}{1}{\elemcoordtwo}\deltwoby{\nodept{\coordinatevector}{2}}
      {\arclengthcoordinate{1}}{\arclengthcoordinate{2}}\nsftwo{2}{1}\nsftwo{2}{2}
  \end{split}
\end{equation}
with a condition of
\begin{equation}
  \dotprod{\deltwoby{\nodept{\coordinatevector}{n}}{\arclengthcoordinate{1}}{\arclengthcoordinate{2}}}{\delby{\nodept{\coordinatevector}
      {n}}{\arclengthcoordinate{1}}}>\dfrac{-3}{\nsftwo{n}{2}}
\end{equation}

Although the Hermite-sector basis function in which the $\arclengthcoordinate{2}$ apex node
derivatives are identically zero have an increased limit on the
cross-derivative constraints (a right hand side numerator of $\pm 3$ instead
of $\pm 2$) they have the problem that as all derivatives vanish at the apex
any interpolated function has a zero Hessian at the apex. As this can cause
numerical problems the Hermite-sector basis functions which have an undefined
$\arclengthcoordinate{2}$ derivative are prefered.

\subsection{Simplex Basis Functions}
\label{subsec:BasisFunctionsSimplex}

A simplex is a generalisation of the idea of a triangle to an
arbitrary number of dimensions. The name simplex derives from the idea
that it is the simplest flat sided geometric shape that can occur in a
given dimensional space.

A $d$ dimensional simplex, $K\subset\rntopology{d}$, is formed from $d+1$ vertices in the
space. Edges are formed between vertices. The family of simplex
elements for different dimensions is shown
in \figref{fig:SimplexFamily}.

\epstexfigure{BasisFunctions/svgs/SimplexFamily.eps_tex}{Family of simplexes for different dimensions.}{Family of simplex elements of dimension $d$ formed by vertices and edges.}{fig:SimplexFamily}{0.75}

If we now consider the matrix $\matr{M}_{d}$ of size
$\rntopology{\pbrac{d+1}\times\pbrac{d+1}}$ formed by column vectors
of the coordinates of the vertices completed by a row of ones \ie
\begin{equation}
\matr{M}_{1}=\begin{bmatrix}
 \nodept{x}{1} & \nodept{x}{2} \\
 1 & 1
\end{bmatrix}\qquad \matr{M}_{2}=\begin{bmatrix}
 \nodept{x}{1} & \nodept{x}{2} & \nodept{x}{3} \\
 \nodept{y}{1} & \nodept{y}{2} & \nodept{y}{3} \\
 1 & 1 & 1
 \end{bmatrix}\qquad \matr{M}_{3}=\begin{bmatrix}
 \nodept{x}{1} & \nodept{x}{2} & \nodept{x}{3} & \nodept{x}{4} \\
 \nodept{y}{1} & \nodept{y}{2} & \nodept{y}{3} & \nodept{y}{4} \\
 \nodept{z}{1} & \nodept{z}{2} & \nodept{z}{3} & \nodept{z}{4} \\
 1 & 1 & 1 & 1
 \end{bmatrix}
\end{equation}
then the volume of the simplex, $V$, is given by
\begin{equation}
V=\determinant{\matr{M}_{d}}=\pm\factorial{d}\abs{K}
\end{equation}
where $\abs{K}$ is the \emph{measure} of the simplex and the
$\pm$ factor is related to the \emph{orientation} of the
simplex.

\subsubsection{Barycentric coordinates}
\label{subsubsec:BasisFunctionsSimplexBarycentricCoords}\

Simplex basis functions are based on \emph{barycentric coordinates},
also called area coordinates. If we consider a $d$ dimensional simplex
the location of a point in space, $\coordinatevector\in\rntopology{d}$, is
given by
\begin{equation}
 \coordinatevector=\gsum{i=1}{d+1}{\fnof{\barycoordinate{i}}{\coordinatevector}\nodept{\coordinatevector}{i}}
\end{equation}
where $\fnof{\barycoordinate{i}}{\coordinatevector}$ is the \nth{i}
barycentric coordinate and $\nodept{\coordinatevector}{i}$ is the location of
the \nth{i} vertex of the simplex.

The barycentric coordinates have the properties that they sum to one \ie
\begin{equation}
  \gsum{i=1}{d+1}{\fnof{\barycoordinate{i}}{\coordinatevector}}=1
\end{equation}
and that they are $1$ at one vertex and $0$ at the other vertices \ie
\begin{equation}
\fnof{\barycoordinate{i}}{\nodept{\coordinatevector}{j}}=\kronecker{i}{j}\qquad i,j=1,\ldots,d+1
\end{equation}

In general the \nth{i} barycentric coordinate can be found from
\begin{equation}
  \fnof{\barycoordinate{i}}{\coordinatevector}=1-
  \dfrac{
    \dotprod{\pbrac{\coordinatevector-\nodept{\coordinatevector}{i}}}{\nodept{\normal}{i}}
  }{ 
    \dotprod{\pbrac{\nodept{\coordinatevector}{f}-\nodept{\coordinatevector}{i}}}{\nodept{\normal}{i}}
  }
\end{equation}
where $\nodept{\normal}{i}$ is the outward normal of the facet
opposite the \nth{i} vertex and $\nodept{\coordinatevector}{f}$ is any
vertex belonging to this facet.

On simplex elements barycentric coordinates are also called area
coordinates as they are related to the relative areas defined areas
formed by the point of interest and the vertices. Consider the \twodal
simplex in \figref{fig:AreaCoordinates}. For a point of interest,
$\coordinatevector$, an area associated with the \nth{i} vertex, can be
formed from the the point of interest and the two other opposite facet
vertices, $\nodept{\coordinatevector}{f}$. The \nth{i} barycentric
coordinate is then found from the ratio of this area to the total area
of the simplex.

\epstexfigure{BasisFunctions/svgs/areacoordinates.eps_tex}{Area coordiantes
  for a quadratic triangular simplex.}{Barycentric or area coordinates
  for a (quadratic) triangular simplex element. For a point of
  interest, $\coordinatevector$, three triangles can be formed from the point
  to two other vertices. The triangle is then associated with the
  vertex opposite the edge associated with the two other vertices. In
  this example the blue triangle is associated with vertiex $1$, the
  green triangle with vertex $2$, and the red triangle with vertex
  $3$. The barycentric coordinate
  $\fnof{\barycoordinate{i}}{\coordinatevector}$ is then given by the ratio
  of area of the triangle associated with the \nth{i} vertex and the
  total area of the simplex (the sum of the blue, green, and red
  areas).}{fig:AreaCoordinates}{0.75}

\subsubsection{Simplex reference element}
\label{subsubsec:BasisFunctionsSimplexReferenceElement}

The reference element for simplex elements can be defined as
in \figref{fig:SimplexReference}. The reference element is defined
with respect to a normalised element coordinate system,
$\elementcoordinatevector$. There is some choice as to how a simplex is mapped to
the reference element coordinate system. For example is the first
vertex mapped to the origin of the reference coordinate or is it the
last vertex? To keep some concistency with the quadralateral reference
elements in \OpenCMISS the simplex reference element of dimension $d$
is defined by mapping vertex 1 to the origin of the elemental
$\elementcoordinatevector$ coordinate system and the \nth{i+1} vertex to location
$1$ on the \nth{i} $\elementcoordinatesymbol$ coordinate.

\epstexfigure{BasisFunctions/svgs/SimplexReference.eps_tex}{Reference elements for simplex elements of different dimensions.}{Mapping of simplex elements of various dimension in the top row to their respective reference elements in the bottom row. The reference elements are defined with respect to normalised elemental coordinates, $\elementcoordinatevector$. Vertex 1 is mapped to the origin of the reference element coordinate system and the \nth{i+1} vertex is mapped to location $1$ on the \nth{i} $\elementcoordinatesymbol$ coordinate.}{fig:SimplexReference}{0.66}

On the reference simplex element the barycentric coordinates are given by
\begin{alignat}{3}
\fnof{\barycoordone}{\elementcoordinatevector}&=1-\gsum{i=1}{d}{\elementcoordinate{i}}& \\
\fnof{\barycoordinate{i+1}}{\elementcoordinatevector}&=\elementcoordinate{i} & i=1,\ldots,d
\end{alignat}

Thus, for simplex line elements there are two area coordinates which are a function of $\elementcoordinatevector=\bbrac{\elemcoordone}$ \ie
\begin{align}
  \fnof{\barycoordone}{\elementcoordinatevector} &= 1-\elemcoordone \\
  \fnof{\barycoordtwo}{\elementcoordinatevector} &= \elemcoordone
\end{align}
and for simplex triangle elements there are three area coordinates which are a function of $\elementcoordinatevector=\bbrac{\elemcoordone,\elemcoordtwo}$ \ie
\begin{align}
  \fnof{\barycoordone}{\elementcoordinatevector} &= 1-\elemcoordone-\elemcoordtwo \\
  \fnof{\barycoordtwo}{\elementcoordinatevector} &= \elemcoordone \\
  \fnof{\barycoordthree}{\elementcoordinatevector} &= \elemcoordtwo
\end{align}
and for simplex tetrahedral elements there are four area coordinates which are a function of $\elementcoordinatevector=\bbrac{\elemcoordone,\elemcoordtwo,\elemcoordthree}$ \ie
\begin{align}
  \fnof{\barycoordone}{\elementcoordinatevector} &= 1-\elemcoordone-\elemcoordtwo-\elemcoordthree \\
  \fnof{\barycoordtwo}{\elementcoordinatevector} &= \elemcoordone \\
  \fnof{\barycoordthree}{\elementcoordinatevector} &= \elemcoordtwo \\
  \fnof{\barycoordfour}{\elementcoordinatevector} &= \elemcoordthree
\end{align}

The interpolation formulae for simplex elements of dimension $d$ is
\begin{equation}
  \fnof{u}{\fnof{\barycoordinatevector}{\elementcoordinatevector}}=\sbfn{i}{\fnof{\barycoordinatevector}{\elementcoordinatevector}}\nodept{u}{i}\qquad i=1,\dots,d+1
\end{equation}

Care must be taken to obtain the derivative of the basis functions with respect to the reference elemental coordinate. For example
\begin{equation}
\begin{aligned}
  \delby{\fnof{u}{\fnof{\barycoordinatevector}{\elemcoordone}}}{\elemcoordone}&=\delby{\pbrac{\sbfn{i}{\fnof{\barycoordinatevector}{\elemcoordone}}\nodept{u}{i}}}{\elemcoordone}\\
  &=\delby{\sbfn{i}{\fnof{\barycoordinatevector}{\elemcoordone}}}{\barycoordinate{j}}\delby{\barycoordinate{j}}{\elemcoordone}\nodept{u}{i}\\
  &=\pbrac{\delby{\sbfn{i}{\fnof{\barycoordinatevector}{\elemcoordone}}}{\barycoordone}\delby{\barycoordone}{\elemcoordone}+ \delby{\sbfn{i}{\fnof{\barycoordinatevector}{\elemcoordone}}}{\barycoordtwo}\delby{\barycoordtwo}{\elemcoordone}}\nodept{u}{i}\\
  &=\pbrac{\delby{\sbfn{i}{\fnof{\barycoordinatevector}{\elemcoordone}}}{\barycoordtwo}-\delby{\sbfn{i}{\fnof{\barycoordinatevector}{\elemcoordone}}}{\barycoordone}}\nodept{u}{i}
\end{aligned} 
\end{equation}

In general, the derivative of the line basis functions vector with
respect to the reference elemental coordinates is given by
\begin{equation}
\begin{aligned}
  \delby{\vect{\sbfnsymb{}}}{\elemcoordone} &=
  -\delby{\vect{\sbfnsymb{}}}{\barycoordone}+\delby{\vect{\sbfnsymb{}}}{\barycoordtwo} \\
  \deltwosqby{\vect{\sbfnsymb{}}}{\elemcoordone} &=
  \deltwosqby{\vect{\sbfnsymb{}}}{\barycoordone}-
  2\deltwoby{\vect{\sbfnsymb{}}}{\barycoordone}{\barycoordtwo}+
  \deltwosqby{\vect{\sbfnsymb{}}}{\barycoordtwo} 
\end{aligned}
\end{equation}

For triangular elements the derivative of the basis functions vector
with respect to the reference elemental coordinates are given by
\begin{align}
  \delby{\vect{\sbfnsymb{}}}{\elemcoordone} &= -\delby{\vect{\sbfnsymb{}}}{\barycoordone}+
  \delby{\vect{\sbfnsymb{}}}{\barycoordtwo} \\
  \delby{\vect{\sbfnsymb{}}}{\elemcoordtwo} &= -\delby{\vect{\sbfnsymb{}}}{\barycoordone}+
  \delby{\vect{\sbfnsymb{}}}{\barycoordthree} \\
  \deltwosqby{\vect{\sbfnsymb{}}}{\elemcoordone} &=
  \deltwosqby{\vect{\sbfnsymb{}}}{\barycoordone}- 
  2\deltwoby{\vect{\sbfnsymb{}}}{\barycoordone}{\barycoordtwo}+
  \deltwosqby{\vect{\sbfnsymb{}}}{\barycoordtwo} \\
  \deltwosqby{\vect{\sbfnsymb{}}}{\elemcoordtwo} &=
  \deltwosqby{\vect{\sbfnsymb{}}}{\barycoordone}- 
  2\deltwoby{\vect{\sbfnsymb{}}}{\barycoordone}{\barycoordthree}+
  \deltwosqby{\vect{\sbfnsymb{}}}{\barycoordthree} \\
  \deltwoby{\vect{\sbfnsymb{}}}{\elemcoordone}{\elemcoordtwo} &=
  \deltwosqby{\vect{\sbfnsymb{}}}{\barycoordone}-
  \deltwoby{\vect{\sbfnsymb{}}}{\barycoordone}{\barycoordtwo}-
  \deltwoby{\vect{\sbfnsymb{}}}{\barycoordone}{\barycoordthree}+
  \deltwoby{\vect{\sbfnsymb{}}}{\barycoordtwo}{\barycoordthree}
\end{align}

For tetrahedral elements the derivative of the basis functions vector
with respect to the reference elemental coordinates are given by
\begin{align}
  \delby{\vect{\sbfnsymb{}}}{\elemcoordone} &=
  -\delby{\vect{\sbfnsymb{}}}{\barycoordone}+
  \delby{\vect{\sbfnsymb{}}}{\barycoordtwo} \\
  \delby{\vect{\sbfnsymb{}}}{\elemcoordtwo} &=
  -\delby{\vect{\sbfnsymb{}}}{\barycoordone}+
  \delby{\vect{\sbfnsymb{}}}{\barycoordthree} \\
  \delby{\vect{\sbfnsymb{}}}{\elemcoordthree} &=
  -\delby{\vect{\sbfnsymb{}}}{\barycoordone}+
  \delby{\vect{\sbfnsymb{}}}{\barycoordfour} \\
  \deltwosqby{\vect{\sbfnsymb{}}}{\elemcoordone} &=
  \deltwosqby{\vect{\sbfnsymb{}}}{\barycoordone}-
  2\deltwoby{\vect{\sbfnsymb{}}}{\barycoordone}{\barycoordtwo}+
  \deltwosqby{\vect{\sbfnsymb{}}}{\barycoordtwo} \\
  \deltwosqby{\vect{\sbfnsymb{}}}{\elemcoordtwo} &=
  \deltwosqby{\vect{\sbfnsymb{}}}{\barycoordone}-
  2\deltwoby{\vect{\sbfnsymb{}}}{\barycoordone}{\barycoordthree}+
  \deltwosqby{\vect{\sbfnsymb{}}}{\barycoordthree} \\
  \deltwosqby{\vect{\sbfnsymb{}}}{\elemcoordthree} &=
  \deltwosqby{\vect{\sbfnsymb{}}}{\barycoordone}-
  2\deltwoby{\vect{\sbfnsymb{}}}{\barycoordone}{\barycoordfour}+
  \deltwosqby{\vect{\sbfnsymb{}}}{\barycoordfour} \\  
  \deltwoby{\vect{\sbfnsymb{}}}{\elemcoordone}{\elemcoordtwo} &=
  \deltwosqby{\vect{\sbfnsymb{}}}{\barycoordone}-
  \deltwoby{\vect{\sbfnsymb{}}}{\barycoordone}{\barycoordtwo}-
  \deltwoby{\vect{\sbfnsymb{}}}{\barycoordone}{\barycoordthree}+
  \deltwoby{\vect{\sbfnsymb{}}}{\barycoordtwo}{\barycoordthree} \\
  \deltwoby{\vect{\sbfnsymb{}}}{\elemcoordone}{\elemcoordthree} &=
  \deltwosqby{\vect{\sbfnsymb{}}}{\barycoordone}-
  \deltwoby{\vect{\sbfnsymb{}}}{\barycoordone}{\barycoordtwo}-
  \deltwoby{\vect{\sbfnsymb{}}}{\barycoordone}{\barycoordfour}+
  \deltwoby{\vect{\sbfnsymb{}}}{\barycoordtwo}{\barycoordfour} \\
  \deltwoby{\vect{\sbfnsymb{}}}{\elemcoordtwo}{\elemcoordthree} &=
  \deltwosqby{\vect{\sbfnsymb{}}}{\barycoordone}-
  \deltwoby{\vect{\sbfnsymb{}}}{\barycoordone}{\barycoordthree}-
  \deltwoby{\vect{\sbfnsymb{}}}{\barycoordone}{\barycoordfour}+
  \deltwoby{\vect{\sbfnsymb{}}}{\barycoordthree}{\barycoordfour} \\
  \delthreeby{\vect{\sbfnsymb{}}}{\elemcoordone}{\elemcoordtwo}{\elemcoordthree} &=
  -\delthreecuby{\vect{\sbfnsymb{}}}{\barycoordone}+
  \deltwodelby{\vect{\sbfnsymb{}}}{\barycoordone}{\barycoordtwo}+
  \deltwodelby{\vect{\sbfnsymb{}}}{\barycoordone}{\barycoordthree}+
  \deltwodelby{\vect{\sbfnsymb{}}}{\barycoordone}{\barycoordfour} \\
   &-\delthreeby{\vect{\sbfnsymb{}}}{\barycoordone}{\barycoordtwo}{\barycoordthree}-
   \delthreeby{\vect{\sbfnsymb{}}}{\barycoordone}{\barycoordtwo}{\barycoordfour}-
   \delthreeby{\vect{\sbfnsymb{}}}{\barycoordone}{\barycoordthree}{\barycoordfour}+
   \delthreeby{\vect{\sbfnsymb{}}}{\barycoordtwo}{\barycoordthree}{\barycoordfour}
\end{align}


\subsubsection{Linear simplex elements}
\label{subsubsec:BasisFunctionsSimplexLinear}

The linear simplex elements are shown in \figref{fig:LinearSimplex}.

\epstexfigure{BasisFunctions/svgs/LinearSimplex.eps_tex}{Linear simplex reference elements.}{Linear simplex reference elements in 1D (left), 2D (middle), and 3D (right).}{fig:LinearSimplex}{0.66}

The linear simplex basis functions for 1D line elements are given by
\begin{equation}
\begin{aligned}
  \sbfn{1}{\fnof{\barycoordinatevector}{\elementcoordinatevector}}&=\barycoordone\\
  &=1-\elemcoordone\\
  \sbfn{2}{\fnof{\barycoordinatevector}{\elementcoordinatevector}}&=\barycoordtwo\\
  &=\elemcoordone
\end{aligned}
\label{eqn:OneDLinearSimplexBasisFunctions}
\end{equation}
and for 2D triangular elements are
\begin{equation}
\begin{aligned}
  \sbfn{1}{\fnof{\barycoordinatevector}{\elementcoordinatevector}}&=\barycoordone\\
  &=1-\elemcoordone-\elemcoordtwo\\
  \sbfn{2}{\fnof{\barycoordinatevector}{\elementcoordinatevector}}&=\barycoordtwo\\
  &=\elemcoordone\\
  \sbfn{3}{\fnof{\barycoordinatevector}{\elementcoordinatevector}}&=\barycoordthree\\
  &=\elemcoordtwo
\end{aligned}
\label{eqn:TwoDLinearSimplexBasisFunctions}
\end{equation}
and for 3D tetrahedral elements are
\begin{equation}
\begin{aligned}
  \sbfn{1}{\fnof{\barycoordinatevector}{\elementcoordinatevector}}&=\barycoordone\\
  &=1-\elemcoordone-\elemcoordtwo-\elemcoordthree\\
  \sbfn{2}{\fnof{\barycoordinatevector}{\elementcoordinatevector}}&=\barycoordtwo\\
  &=\elemcoordone\\
  \sbfn{3}{\fnof{\barycoordinatevector}{\elementcoordinatevector}}&=\barycoordthree\\
  &=\elemcoordtwo\\
  \sbfn{4}{\fnof{\barycoordinatevector}{\elementcoordinatevector}}&=\barycoordfour\\
  &=\elemcoordthree
\end{aligned}
\label{eqn:ThreeDLinearSimplexBasisFunctions}
\end{equation}


\begin{figure}[hbtp]
   \centering
   \gnuplotsubfigure{BasisFunctions/gnuplots/linsimplextriangle1.gnuplot}{2D linear simplex basis function $1$.}
     {2D linear simplex basis function $1$, $\sbfn{1}{\elemcoordone,\elemcoordtwo}$.}{subfig:LinearSimplexTriangle1}{0.45\linewidth}{10}{3}{}
   \hfil
   \gnuplotsubfigure{BasisFunctions/gnuplots/linsimplextriangle2.gnuplot}{2D linear simplex basis function $2$.}
     {2D linear simplex basis function $2$, $\sbfn{2}{\elemcoordone,\elemcoordtwo}$.}{subfig:LinearSimplexTriangle2}{0.45\linewidth}{10}{3}{}
   \hfil
   \gnuplotsubfigure{BasisFunctions/gnuplots/linsimplextriangle3.gnuplot}{2D linear simplex basis function $3$.}
     {2D linear simplex basis function $3$, $\sbfn{3}{\elemcoordone,\elemcoordtwo}$.}{subfig:LinearSimplexTriangle3}{0.45\linewidth}{10}{3}{}
   \caption[2D linear simplex basis functions.]{2D linear simplex basis functions.}
   \label{fig:LinearSimplexTriangleBasisFunctions}
\end{figure}
            


\subsubsection{Quadratic simplex elements}
\label{subsubsec:BasisFunctionsSimplexQuadratic}

The quadratic simplex elements are shown in \figref{fig:QuadraticSimplex}.

\epstexfigure{BasisFunctions/svgs/QuadraticSimplex.eps_tex}{Quadratic simplex reference elements.}{Quadratic simplex reference elements in 1D (left), 2D (middle), and 3D (right).}{fig:QuadraticSimplex}{0.66}

The quadratic simplex basis functions for 1D line elements are given by
\begin{equation}
\begin{aligned}
  \sbfn{1}{\fnof{\barycoordinatevector}{\elementcoordinatevector}}&=\barycoordone\pbrac{2\barycoordone-1}=2\barycoordonesq-\barycoordone\\
  &=\pbrac{1-\elemcoordone}\pbrac{2\pbrac{1-\elemcoordone}-1}\\
  &=2\elemcoordonesq-3\elemcoordone+1\\
  \sbfn{2}{\fnof{\barycoordinatevector}{\elementcoordinatevector}}&=4\barycoordone\barycoordtwo \\
  &=4\pbrac{1-\elemcoordone}\elemcoordone\\
  &=-4\elemcoordonesq+4\elemcoordone\\
  \sbfn{3}{\fnof{\barycoordinatevector}{\elementcoordinatevector}}&=\barycoordtwo\pbrac{2\barycoordtwo-1}=2\barycoordtwosq-\barycoordtwo\\
  &=\elemcoordone\pbrac{2\elemcoordone-1}\\
  &=2\elemcoordonesq-2\elemcoordone
\end{aligned}
\label{eqn:OneDQuadraticSimplexBasisFunctions}
\end{equation}
and for 2D triangular elements are
\begin{equation}
\begin{aligned}
  \sbfn{1}{\fnof{\barycoordinatevector}{\elementcoordinatevector}}&=\barycoordone\pbrac{2\barycoordone-1}=2\barycoordonesq-\barycoordone\\
  &=\pbrac{1-\elemcoordone-\elemcoordtwo}\pbrac{2\pbrac{1-\elemcoordone-\elemcoordtwo}-1}\\
  &=2\pbrac{\elemcoordonesq+\elemcoordtwosq}+4\elemcoordone\elemcoordtwo-3\pbrac{\elemcoordone+\elemcoordtwo}+1\\
  \sbfn{2}{\fnof{\barycoordinatevector}{\elementcoordinatevector}}&=\barycoordtwo\pbrac{2\barycoordtwo-1}=2\barycoordtwosq-\barycoordtwo\\
  &=\elemcoordone\pbrac{2\elemcoordone-1}\\
  &=2\elemcoordonesq-\elemcoordone\\
  \sbfn{3}{\fnof{\barycoordinatevector}{\elementcoordinatevector}}&=\barycoordthree\pbrac{2\barycoordthree-1}=2\barycoordthreesq-\barycoordthree\\
  &=\elemcoordtwo\pbrac{2\elemcoordtwo-1}\\
  &=2\elemcoordtwosq-\elemcoordtwo\\
  \sbfn{4}{\fnof{\barycoordinatevector}{\elementcoordinatevector}}&=4\barycoordone\barycoordtwo\\
  &=\pbrac{1-\elemcoordone-\elemcoordtwo}\elemcoordone\\
  &=-\elemcoordonesq-\elemcoordone\elemcoordtwo+\elemcoordone\\
  \sbfn{5}{\fnof{\barycoordinatevector}{\elementcoordinatevector}}&=4\barycoordtwo\barycoordthree\\
  &=\elemcoordone\elemcoordtwo\\
  \sbfn{6}{\fnof{\barycoordinatevector}{\elementcoordinatevector}}&=4\barycoordone\barycoordthree\\
  &=\pbrac{1-\elemcoordone-\elemcoordtwo}\elemcoordtwo\\
  &=-\elemcoordtwosq-\elemcoordone\elemcoordtwo+\elemcoordtwo
\end{aligned}
\label{eqn:TwoDQuadraticSimplexBasisFunctions}
\end{equation}
and for 3D tetrahedral elements are
\begin{equation}
\begin{aligned}
  \sbfn{1}{\fnof{\barycoordinatevector}{\elementcoordinatevector}}&=\barycoordone\pbrac{2\barycoordone-1}=2\barycoordonesq-\barycoordone\\
  &=\pbrac{1-\elemcoordone-\elemcoordtwo-\elemcoordthree}\pbrac{2\pbrac{1-\elemcoordone-\elemcoordtwo-\elemcoordthree}-1}\\
  &=2\pbrac{\elemcoordonesq+\elemcoordtwosq+\elemcoordthreesq}+4\pbrac{\elemcoordone\elemcoordtwo+\elemcoordtwo\elemcoordthree+\elemcoordone\elemcoordthree}-3\pbrac{\elemcoordone+\elemcoordtwo+\elemcoordthree}+1\\
  \sbfn{2}{\fnof{\barycoordinatevector}{\elementcoordinatevector}}&=\barycoordtwo\pbrac{2\barycoordtwo-1}=2\barycoordtwosq-\barycoordtwo\\
  &=\elemcoordone\pbrac{2\elemcoordone-1}\\
  &=2\elemcoordonesq-\elemcoordone\\
  \sbfn{3}{\fnof{\barycoordinatevector}{\elementcoordinatevector}}&=\barycoordthree\pbrac{2\barycoordthree-1}=2\barycoordthreesq-\barycoordthree\\
  &=\elemcoordtwo\pbrac{2\elemcoordtwo-1}\\
  &=2\elemcoordtwosq-\elemcoordtwo\\
  \sbfn{4}{\fnof{\barycoordinatevector}{\elementcoordinatevector}}&=\barycoordfour\pbrac{2\barycoordfour-1}=2\barycoordfoursq-\barycoordfour\\
  &=\elemcoordthree\pbrac{2\elemcoordthree-1}\\
  &=2\elemcoordthreesq-\elemcoordthree\\
  \sbfn{5}{\fnof{\barycoordinatevector}{\elementcoordinatevector}}&=\barycoordone\barycoordtwo\\
  &=\pbrac{1-\elemcoordone-\elemcoordtwo-\elemcoordthree}\elemcoordone\\
  &=-\elemcoordonesq-\elemcoordone\elemcoordtwo-\elemcoordone\elemcoordtwo+\elemcoordone\\
  \sbfn{6}{\fnof{\barycoordinatevector}{\elementcoordinatevector}}&=\barycoordone\barycoordthree\\
  &=\pbrac{1-\elemcoordone-\elemcoordtwo-\elemcoordthree}\elemcoordtwo\\
  &=-\elemcoordtwosq-\elemcoordone\elemcoordtwo-\elemcoordtwo\elemcoordthree+\elemcoordtwo\\
  \sbfn{7}{\fnof{\barycoordinatevector}{\elementcoordinatevector}}&=\barycoordone\barycoordfour\\
  &=\pbrac{1-\elemcoordone-\elemcoordtwo-\elemcoordthree}\elemcoordthree\\
  &=-\elemcoordthreesq-\elemcoordone\elemcoordthree-\elemcoordtwo\elemcoordthree+\elemcoordthree\\
  \sbfn{8}{\fnof{\barycoordinatevector}{\elementcoordinatevector}}&=\barycoordtwo\barycoordthree\\
  &=\elemcoordone\elemcoordtwo\\
  \sbfn{9}{\fnof{\barycoordinatevector}{\elementcoordinatevector}}&=\barycoordthree\barycoordfour\\
  &=\elemcoordtwo\elemcoordthree\\
  \sbfn{10}{\fnof{\barycoordinatevector}{\elementcoordinatevector}}&=\barycoordtwo\barycoordfour\\
  &=\elemcoordone\elemcoordthree
\end{aligned}
\label{eqn:ThreeDQuadraticSimplexBasisFunctions}
\end{equation}
 
\subsubsection{Cubic simplex elements}
\label{subsubsec:BasisFunctionsSimplexCubic}

The cubic simplex elements are shown in \figref{fig:CubicSimplex}.

\epstexfigure{BasisFunctions/svgs/CubicSimplex.eps_tex}{Cubic simplex reference elements.}{Cubic simplex reference elements in 1D (left), 2D (middle), and 3D (right).}{fig:CubicSimplex}{0.66}

The cubic simplex basis functions for 1D line elements are given by
\begin{equation}
\begin{aligned}
  \sbfn{1}{\fnof{\barycoordinatevector}{\elementcoordinatevector}}&=\frac{1}{2}\barycoordone\pbrac{3\barycoordone-1}\pbrac{3\barycoordone-2}=\frac{1}{2}\pbrac{9\barycoordonecube-9\barycoordonesq+2\barycoordone}\\
  &=\frac{1}{2}\pbrac{1-\elemcoordone}\pbrac{3\pbrac{1-\elemcoordone}-1}\pbrac{3\pbrac{1-\elemcoordone}-2}\\
  &=\frac{1}{2}\pbrac{-9\elemcoordonecube+18\elemcoordonesq-11\elemcoordone+2}\\
  \sbfn{2}{\fnof{\barycoordinatevector}{\elementcoordinatevector}}&=\frac{9}{2}\barycoordone\pbrac{3\barycoordone-1}\barycoordtwo=\frac{9}{2}\pbrac{3\barycoordonesq\barycoordtwo-\barycoordone\barycoordtwo}\\
  &=\frac{9}{2}\pbrac{1-\elemcoordone}\pbrac{3\pbrac{1-\elemcoordone}-1}\elemcoordone\\
  &=\frac{9}{2}\pbrac{3\elemcoordonecube-5\elemcoordonesq+2\elemcoordone}\\
  \sbfn{3}{\fnof{\barycoordinatevector}{\elementcoordinatevector}}&=\frac{9}{2}\barycoordone\pbrac{3\barycoordtwo-1}\barycoordtwo=\frac{9}{2}\pbrac{3\barycoordone\barycoordtwosq-\barycoordone\barycoordtwo}\\
  &=\frac{9}{2}\pbrac{1-\elemcoordone}\pbrac{3\elemcoordone-1}\elemcoordone\\
  &=\frac{9}{2}\pbrac{-3\elemcoordonesq+4\elemcoordonesq-\elemcoordone}\\
  \sbfn{4}{\fnof{\barycoordinatevector}{\elementcoordinatevector}}&=\frac{1}{2}\barycoordtwo\pbrac{3\barycoordtwo-1}\pbrac{3\barycoordtwo-2}=\frac{1}{2}\pbrac{9\barycoordtwocube-9\barycoordonesq+2\barycoordtwo}\\
  &=\frac{1}{2}\elemcoordone\pbrac{3\elemcoordone-1}\pbrac{3\elemcoordone-2}\\
  &=\frac{1}{2}\pbrac{9\elemcoordonecube-9\elemcoordonesq+2\elemcoordone}
\end{aligned}
\label{eqn:OneDCubicSimplexBasisFunctions}
\end{equation}
and for 2D triangle elements are
\footnotesize
\begin{equation}
\begin{aligned}
  \sbfn{1}{\fnof{\barycoordinatevector}{\elementcoordinatevector}}&=\frac{1}{2}\barycoordone\pbrac{3\barycoordone-1}\pbrac{3\barycoordone-2}=\frac{1}{2}\pbrac{9\barycoordonecube-9\barycoordonesq+2\barycoordone}\\
  &=\frac{1}{2}\pbrac{1-\elemcoordone-\elemcoordtwo}\pbrac{3\pbrac{1-\elemcoordone-\elemcoordtwo}-1}\pbrac{3\pbrac{1-\elemcoordone-\elemcoordtwo}-2}\\
  &=\frac{1}{2}\pbrac{-9\pbrac{\elemcoordonecube+\elemcoordtwocube}-27\pbrac{\elemcoordonesq\elemcoordtwo+\elemcoordone\elemcoordtwosq}+18\pbrac{\elemcoordonesq+\elemcoordtwosq}+36\elemcoordone\elemcoordtwo-8\pbrac{\elemcoordone+\elemcoordtwo}+2}\\
  \sbfn{2}{\fnof{\barycoordinatevector}{\elementcoordinatevector}}&=\frac{1}{2}\barycoordtwo\pbrac{3\barycoordtwo-1}\pbrac{3\barycoordtwo-2}=\frac{1}{2}\pbrac{9\barycoordtwocube-9\barycoordtwosq+2\barycoordtwo}\\
  &=\frac{1}{2}\elemcoordone\pbrac{3\elemcoordone-1}\pbrac{3\elemcoordone-2}\\
  &=\frac{1}{2}\pbrac{9\elemcoordonecube-9\elemcoordonesq+2\elemcoordone}\\
  \sbfn{3}{\fnof{\barycoordinatevector}{\elementcoordinatevector}}&=\frac{1}{2}\barycoordthree\pbrac{3\barycoordthree-1}\pbrac{3\barycoordthree-2}=\frac{1}{2}\pbrac{9\barycoordthreecube-9\barycoordthreesq+2\barycoordthree}\\
  &=\frac{1}{2}\elemcoordtwo\pbrac{3\elemcoordtwo-1}\pbrac{3\elemcoordtwo-2}\\
  &=\frac{1}{2}\pbrac{9\elemcoordtwocube-9\elemcoordtwosq+2\elemcoordtwo}\\
  \sbfn{4}{\fnof{\barycoordinatevector}{\elementcoordinatevector}}&=\frac{9}{2}\barycoordone\pbrac{3\barycoordone-1}\barycoordtwo=\frac{9}{2}\pbrac{3\barycoordonesq\barycoordtwo-\barycoordone\barycoordtwo}\\
  &=\frac{9}{2}\pbrac{1-\elemcoordone-\elemcoordtwo}\pbrac{3\pbrac{1-\elemcoordone-\elemcoordtwo}-1}\elemcoordone\\
  &=\frac{9}{2}\pbrac{3\elemcoordonecube+6\elemcoordonesq\elemcoordtwo+3\elemcoordone\elemcoordtwosq-5\elemcoordonesq-5\elemcoordone\elemcoordtwo+2\elemcoordone}\\
  \sbfn{5}{\fnof{\barycoordinatevector}{\elementcoordinatevector}}&=\frac{9}{2}\barycoordone\pbrac{3\barycoordtwo-1}\barycoordtwo=\frac{9}{2}\pbrac{3\barycoordone\barycoordtwosq-\barycoordone\barycoordtwo}\\
  &=\frac{9}{2}\pbrac{1-\elemcoordone-\elemcoordtwo}\pbrac{3\elemcoordone-1}\elemcoordone\\
  &=\frac{9}{2}\pbrac{-3\pbrac{\elemcoordonesq\elemcoordtwo+\elemcoordone\elemcoordtwosq}+\elemcoordtwosq+4\elemcoordone\elemcoordtwo-\elemcoordtwo}\\
  \sbfn{6}{\fnof{\barycoordinatevector}{\elementcoordinatevector}}&=\frac{9}{2}\barycoordtwo\pbrac{3\barycoordtwo-1}\barycoordthree=\frac{9}{2}\pbrac{3\barycoordtwosq\barycoordthree-\barycoordtwo\barycoordthree}\\
  &=\frac{9}{2}\elemcoordone\pbrac{3\elemcoordone-1}\elemcoordtwo\\
  &=\frac{9}{2}\pbrac{3\elemcoordonesq\elemcoordtwo-\elemcoordone\elemcoordtwo}\\
  \sbfn{7}{\fnof{\barycoordinatevector}{\elementcoordinatevector}}&=\frac{9}{2}\barycoordtwo\pbrac{3\barycoordthree-1}\barycoordthree=\frac{9}{2}\pbrac{3\barycoordtwo\barycoordthreesq-\barycoordtwo\barycoordthree}\\
  &=\frac{9}{2}\elemcoordone\pbrac{3\elemcoordtwo-1}\elemcoordtwo\\
  &=\frac{9}{2}\pbrac{3\elemcoordone\elemcoordtwosq-\elemcoordone\elemcoordtwo}\\
  \sbfn{8}{\fnof{\barycoordinatevector}{\elementcoordinatevector}}&=\frac{9}{2}\barycoordone\pbrac{3\barycoordthree-1}\barycoordthree=\frac{9}{2}\pbrac{3\barycoordone\barycoordthreesq-\barycoordone\barycoordthree}\\
  &=\frac{9}{2}\pbrac{1-\elemcoordone-\elemcoordtwo}\pbrac{3\elemcoordtwo-1}\elemcoordtwo\\
  &=\frac{9}{2}\pbrac{-3\elemcoordtwocube-3\elemcoordone\elemcoordtwosq+4\elemcoordtwosq+\elemcoordone\elemcoordtwo-\elemcoordtwo}\\
  \sbfn{9}{\fnof{\barycoordinatevector}{\elementcoordinatevector}}&=\frac{9}{2}\barycoordone\pbrac{3\barycoordone-1}\barycoordthree=\frac{9}{2}\pbrac{3\barycoordonesq\barycoordthree-\barycoordone\barycoordthree}\\
  &=\frac{9}{2}\pbrac{1-\elemcoordone-\elemcoordtwo}\pbrac{3\pbrac{1-\elemcoordone-\elemcoordtwo}-1}\elemcoordtwo\\
  &=\frac{9}{2}\pbrac{3\elemcoordtwocube+3\elemcoordonesq\elemcoordtwo+6\elemcoordone\elemcoordtwosq-5\elemcoordtwosq-5\elemcoordone\elemcoordtwo+2\elemcoordtwo}\\
  \sbfn{10}{\fnof{\barycoordinatevector}{\elementcoordinatevector}}&=27\barycoordone\barycoordtwo\barycoordthree\\
  &=27\pbrac{1-\elemcoordone-\elemcoordtwo}\elemcoordone\elemcoordtwo\\
  &=-\elemcoordonesq\elemcoordtwo-\elemcoordone\elemcoordtwosq+\elemcoordone\elemcoordtwo  
\end{aligned}
\label{eqn:TwoDCubicSimplexBasisFunctions}
\end{equation}
\normalsize
and for 3D tetrahedral elements are
\footnotesize
\begin{equation}
\begin{aligned}
  \sbfn{1}{\fnof{\barycoordinatevector}{\elementcoordinatevector}}&=\frac{1}{2}\barycoordone\pbrac{3\barycoordone-1}\pbrac{3\barycoordone-2}=\frac{1}{2}\pbrac{9\barycoordonecube-9\barycoordonesq+2\barycoordone}\\
  &=\frac{1}{2}\pbrac{1-\elemcoordone-\elemcoordtwo-\elemcoordthree}\pbrac{3\pbrac{1-\elemcoordone-\elemcoordtwo-\elemcoordthree}-1}\pbrac{3\pbrac{1-\elemcoordone-\elemcoordtwo-\elemcoordthree}-2}\\
  &=\frac{1}{2}\left(-9\pbrac{\elemcoordonecube+\elemcoordtwocube+\elemcoordthreecube}-27\pbrac{\elemcoordonesq\elemcoordtwo+\elemcoordone\elemcoordtwosq+\elemcoordtwosq\elemcoordthree+\elemcoordtwo\elemcoordthreesq+\elemcoordonesq\elemcoordthree+\elemcoordone\elemcoordthreesq}\right.\\
  &\qquad\left.-54\elemcoordone\elemcoordtwo\elemcoordthree+18\pbrac{\elemcoordonesq+\elemcoordtwosq+\elemcoordthreesq}+36\pbrac{\elemcoordone\elemcoordtwo+\elemcoordtwo\elemcoordthree+\elemcoordone\elemcoordthree}-11\pbrac{\elemcoordone+\elemcoordtwo+\elemcoordthree}+2\right)\\
  \sbfn{2}{\fnof{\barycoordinatevector}{\elementcoordinatevector}}&=\frac{1}{2}\barycoordtwo\pbrac{3\barycoordtwo-1}\pbrac{3\barycoordtwo-2}=\frac{1}{2}\pbrac{9\barycoordtwocube-9\barycoordtwosq+2\barycoordtwo}\\
  &=\frac{1}{2}\elemcoordone\pbrac{3\elemcoordone-1}\pbrac{3\elemcoordone-2}\\
  &=\frac{1}{2}\pbrac{9\elemcoordonecube-9\elemcoordonesq+2\elemcoordone}\\
  \sbfn{3}{\fnof{\barycoordinatevector}{\elementcoordinatevector}}&=\frac{1}{2}\barycoordthree\pbrac{3\barycoordthree-1}\pbrac{3\barycoordthree-2}=\frac{1}{2}\pbrac{9\barycoordthreecube-9\barycoordthreesq+2\barycoordthree}\\
  &=\frac{1}{2}\elemcoordtwo\pbrac{3\elemcoordtwo-1}\pbrac{3\elemcoordtwo-2}\\
  &=\frac{1}{2}\pbrac{9\elemcoordtwocube-9\elemcoordtwosq+2\elemcoordone}\\
  \sbfn{4}{\fnof{\barycoordinatevector}{\elementcoordinatevector}}&=\frac{1}{2}\barycoordfour\pbrac{3\barycoordfour-1}\pbrac{3\barycoordfour-2}=\frac{1}{2}\pbrac{9\barycoordfourcube-9\barycoordfoursq+2\barycoordfour}\\  
  &=\frac{1}{2}\elemcoordthree\pbrac{3\elemcoordthree-1}\pbrac{3\elemcoordthree-2}\\
  &=\frac{1}{2}\pbrac{9\elemcoordthreecube-9\elemcoordthreesq+2\elemcoordthree}\\
  \sbfn{5}{\fnof{\barycoordinatevector}{\elementcoordinatevector}}&=\frac{9}{2}\barycoordone\pbrac{3\barycoordone-1}\barycoordtwo=\frac{9}{2}\pbrac{3\barycoordonesq\barycoordtwo-\barycoordone\barycoordtwo}\\
  &=\frac{9}{2}\pbrac{1-\elemcoordone-\elemcoordtwo-\elemcoordthree}\pbrac{3\pbrac{1-\elemcoordone-\elemcoordtwo-\elemcoordthree}-1}\elemcoordone\\
  \sbfn{6}{\fnof{\barycoordinatevector}{\elementcoordinatevector}}&=\frac{9}{2}\barycoordone\pbrac{3\barycoordtwo-1}\barycoordtwo=\frac{9}{2}\pbrac{3\barycoordone\barycoordtwosq-\barycoordone\barycoordtwo}\\
  &=\frac{9}{2}\pbrac{1-\elemcoordone-\elemcoordtwo-\elemcoordthree}\pbrac{3\elemcoordone-1}\pbrac{3\elemcoordone-2}\\
  \sbfn{7}{\fnof{\barycoordinatevector}{\elementcoordinatevector}}&=\frac{9}{2}\barycoordone\pbrac{3\barycoordone-1}\barycoordthree=\frac{9}{2}\pbrac{3\barycoordonesq\barycoordthree-\barycoordone\barycoordthree}\\
  &=\frac{9}{2}\pbrac{1-\elemcoordone-\elemcoordtwo-\elemcoordthree}\pbrac{3\pbrac{1-\elemcoordone-\elemcoordtwo-\elemcoordthree}-1}\elemcoordtwo\\
  \sbfn{8}{\fnof{\barycoordinatevector}{\elementcoordinatevector}}&=\frac{9}{2}\barycoordone\pbrac{3\barycoordthree-1}\barycoordthree=\frac{9}{2}\pbrac{3\barycoordone\barycoordthreesq-\barycoordone\barycoordthree}\\
  &=\frac{9}{2}\pbrac{1-\elemcoordone-\elemcoordtwo-\elemcoordthree}\pbrac{3\elemcoordtwo-1}\elemcoordtwo\\
  \sbfn{9}{\fnof{\barycoordinatevector}{\elementcoordinatevector}}&=\frac{9}{2}\barycoordone\pbrac{3\barycoordone-1}\barycoordfour=\frac{9}{2}\pbrac{3\barycoordonesq\barycoordfour-\barycoordone\barycoordfour}\\
  &=\frac{9}{2}\pbrac{1-\elemcoordone-\elemcoordtwo-\elemcoordthree}\pbrac{3\pbrac{1-\elemcoordone-\elemcoordtwo-\elemcoordthree}-1}\elemcoordthree\\
  \sbfn{10}{\fnof{\barycoordinatevector}{\elementcoordinatevector}}&=\frac{9}{2}\barycoordone\pbrac{3\barycoordfour-1}\barycoordfour=\frac{9}{2}\pbrac{3\barycoordone\barycoordfoursq-\barycoordone\barycoordfour}\\
  &=\frac{9}{2}\pbrac{1-\elemcoordone-\elemcoordtwo-\elemcoordthree}\pbrac{3\elemcoordthree-1}\elemcoordthree
\end{aligned}
\label{eqn:ThreeDCubicSimplexBasisFunctions1}
\end{equation}
\normalsize
and
\footnotesize
\begin{equation}
\begin{aligned}
  \sbfn{11}{\fnof{\barycoordinatevector}{\elementcoordinatevector}}&=\frac{9}{2}\barycoordtwo\pbrac{3\barycoordtwo-1}\barycoordthree=\frac{9}{2}\pbrac{3\barycoordtwosq\barycoordthree-\barycoordtwo\barycoordthree}\\
  &=\frac{9}{2}\elemcoordone\pbrac{3\elemcoordone-1}\elemcoordtwo\\
  \sbfn{12}{\fnof{\barycoordinatevector}{\elementcoordinatevector}}&=\frac{9}{2}\barycoordtwo\pbrac{3\barycoordthree-1}\barycoordthree=\frac{9}{2}\pbrac{3\barycoordtwo\barycoordthreesq-\barycoordtwo\barycoordthree}\\
  &=\frac{9}{2}\elemcoordone\pbrac{3\elemcoordtwo-1}\elemcoordtwo\\
  \sbfn{13}{\fnof{\barycoordinatevector}{\elementcoordinatevector}}&=\frac{9}{2}\barycoordthree\pbrac{3\barycoordthree-1}\barycoordfour=\frac{9}{2}\pbrac{3\barycoordthreesq\barycoordfour-\barycoordthree\barycoordfour}\\
  &=\frac{9}{2}\elemcoordtwo\pbrac{3\elemcoordtwo-1}\elemcoordthree\\
  \sbfn{14}{\fnof{\barycoordinatevector}{\elementcoordinatevector}}&=\frac{9}{2}\barycoordthree\pbrac{3\barycoordfour-1}\barycoordfour=\frac{9}{2}\pbrac{3\barycoordthree\barycoordfoursq-\barycoordthree\barycoordfour}\\
  &=\frac{9}{2}\elemcoordtwo\pbrac{3\elemcoordthree-1}\elemcoordthree\\
  \sbfn{15}{\fnof{\barycoordinatevector}{\elementcoordinatevector}}&=\frac{9}{2}\barycoordtwo\pbrac{3\barycoordtwo-1}\barycoordfour=\frac{9}{2}\pbrac{3\barycoordtwosq\barycoordfour-\barycoordtwo\barycoordfour}\\
  &=\frac{9}{2}\elemcoordone\pbrac{3\elemcoordone-1}\elemcoordthree\\
  \sbfn{16}{\fnof{\barycoordinatevector}{\elementcoordinatevector}}&=\frac{9}{2}\barycoordtwo\pbrac{3\barycoordfour-1}\barycoordfour=\frac{9}{2}\pbrac{3\barycoordtwo\barycoordfoursq-\barycoordtwo\barycoordfour}\\
  &=\frac{9}{2}\elemcoordone\pbrac{3\elemcoordthree-1}\elemcoordthree\\
  \sbfn{17}{\fnof{\barycoordinatevector}{\elementcoordinatevector}}&=27\barycoordone\barycoordtwo\barycoordthree\\
  &=27\pbrac{1-\elemcoordone-\elemcoordtwo-\elemcoordthree}\elemcoordone\elemcoordtwo\\
  \sbfn{18}{\fnof{\barycoordinatevector}{\elementcoordinatevector}}&=27\barycoordone\barycoordtwo\barycoordfour\\
  &=27\pbrac{1-\elemcoordone-\elemcoordtwo-\elemcoordthree}\elemcoordone\elemcoordthree\\
  \sbfn{19}{\fnof{\barycoordinatevector}{\elementcoordinatevector}}&=27\barycoordone\barycoordthree\barycoordfour\\
  &=27\pbrac{1-\elemcoordone-\elemcoordtwo-\elemcoordthree}\elemcoordtwo\elemcoordthree\\
  \sbfn{20}{\fnof{\barycoordinatevector}{\elementcoordinatevector}}&=27\barycoordtwo\barycoordthree\barycoordfour\\
  &=27\elemcoordone\elemcoordtwo\elemcoordthree
\end{aligned}
\label{eqn:ThreeDCubicSimplexBasisFunctions2}
\end{equation}
\normalsize

\subsection{General linear map}

From the definition of Christoffel symbols
\begin{equation}
  \deltwoby{x^{p}}{\elementcoordinate{j}}{\elementcoordinate{k}}=\christoffel{m}{k}{j}\delby{x^{p}}{\elementcoordinate{m}}
\end{equation}
\ie
\begin{equation}
  \christoffel{i}{k}{j}=\deltwoby{x^{l}}{\elementcoordinate{k}}{\elementcoordinate{j}}\delby{\elementcoordinate{i}}{x^{l}}
\end{equation}
or in vector form
\begin{equation}
  \christoffel{i}{j}{k}=\dotprod{\deltwoby{\coordinatevector}{\elementcoordinate{k}}{\elementcoordinate{j}}}{\gradient{}{\elementcoordinate{i}}}
\end{equation}

